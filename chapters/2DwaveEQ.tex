\chapter{The 2-D Wave Equation With Staggered Leapfrog}
\label{Lab:15}
 \index{Wave equation!two dimensions}
 \index{Two-dimensional wave equation}

\labsection{Two dimensional grids}
\index{Two-dimensional grids}
\index{Grids!two-dimensional}

\marginfig{Figures/f06Grids2d}{Two types of 2-D grids.}

In this lab we will do problems in two spatial dimensions, $x$ and
$y$, so we need to spend a little time thinking about how to
represent 2-D grids.  For a simple rectangular grid where all of the
cells are the same size, 2-D grids are pretty straightforward. We
just divide the $x$-dimension into equally sized regions and the
$y$-dimension into equally sized regions, and the two one dimensional
grids intersect to create rectangular cells.  Then we put grid points
either at the corners of the cells (cell-edge) or at the centers of
the cells (cell-centered).  On a cell-center grid we'll usually want
ghost point outside the region of interest so we can get the boundary
conditions right.

\begin{enumerate}
\probtwo \label{P:15.1} Numpy has a nice way of representing
    rectangular two-dimensional grids using the {\tt meshdgrid}
    command.\index{meshdgrid} Let's take a minute to
    learn how to make surface plots using this command.
    Consider a 2-d rectangle defined by $x \in [a,b]$ and $y
    \in [c,d]$. To create a 30-point cell-edge grid in $x$ and
    a 50-point cell-edge grid in $y$ with $a=0$, $b=2$, $c=-1$,
    $d=3$, we use the following code:
\begin{Verbatim}
from numpy import linspace,meshgrid
a = 0
b = 2
Nx = 30
x,dx = linspace(a,b,Nx,retstep = True)

c = 1
d = 3
Ny = 50
y,dy = linspace(c,d,Ny,retstep = True)


X,Y=meshgrid(x,y);
\end{Verbatim}
    \begin{enumerate}

\subprob Put the code fragment above in a script and run
    it to create the 2-D grid. Examine the contents of {\tt
    X} and {\tt Y} thoroughly enough that you can explain
    what the {\tt meshgrid} command does.  See the first
    section of chapter 9 in \emph{Introduction to Python}
    for help understanding the indexing.


\subprob Using this 2-D grid, evaluate the following function
    of $x$ and $y$:
    \begin{equation}
        f(x,y) = e^{(-(x^2+y^2))} \cos{(5 \sqrt{x^2+y^2})}
    \end{equation}

    Use Python's \texttt{plot\_surface} or \texttt{plot\_wireframe}
    command to make a surface plot of this function. \marginfig{Figures/f06p1}{Plot
      from Problem~\ref{P:15.1}} Then properly label the $x$ and $y$
    axes with the symbols $x$ and $y$, to get a plot like
    Fig.~\ref{Figures/f06p1}.
\end{enumerate}
\end{enumerate}
\ifsolutions
\textit{Solution:}\\
\begin{codeexample}
\begin{VerbatimOut}{\listingFile}
from numpy import linspace, meshgrid,exp,sqrt,cos
from matplotlib import pyplot
from mpl_toolkits.mplot3d import Axes3D

a = 0
b = 2
c = -3
d = 3
Nx = 30
Ny = 50
x,dx = linspace(a,b,Nx,retstep = True)
y,dy = linspace(c,d,Ny,retstep = True)

X,Y = meshgrid(x,y)

z = exp(-(X**2 + Y**2)) * cos(5 * sqrt(X**2 + Y**2))
fig = pyplot.figure()
ax = fig.gca(projection='3d')
ax.plot_surface(X,Y,z)
pyplot.show()
\end{VerbatimOut}
\end{codeexample}
\else
\noindent\rule{5 in}{0.01 in}
\fi


\labsection{The two-dimensional wave equation}


The wave equation for transverse waves on a rubber sheet is
\footnote{N.\ Asmar, {\it Partial Differential Equations and Boundary
Value Problems} (Prentice Hall, New Jersey, 2000), p. 129-134.}
\begin{equation}\label{eq:2dwave}
    \mu {\partial^2 z \over \partial t^2}= \sigma \left(
    {\partial^2 z \over \partial x^2} +
    {\partial^2 z \over \partial y^2} \right)
\end{equation}
In this equation $\mu$ is the surface mass density of the sheet, with
units of mass/area. The quantity $\sigma$ is the surface tension,
which has rather odd units. By inspecting the equation above you can
find that $\sigma$ has units of force/length, which doesn't seem
right for a surface. But it is, in fact, correct as you can see by
performing the following thought experiment. Cut a slit of length $L$
in the rubber sheet and think about how much force you have to exert
to pull the lips of this slit together. Now imagine doubling $L$;
doesn't it seem that you should have to pull twice as hard to close
the slit? Well, if it doesn't, it should; the formula for this
closing force is given by $\sigma L$, which defines the meaning of
$\sigma$.

We can solve the two-dimensional wave equation using the same
staggered leapfrog techniques that we used for the one-dimensional
case, except now we need to use a two dimensional grid to represent
$z(x,y,t)$.  We'll use the notation $z_{j,k}^n = z(x_j,y_k,t_n)$ to
represent the function values.  With this notation, the derivatives
can be approximated as
\begin{equation}
    \frac{\partial^2 z}{\partial t^2} \approx
    \frac{z_{j,k}^{n+1} - 2 z_{j,k}^{n} + z_{j,k}^{n-1}}{\tau^2}
\end{equation}
\begin{equation}
    \frac{\partial^2 z}{\partial x^2} \approx
    \frac{z_{j+1,k}^{n} - 2 z_{j,k}^{n} + z_{j-1,k}^{n}}{h_x^2}
\end{equation}
\begin{equation}
    \frac{\partial^2 z}{\partial y^2} \approx
    \frac{z_{j,k+1}^{n} - 2 z_{j,k}^{n} + z_{j,k-1}^{n}}{h_y^2}
\end{equation}
where $h_x$ and $h_y$ are the grid spacings in the $x$ and $y$
dimensions.  We insert these three equations into
Eq.~\eqref{eq:2dwave} to get an expression that we can solve for $z$
at the next time (i.e. $z_{j,k}^{n+1}$).  Then we use this expression
along with the discrete version of the initial velocity condition
\begin{equation}\label{eq:tDeriv}
    v_0(x_j,y_k) \approx \frac{z_{j,k}^{n+1} - z_{j,k}^{n-1}}{2 \tau}
\end{equation}
to find an expression for the initial value of $z_{j,k}^{n-1}$
(i.e. {\tt zold}) so we can get things started.

\marginfig[-1in]{Figures/f06p2}{A wave on a rubber sheet with fixed
edges.}

\begin{enumerate}
\probtwo \label{P:15.2}
\begin{enumerate}
\subprob Derive the staggered leapfrog algorithm for the case
    of square cells with $h_x = h_y = h$. You should arrive at the
    following equation:
\begin{equation}
z_{j,k}^{n+1} = 2 z_{j,k}^n - z_{j,k}^{n-1} + \frac{\sigma \tau^2}{\mu
h^2} \left( z_{j+1,k}^n - 4 z_{j,k}^n + z_{j-1,k}^n + z_{j,k+1}^n + z_{j,k-1}^n\right)
\end{equation}

\subprob Notice that to find the state of the wave at a future moment,
you are going to need the state of the wave at two previous moments.
Use equation \eqref{eq:tDeriv} together with your result from part (a)
to find an expression for \texttt{zold}. You've done this several
times before so this should not feel too foreign to you. You should
find that:
\begin{equation}
z_{j,k}^{-1} = z_{j,k}^0 - v_{j,k}^0 \tau + \frac{\sigma \tau^2}{2\mu
h^2} \left( z_{j+1,k}^0 - 4 z_{j,k}^0 + z_{j-1,k}^0 + z_{j,k+1}^0 + z_{j,k-1}^0\right)
\end{equation}

\subprob Write a Python script that animates the solution of the two
dimensional wave equation on a square region that is
$[-5,5]\times[-5,5]$ and that has fixed edges. Use a
\textbf{cell-edge} square grid with the edge-values pinned to
zero. Choose $\sigma=2$~N/m and $\mu=0.3$~kg/m$^2$ and use a
displacement initial condition that is a Gaussian pulse with zero
velocity
    \begin{equation}
        z(x,y,0) = e^{-5(x^2+y^2)}
        \label{sheetpulse}
    \end{equation}
    This initial condition doesn't strictly satisfy the
    boundary conditions, so you should pin the edges to zero.
%Define index variables {\tt j} and {\tt k} like this
%\begin{Verbatim}
%    j = 2:Nx+1;
%    k = 2:Ny+1;
%\end{Verbatim}
%so you can write the code cleanly like this:
%\begin{Verbatim}
%znew(j,k)= ...  z(j-1,k) + z(j+1,k) ...
%\end{Verbatim}

    Run the simulation long enough that you see the effect of
    repeated reflections from the edges.  

\subprob You will find that this two-dimensional problem has
    a Courant condition similar to the one-dimensional case,
    but with a factor out front:
    \begin{equation}
        \tau < f {h \over c}
    \end{equation}
    where $c = \sqrt{\sigma \over \mu}$.  Determine the value of the
    constant $f$ by numerical experimentation. (Try various values of
    $\tau$ and discover where the boundary is between numerical
    stability and instability.)

\subprob Also watch what happens at the center of the sheet
    by making a plot of $z(0,0,t)$ there. In one dimension
    the pulse propagates away from its initial position
    making that point quickly come to rest with $z=0$. This
    also happens for the three-dimensional wave equation. But
    something completely different happens in two (and
    higher) even dimensions; you should be able to see it in
    your plot by looking at the behavior of $z(0,0,t)$ before
    the first reflection comes back.


\subprob Finally, change the initial conditions so that the
    sheet is initially flat but with the initial velocity
    given by the Gaussian pulse of Eq.~(\ref{sheetpulse}). In
    one dimension when you pulse the system like this the
    string at the point of application of the pulse moves up
    and stays up until the reflection comes back from the
    ends of the system. Does the same thing happen in
    the middle of the sheet when you apply this initial
    velocity pulse? Answer this question by looking at your
    plot of $z(0,0,t)$. You should find that the
    two-dimensional wave equation behaves very differently
    from the one-dimensional wave equation.
\end{enumerate}
\end{enumerate}
\ifsolutions
\textit{Solution:}\\
\begin{codeexample}
\begin{VerbatimOut}{\listingFile}




class TwoDWave():

    def __init__(self,a,b,c,d,N,mu,sigma,tau,tMax):
        from numpy import linspace,meshgrid,sqrt
        self.tau = tau
        self.mu = mu
        self.sigma = sigma
        self.N = N
        self.tMax = tMax
        self.c = sqrt(sigma/mu)
        x,self.dx = linspace(a,b,N,retstep = True)
        y,self.dy = linspace(c,d,N,retstep = True)
        self.X,self.Y = meshgrid(x,y)
        print(self.dx)
        print( "Courant condition: ", self.dx/self.c/sqrt(2))
        if self.tau > 1/( sqrt(2)) *  self.dx/self.c:
            print(self.tau)
            print(1/( sqrt(2)) *  self.dx/self.c)
            print("Instability headed your way!")
            import sys
            sys.exit()

    def initializeWave(self):
        from numpy import zeros_like,exp
        
        self.z = exp(-5 *(self.X**2 + self.Y**2))
        self.v = zeros_like(self.z)
        self.z[0] = 0
        self.z[-1] = 0
        self.z[:,0] = 0
        self.z[:,-1] = 0
    def animate(self,movie=True):
        from numpy import zeros_like,copy
        from matplotlib import pyplot,cm
        from mpl_toolkits.mplot3d import Axes3D
        
        constant = self.sigma * self.tau**2 /(self.mu * self.dx**2)
        #        constant = self.c**2 * self.tau**2 /self.dx**2
        zOld = zeros_like(self.z)
        zOld[1:self.N - 1,1:self.N - 1] = - self.tau * self.v[1:self.N - 1,1:self.N - 1] + self.z[1:self.N -1, 1:self.N -1] + constant/2 * (self.z[2:self.N,1:self.N - 1] - 2 * self.z[1:self.N - 1,1:self.N - 1] + self.z[0:self.N - 2,1:self.N - 1] + self.z[1:self.N - 1,2:self.N] - 2 * self.z[1:self.N - 1,1:self.N - 1] + self.z[1:self.N - 1,0:self.N - 2])
        zOld[0] = 0
        zOld[-1] = 0
        zOld[:,0] = 0
        zOld[0,:] = 0

        counter = 0
        t = 0
        fig = pyplot.figure()
        self.singlePoint = []
        while t < self.tMax:
            zNew = zeros_like(self.z)
            zNew[1:self.N - 1,1:self.N - 1] = 2 * self.z[1:self.N - 1,1:self.N - 1] - zOld[1:self.N - 1, 1: self.N - 1] + constant * (self.z[2:self.N,1:self.N - 1] - 2 * self.z[1:self.N - 1,1:self.N - 1] + self.z[0:self.N - 2,1:self.N - 1] + self.z[1:self.N - 1,2:self.N] - 2 * self.z[1:self.N - 1,1:self.N - 1] + self.z[1:self.N - 1,0:self.N - 2])
            zNew[0] = 0
            zNew[-1] = 0
            zNew[:,0] = 0
            zNew[0,:] = 0
            self.singlePoint.append(self.z[int(self.N/2),int(self.N/2)])
            if counter %50 == 0 and movie:
                ax = fig.gca(projection='3d')
                ax.plot_surface(self.X,self.Y,self.z,cmap=cm.coolwarm)
                ax.set_zlim(-.5,.5)
                pyplot.draw()
                pyplot.pause(.0001)
                pyplot.clf()
                
            zOld = copy(self.z)
            self.z = copy(zNew)
            t += self.tau
            counter += 1
    def plotSinglePoint(self):

        from matplotlib import pyplot
        from numpy import linspace
        pyplot.plot(linspace(0,self.tMax,len(self.singlePoint)),self.singlePoint)
        pyplot.show()

a = -5
b = 5
c = -5
d = 5
N = 100
mu = 0.3
sigma = 2
tau = .005
tMax = 50
my2Dwave = TwoDWave(a,b,c,d,N,mu,sigma,tau,tMax)
my2Dwave.initializeWave()
my2Dwave.animate()
my2Dwave.plotSinglePoint()

#from numpy import linspace, meshgrid,exp,sqrt,cos
#from matplotlib import pyplot
#from mpl_toolkits.mplot3d import Axes3D
\end{VerbatimOut}
\end{codeexample}
\else
\noindent\rule{5 in}{0.01 in}
\fi

\labsection{Homework}
\begin{enumerate}
\prob Consider the same square membrane from problem \ref{P:15.2} but
instead of a constant mass density, let's use the following function:
\begin{equation}
\mu(x,y) = 2.3 - 0.2 x - 0.2 y
\end{equation}
Modify your code to model the motion of this membrane.  Think about
how you will calculate your Courant condition.
\end{enumerate}
\ifsolutions
\textit{Solution:}\\
\begin{codeexample}
\begin{VerbatimOut}{\listingFile}




class TwoDWave():

    def __init__(self,a,b,c,d,N,mu,sigma,tau,tMax):
        from numpy import linspace,meshgrid,sqrt,min
        self.tau = tau
        self.sigma = sigma
        self.N = N
        self.tMax = tMax
        x,self.dx = linspace(a,b,N,retstep = True)
        y,self.dy = linspace(c,d,N,retstep = True)
        self.X,self.Y = meshgrid(x,y)
        self.mu = 2.3 - 0.2 * self.X - 0.2 * self.Y
        self.c = sqrt(sigma/self.mu)

        if self.tau > 1/( sqrt(2)) *  self.dx/min(self.c):
            print(self.tau)
            print(1/( sqrt(2)) *  self.dx/min(self.c))
            print("Instability headed your way!")
            import sys
            sys.exit()

    def initializeWave(self):
        from numpy import zeros_like,exp
        
        self.z = exp(-5 *(self.X**2 + self.Y**2))
        self.v = zeros_like(self.z)
        self.z[0] = 0
        self.z[-1] = 0
        self.z[:,0] = 0
        self.z[:,-1] = 0
    def animate(self,movie=True):
        from numpy import zeros_like,copy
        from matplotlib import pyplot,cm
        from mpl_toolkits.mplot3d import Axes3D
        
        constant = self.sigma * self.tau**2 /(self.mu * self.dx**2)
        #        constant = self.c**2 * self.tau**2 /self.dx**2
        zOld = zeros_like(self.z)
        zOld[1:self.N - 1,1:self.N - 1] = - self.tau * self.v[1:self.N - 1,1:self.N - 1] + self.z[1:self.N -1, 1:self.N -1] + constant[1:self.N -1,1:self.N -1] /2 * (self.z[2:self.N,1:self.N - 1] - 2 * self.z[1:self.N - 1,1:self.N - 1] + self.z[0:self.N - 2,1:self.N - 1] + self.z[1:self.N - 1,2:self.N] - 2 * self.z[1:self.N - 1,1:self.N - 1] + self.z[1:self.N - 1,0:self.N - 2])
        zOld[0] = 0
        zOld[-1] = 0
        zOld[:,0] = 0
        zOld[0,:] = 0

        counter = 0
        t = 0
        fig = pyplot.figure()
        self.singlePoint = []
        while t < self.tMax:
            zNew = zeros_like(self.z)
            zNew[1:self.N - 1,1:self.N - 1] = 2 * self.z[1:self.N - 1,1:self.N - 1] - zOld[1:self.N - 1, 1: self.N - 1] + constant[1:self.N -1,1:self.N -1] * (self.z[2:self.N,1:self.N - 1] - 2 * self.z[1:self.N - 1,1:self.N - 1] + self.z[0:self.N - 2,1:self.N - 1] + self.z[1:self.N - 1,2:self.N] - 2 * self.z[1:self.N - 1,1:self.N - 1] + self.z[1:self.N - 1,0:self.N - 2])
            zNew[0] = 0
            zNew[-1] = 0
            zNew[:,0] = 0
            zNew[0,:] = 0
            self.singlePoint.append(self.z[int(self.N/2),int(self.N/2)])
            if counter %50 == 0 and movie:
                ax = fig.gca(projection='3d')
                ax.plot_surface(self.X,self.Y,self.z,cmap=cm.coolwarm)
                ax.set_zlim(-0.3,.3)
                pyplot.draw()
                pyplot.pause(.0001)
                pyplot.clf()
                
            zOld = copy(self.z)
            self.z = copy(zNew)
            t += self.tau
            counter += 1
    def plotSinglePoint(self):

        from matplotlib import pyplot
        from numpy import linspace
        pyplot.plot(linspace(0,self.tMax,len(self.singlePoint)),self.singlePoint)
        pyplot.show()

a = -5
b = 5
c = -5
d = 5
N = 200
mu = 0.3
sigma = 2
tau = .005
tMax = 50
my2Dwave = TwoDWave(a,b,c,d,N,mu,sigma,tau,tMax)
my2Dwave.initializeWave()
my2Dwave.animate()
my2Dwave.plotSinglePoint()

\end{VerbatimOut}
\end{codeexample}
\else
\noindent\rule{5 in}{0.01 in}
\fi

\labsection{Elliptic, hyperbolic, and parabolic PDEs and their boundary conditions}
\index{Elliptic equations}
\index{Parabolic equations}
\index{Hyperbolic equations}
\index{Partial differential equations, types}

Now let's take a step back and look at some general concepts
related to solving partial differential equations. Probably the
three most famous PDEs of classical physics are
\begin{enumerate}
  \item[(i)] Poisson's equation for the electrostatic potential $V(x,y)$ given the charge density
$\rho(x,y)$
\begin{equation}
    {\partial^2 V \over \partial x^2} +
    {\partial^2 V \over \partial y^2}
    = {-\rho \over \epsilon_0}
    ~~~~+~{\rm Boundary~~Conditions}
\end{equation}


\item[(ii)] The wave equation for the wave displacement $y(x,t)$
\begin{equation}
    {\partial^2 y \over \partial x^2} - {1 \over c^2}
    {\partial^2 y \over \partial t^2} =0
    ~~~~+~{\rm Boundary~~Conditions}
\end{equation}


\item[(iii)] The thermal diffusion equation for the temperature
    distribution $T(x,t)$ in a medium with diffusion coefficient $D$
\begin{equation}
    {\partial T \over \partial t} = D
    {\partial^2 T \over \partial x^2}
    ~~~~+~{\rm Boundary~~Conditions}
    \label{diffusion}
\end{equation}

\end{enumerate}
To this point in the course, we've focused mostly on the wave
equation, but over the next several labs we'll start to tackle
some of the other PDEs.

Mathematicians have special names for these three types of partial
differential equations, and people who study numerical methods often
use these names, so let's discuss them a bit. The three names are
{\it elliptic}, {\it hyperbolic}, and {\it parabolic}. You can
remember which name goes with which of the equations above by
remembering the classical formulas for these conic sections:
\begin{align}
    {\rm ellipse:~~~~}& {x^2 \over a^2} + {y^2 \over b^2}=1
    \\
    {\rm hyperbola:~~~~}& {x^2 \over a^2} - {y^2 \over b^2}=1
    \\
    {\rm parabola:~~~~}& y=a x^2
\end{align}
Compare these equations with the classical PDE's above and make sure
you can use their resemblances to each other to remember the
following rules: Poisson's equation is elliptic, the wave equation
is hyperbolic, and the diffusion equation is parabolic. These names
are important because each different type of equation requires a
different type of algorithm and boundary conditions. Fortunately,
because you are physicists and have developed some intuition about
the physics of these three partial differential equations, you can
remember the proper boundary conditions by thinking about physical
examples instead of memorizing theorems. And in case you haven't
developed this level of intuition, here is a brief review of the
matter.

\index{Boundary conditions!PDEs}

Elliptic equations require the same kind of boundary conditions as
Poisson's equation: $V(x,y)$ specified on all of the surfaces
surrounding the region of interest. Since we will be talking about
time-dependence in the hyperbolic and parabolic cases, notice that
there is no time delay in electrostatics. When all of the bounding
voltages are specified, Poisson's equation says that $V(x,y)$ is
determined instantly throughout the region surrounded by these
bounding surfaces. Because of the finite speed of light this is
incorrect, but Poisson's equation is a good approximation to use in
problems where things happen slowly compared to the time it takes
light to cross the computing region.

To understand hyperbolic boundary conditions, think about a guitar
string described by the transverse displacement function $y(x,t)$.
It makes sense to give end conditions at the two ends of the string,
but it makes no sense to specify conditions at both $t=0$ and
$t=t_\mathrm{final}$ because we don't know the displacement in the future.
This means that you can't pretend that $(x,t)$ are like $(x,y)$ in
Poisson's equation and use ``surrounding''-type boundary conditions.
But we can see the right thing to do by thinking about what a guitar
string does. With the end positions specified, the motion of the
string is determined by giving it an initial displacement $y(x,0)$
and an initial velocity $\partial y(x,t) / \partial t |_{t=0}$, and
then letting the motion run until we reach the final time. So for
hyperbolic equations the proper boundary conditions are to specify
end conditions on $y$ as a function of time and to specify the
initial conditions $y(x,0)$ and $\partial y(x,t)/\partial t|_{t=0}$.

Parabolic boundary conditions are similar to hyperbolic ones, but
with one difference. Think about a thermally-conducting bar with its
ends held at fixed temperatures. Once again, surrounding-type
boundary conditions are inappropriate because we don't want to
specify the future. So as in the hyperbolic case, we can specify
conditions at the ends of the bar, but we also want to give initial
conditions at $t=0$. For thermal diffusion we specify the initial
temperature $T(x,0)$, but that's all we need; the ``velocity''
$\partial T / \partial t$ is determined by Eq.~(\ref{diffusion}), so
it makes no sense to give it as a separate boundary condition.
Summarizing: for parabolic equations we specify end conditions and a
single initial condition $T(x,0)$ rather than the two required by
hyperbolic equations.

If this seems like an arcane side trip into theory, we're sorry, but
it's important. When you numerically solve partial differential
equations you will spend 10\% of your time coding the equation
itself and 90\% of your time trying to make the boundary conditions
work. It's important to understand what the appropriate boundary
conditions are.

Finally, there are many more partial differential equations in
physics than just these three. Nevertheless, if you clearly
understand these basic cases you can usually tell what boundary
conditions to use when you encounter a new one. \index{Schr\"{o}dinger
equation} Here, for instance, is Schr\"{o}dinger's equation:
\begin{equation}
    i \hbar {\partial \psi \over \partial t} = -{\hbar^2 \over 2 m}
    {\partial^2 \psi \over \partial x^2} + V \psi
\end{equation}
which is the basic equation of quantum (or ``wave'') mechanics. The
wavy nature of the physics described by this equation might lead you
to think that the proper boundary conditions on $\psi(x,t)$ would be
hyperbolic: end conditions on $\psi$ and initial conditions on
$\psi$ and $\partial \psi / \partial t$. But if you look at the form
of the equation, it looks like thermal diffusion. Looks are not
misleading here; to solve this equation you only need to specify
$\psi$ at the ends in $x$ and the initial distribution $\psi(x,0)$,
but not its time derivative.

And what are you supposed to do when your system is both hyperbolic
and parabolic, like the wave equation with damping?
\begin{equation}
    {\partial^2 y \over \partial x^2} - {1 \over c^2}
    {\partial^2 y \over \partial t^2} -
    {1 \over D} {\partial y \over \partial t}=0
\end{equation}
The rule is that the highest-order time derivative wins, so this
equation needs hyperbolic boundary conditions.

\begin{enumerate}
    \prob Make sure you understand this material well
    enough that you are comfortable answering basic
    questions about PDE types and what types of boundary
    conditions go with them on a quiz and/or an exam.
\end{enumerate}
