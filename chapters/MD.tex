\chapter{Molecular Dynamics}
\label{Lab:21}

In this lab we will attempt to simulate the motion of a collection of
particles that are allowed to interact with each other.  To put it
rather simply, we're going to put a bunch of particles in a box, give
them all velocities and let them bounce around.  The particles
could be anything, but they are usually atoms or molecules that
constitute a solid,liquid, or gas.  Molecular dynamic simulations are
typically used to investigate the properties of solids and liquids.
Firstly, we'll focus our attention on building the simulation.  After
that is done, we'll try to answer some interesting physics questions

Building this simulation will involve building the biggest and most
complex code that you have done all semester. We will attack the
problem one step at a time until we have built the entire simulation.
Here is a rough outline of what we must accomplish:

\begin{enumerate}
\item Initialize the position and velocities of the particles.  There
  are two options here:
\begin{enumerate}
\item Initialize the particles to have random initial
  position and velocities.
\item Initialize the particles to a periodic arrangement of points
  (also called a lattice)
\end{enumerate}
\item Iterate over time, updating the positions and velocites of each
  particle at each iteration.  This means that at each iteration we must
\begin{enumerate}
\item Calculate the force on \ul{every} particle due to all the other
  particles that are exerting forces on it.
\item Use the force to update the x and y components of the position
  and velocity vector \ul{for each particle}.
\item Decide what we want to do if a particle moves beyond the
  simulation box(cell).  We could implement periodic boundary
  conditions (particles leaving one side of the box are entering the other
  side) or make the walls of the cell be ``hard'' so that the
  particles bounce off.
\end{enumerate}
\end{enumerate}

We'll have one section for each point, discussing the relevant issues
and building code as we go.

\labsection{Initialization}
There are two ways that you can initialize the positions and
velocities of your particles.  You can either (i) choose to put the
particles at random locations and give them random velocities or you
can (ii) place the particles on a regular, periodic array of points.
Let's do random first
\subsection*{Random}
\marginfig{Figures/LJ.png}{The Lennard-Jones potential} The potential
that we will be using is called a Lennard-Jones potential (see figure
\ref{Figures/LJ.png}).  Notice that the potential increases very
rapidly when the particles get very close.  If you decide to give your
particles random initial positions and velocites, you'll need to
ensure that no two particles are initialized at locations closer than
$1$ unit apart.

\begin{enumerate}
\probtwo  Let's start coding
\begin{enumerate}
\item Build a class (call it \verb!MD! or something else that
makes sense to  you) and construct the class initializer (the
\verb!__init__! function) to initialize the following
variables
\begin{enumerate}
\item the box size (assume a square box)
\item the number of particles in the box
\end{enumerate}
\item Now build a function that initilializes the positions to random
  locations inside of the box, while ensuring that the particles are
  no closer than $1$. Initialize the components of the velocity
  vectors to be random numbers between $(-10,-10)$. (see figure
  \ref{Figures/MD_random.png}) Selecting a random number in a given
  range can be done like this:
\begin{Verbatim}
from numpy.random import uniform
a = uniform(-10,10,size= 2) # Generates a random vector of length 2 in
                            # the range (-10,10) 
\end{Verbatim}
Warning: There are only so many particles that you can put in a box
that satisfy the criteria that no two particles be closer than 1 unit
away.  If you try putting more particles in your box than is possible,
you may get stuck in an infinite loop.
\item Build a function that plots the locations of all the particles
  and call the function to visually verify that you accomplished your goal.
\end{enumerate}
\end{enumerate}
\marginfig{Figures/MD_random.png}{A MD simulation cell with particles
  placed at random locations.}

\subsection*{Regular grid of points}
We'll also experiment with placing the particles on a regular grid of
points.  For example, you could choose a square grid, defined by the
lattice vectors:
\[\vec{v}_1 = 1 \hat{i} + 0 \hat{j}\]
\[\vec{v}_2 = 0 \hat{i} + 1 \hat{j}\]
If you take multiples of these vectors and add them up, you can arrive
at any point on the square grid of points.  In other words:
\begin{equation}
\vec{r}_{m,m} = n \vec{v}_1 + m \vec{v}_2
\end{equation}
The lattice vectors for a triangular lattice are: 
\[\vec{v}_1 = 1 \hat{i} + 0 \hat{j}\]
\[\vec{v}_2 = 0.5 \hat{i} + 0.8666 \hat{j}\]
\begin{enumerate}
  \probtwo Add another function to your class which initializes the
  positions of your particles to be on a square lattice. (see figure \ref{Figures/MD_square.png})  The
  initialization of the velocity vectors is the same as
  before. \textbf{Plot the positions of your particles to see if you
    have done it correctly.}  Here are a few things to watch out for:
\begin{enumerate}
\item  Design the function so that you can pass the lattice vectors
  into the function.
\item  You won't be able to specify how many particles go in the
  simulation box.  Rather the number of particles in your box
  is determined by the lattice site spacing.  So don't try to force
  a certain number of particles into your box.  Just design the function to put
  particles on the lattice sites until the box is filled.
\end{enumerate}
\end{enumerate}
\marginfig{Figures/MD_square.png}{An MD simulation cell with particles
  placed on square lattice sites.}
\marginfig{Figures/MD_triangle.png}{An MD simulation cell with particles
  placed on triangular lattice sites.}

\labsection{Calculating forces}
Before we can talk about actually moving particles around, we must
understand the forces that the particles are exerting on each other.
As mentioned previously, we are using a Lennard-Jones potential (see
figure \ref{Figures/LJ.png}), which is given by the following
function:
\begin{equation}\label{eq:LJ}
V(r) = 4 \left[\left({1\over r}\right)^{12} - \left({1 \over r}\right)^6 \right]
\end{equation}

.  As you can see, the potential is
attractive for large particle separation and gets very large if the
particles ever get very close to one another.  

You may recall that force and potential are related via:

\begin{equation}\label{eq:FtoV}
\vec{F} = - \nabla V
\end{equation}

\begin{enumerate}
\probtwo Evaluate equation \ref{eq:FtoV} by taking a derivative of
  equation \ref{eq:LJ} with respect to r.  You should find that:
\begin{equation}\label{eq:LJForce}
F(r) = 24 \left( {2 \over r^{13}} - {1\over r^7}\right)
\end{equation}
\end{enumerate}

Equation \ref{eq:LJForce} is the magnitude of the force.  The
components can be expressed as:
\begin{equation}\label{eq:components}
F_x = 24 \left( {2 \over r^{13}} - {1\over r^7}\right) {x\over
  |\vec{r}|} \mathrm{~~~~~} F_y = 24 \left( {2 \over r^{13}} - {1\over r^7}\right) {y\over
  |\vec{r}|}
\end{equation}
\begin{enumerate}
  \probtwo Add a function to your class that calculates the
  \textbf{net} force on a given particle and returns it's force vector
  . When you are done, \textbf{call your function for several
    particles and visuallly verify that the returned force vectors make
    sense.} Here are some hints:
\begin{enumerate}
\item  Pass in the position vector of the particle of interest as an argument to this function.
\item You'll need to loop over your list of particles and,
  one-by-one, calculate \ref{eq:components} for the particle in
  question. You won't want to include the force of the particle on
  itself (obviously) and you can neglect forces from particles that
  are more than $4$ length units away (force gets small fast).
\item You'll want to code this function to work for both periodic
  boundary conditions and hard-wall conditions.  For periodic boundary
  conditions there is the possibility that two particles that seem to
  be very far apart are actually quite close. Think about how to
  handle this.
\end{enumerate}

\end{enumerate}

\labsection{Animating}
Now we're ready to make the particles move and interact with each
other.  It may be tempting to reach for Euler's method to update the
positions and velocities of the particles.  However, in molecular
dynamics situations, you will be performing a very large number of
iterations. Numerical errors associated with Euler's method are too
large and will compound over the course of the simulation.  Instead, we
will use a method very similar to the leapfrog method that we studied
in chapter \ref{Lab:8}.   Let's start with the leapfrog equations
(found in equations \eqref{eq:leapfrogOne} through
\eqref{eq:leapfrogThree}), but written in terms of position and
velocity:

\begin{align}
x_{n + {1\over 2}} = x_n + {\tau\over 2} v(x_n,t_n)& ~~\mathrm{(Euler's~ step)}\\
x_{n + 1} = x_n + \tau v(x_{n + {1\over 2}},t_{n + {1\over 2}}) &
~~\mathrm{(\nth{2} ~Order~ Runge-Kutta)}\\
x_{n + {3\over 2}} = x_{n+1} + {\tau \over 2}v(x_{n+1},t_{n+1})&
\end{align}

Recall that with this method, we had to calculate both $x$ and $v$ at every full
time step \textbf{and} at every half time step.  This was because $a$
(the right hand side of ${dv \over dt}$) depended on $v$.  In our
case, that dependency is missing, which means we ought to be able to
reduce the number of steps needed. 

Recall that the Euler step we took was needed so that we could
calculate the velocity one half step forward in time.  This velocity
was then used to calculate the the position one full step forward in
time.  In our case, the velocity update doesn't involve position and
therefore, the first half step for the position is not needed.  In
other words, because the differential equations that we are solving
have the following structure (i.e. no $v_x$ dependence on ${dv_x\over
dt}$, no $v_y$ dependence on ${dv_y\over dt}$, no $x$ dependence on
${dx \over dt}$, and no $y$ dependence on ${dy \over dt}$):

\begin{equation}
{d x \over dt} = v_x \mathrm{~~~~~~} {{d v_x \over dt} = {F_x \over m}}
\end{equation}
\begin{equation}
{d y \over dt} = v_y \mathrm{~~~~~~} {{d v_y \over dt} = {F_y \over m}}
\end{equation}

we can just iterate over the equations below, finding
the velocities at half time steps and positions at full time steps:

\begin{align}
v_{n + {1\over 2}} = v_n + {\tau\over 2} a(x_n,t_n)& ~~\mathrm{(Euler's~ step)}\\
x_{n + 1} = x_n + \tau v(t_{n + {1\over 2}}) &
~~\mathrm{(\nth{2} ~Order~ Runge-Kutta)}\\
v_{n + {3\over 2}} = v_{n+ {1\over 2}} + \tau a(x_{n+ 1},t_{n + 1})& ~~\mathrm{(\nth{2} ~Order~ Runge-Kutta)}\\
\end{align}
This method is called the \textbf{Verlet} method and it is equivalent
to the leapfrog method we learned in chapter \ref{Lab:8}, except that
we are not calculating the position at half time steps or the
velocities at full time steps because we don't need them.  Because we
have reduced the number of calculations, this method should be faster.
Notice also that we are using a centered derivative for all updates
except the very first one, where we needed an Euler step to get us
going.

\begin{enumerate}
\probtwo  Let's build the animate function now
\begin{enumerate}
\subprob Build a member function called \verb!animate! that implements
the Verlet algorithm.  Assume that the masses of the particles are $1$
\subprob After the positions and velocities have been updated, you'll
need to enforce the boundary conditions.  Write one last function that
checks to see if a particle has left the simulation box.  Then think
about how you want to modify that particles position and velocity for
both types of boudary conditions.
\end{enumerate}
\end{enumerate}
\ifsolutions
\textit{Solution:}\\
\begin{codeexample}
\begin{VerbatimOut}{\listingFile}






class MD:

    def __init__(self,nParticles,boxSize,dt,tMax):
        from numpy import zeros
        self.nParticles = nParticles
        self.boxSize = boxSize
        self.dt = dt
        self.tMax = tMax

    def turnOnHardWalls(self):
        self.hardWalls = True

    def initializeParticlesRandom(self):
        from numpy.linalg import norm
        from numpy.random import uniform
        from numpy import copy,zeros
        v0 = 10.0
        index = 0
        self.positions = zeros([self.nParticles,2])
        self.velocities = zeros([self.nParticles,2])
        tooCloseCounter = 0
        while index < self.nParticles:
            tooClose = False
            randomVector = uniform(0,self.boxSize,size = 2)
            for i in self.positions:
                distance = norm(randomVector - i)

                if distance < 1:
                    tooClose = True
                    break
            if tooCloseCounter > 10000:
                print("Added {} particles. Can't add any more".format(index))
                break
            if tooClose:
                tooCloseCounter += 1
                continue

            self.positions[index] = copy(randomVector)
            self.velocities[index] = v0 * uniform(-1,1,size = 2)
            index += 1
        self.nParticles = index
        self.positions = self.positions[:index,:]
        self.velocities = self.velocities[:index,:]

    def initializeParticlesLattice(self,lVecs):
        from numpy.random import uniform
        from numpy import any,copy,zeros
        index = 0
        v0 = 10
        self.positions = zeros([1000,2])
        self.velocities = zeros([1000,2])
        for i in range(-self.boxSize,self.boxSize):
            for j in range(-self.boxSize,self.boxSize):
                candidate = i * lVecs[0] + j * lVecs[1] + 0.5# + 0.001 * uniform(-1,1,size = 2)
                print(candidate)
                if any(candidate > self.boxSize) or any(candidate < 0.1):
                    print('not adding')
                    continue
                self.positions[index] = candidate
                self.velocities[index] = v0 * uniform(-1,1,size = 2)
                index += 1
        self.nParticles = index
        print(index, 'no of particles')
        self.positions = copy(self.positions[:index])
        self.velocities = copy(self.velocities[:index])
        print(self.positions)



    def getForceOnSingleParticle(self,particle):
        from numpy import array,copy,sign
        from numpy.linalg import norm
        fX = 0
        fY = 0
        for i in self.positions:

            xDiff = (particle - i)[0]
            yDiff = (particle - i)[1]

            if not self.hardWalls:
                if abs(xDiff) > self.boxSize/2 and abs(yDiff) > self.boxSize/2:
                    modifiedPos = array([i[0] + sign(xDiff) * self.boxSize,i[1] + sign(yDiff) * self.boxSize])
                elif abs(xDiff) > self.boxSize/2:
                    modifiedPos = array([i[0] + sign(xDiff) * self.boxSize,i[1]])
                elif abs(yDiff) > self.boxSize/2:
                    modifiedPos = array([i[0],i[1] + sign(yDiff) * self.boxSize])
                else:
                    modifiedPos = copy(i)
            else:
                modifiedPos = copy(i)
            xDiff = (particle - modifiedPos)[0]
            yDiff = (particle - modifiedPos)[1]
            distance = norm(particle - modifiedPos)
            if 0.1 < distance < 4:
                fX += 24 * (2/distance**13 - 1/distance**7) * xDiff/distance
                fY += 24 * (2/distance**13 - 1/distance**7) * yDiff/distance
        return array([fX,fY])


    def checkBoundary(self):
        # Now let's handle the boundary conditions, whether
        # they are hard walls or periodict
        for i,k in enumerate(self.positions):
            if k[0] > self.boxSize:
                if self.hardWalls:
                    self.positions[i][0] = self.boxSize - (self.positions[i][0] - self.boxSize)
                    self.velocitiesHalf[i][0] *= -1
                else:
                    self.positions[i][0] = self.positions[i][0] - self.boxSize
            if k[0] < 0:
                if self.hardWalls:
                    self.positions[i][0] = -1 * self.positions[i][0]
                    self.velocitiesHalf[i][0] *= -1
                else:
                    self.positions[i][0] = self.positions[i][0] + self.boxSize

            if k[1] > self.boxSize:
                if self.hardWalls:
                    self.positions[i][1] = self.boxSize - (self.positions[i][1] - self.boxSize)
                    self.velocitiesHalf[i][1] *= -1
                else:
                    self.positions[i][1] = self.positions[i][1] - self.boxSize
            if k[1] < 0:
                if self.hardWalls:
                    self.positions[i][1] = -1 * self.positions[i][1]
                    self.velocitiesHalf[i][1] *= -1
                else:
                    self.positions[i][1] = self.positions[i][1] + self.boxSize

    def animate(self,speed = 10):
        from numpy import zeros
        index = 0
        time = 0

        forces = zeros([self.nParticles,2])
        for i,k in enumerate(self.positions):
            fVec = self.getForceOnSingleParticle(k)
            forces[i] = fVec
        self.velocitiesHalf = self.velocities + self.dt/2 * forces

        while time < self.tMax:
            if index % speed == 0:
                self.plotSimCell(anim=True)


            self.positions = self.positions + self.dt * self.velocitiesHalf

            forces = zeros([self.nParticles,2])
            for i,k in enumerate(self.positions):
                fVec = self.getForceOnSingleParticle(k)
                forces[i] = fVec

            self.velocitiesHalf = self.velocitiesHalf + self.dt * forces

            self.checkBoundary()


            index += 1
            time += self.dt
    def plotSimCell(self,anim = False):
        from matplotlib import pyplot
        pyplot.scatter([n[0] for n in self.positions],[n[1] for n in self.positions])
        pyplot.xlim(0,self.boxSize)
        pyplot.ylim(0,self.boxSize)
        if anim:
            pyplot.draw()
            pyplot.pause(1e-5)
            pyplot.clf()
        else:
            pyplot.show()

from numpy import array
nParticles = 200
boxSize = 10
lVecs = 3 * array([ array([1,0]), array([0,1])])
dt = 0.0001
tMax = 16
speed = 20
myMD = MD(nParticles,boxSize,dt,tMax)
myMD.initializeParticlesRandom()
myMD.turnOnHardWalls()
#myMD.initializeParticlesLattice(lVecs)
myMD.animate(speed)
myMD.getForceOnSingleParticle(myMD.positions[0])
myMD.plotSimCell()


\end{VerbatimOut}
\end{codeexample}
\else
\noindent\rule{5 in}{0.01 in}
\fi

\labsection{Using the model}
Now that you have a working molecular dynamics model, we should try to
use it to answer fun physics questions.  There are many questions that
you could ask, but for today, let's try to observe (and detect) phase
transitions from gas to liquid and then a solid.  To do this, we need
a way to cool our collection of particles.  One straightforward way to
do this is to periodically rescale the velocities according to 

\begin{equation}\label{eq:coolDown}
v \rightarrow 0.95 v
\end{equation}
Over time, the velocities of the particles will decrease and hence the
temperature will go down.

\labsection{Homework}
\begin{enumerate}
\prob Modify your code to start out hot and gradually cool down so
that you can observe transitions from gas to liquid and finally to
solid.  Follow the guidelines below:

\begin{enumerate}
\item Begin by randomly placing 16 particles in a 4.3 x 4.3 box and giving them
  random velocities from the following distribution:
\begin{Verbatim}
velocities = choice([-1,1],size = 2)* normal(100,1,size = 2)
\end{Verbatim}
Can you figure out what this line of code is doing?  Play around with
it until you understand.
\item Use equation \eqref{eq:coolDown} to cool your system
  periodically.  Be careful to ensure that you don't cool the system
  too fast.  You need to let the system equilibriate at one
  temperature before you move to the next
\item One way to detect the phase transition is to watch your
  simulation and see when particles begin to coalesce.  Watch your
  simulation and see if you can observe the transitions.
\item We need a more quantitative way to spot the transitions.  One way
  would be to pick any two particles and track the square of their
  separation over time:
\begin{equation}
\Delta r = |\vec{r}_1 - \vec{r}_2|
\end{equation}
When the system is a liquid, the separation should flucuation in a
large range.  Once the system transitions to a solid, the particles
will remain nearly stationary and thus you would expect the separation
of the two particles to remain fairly constant.  By looking at a plot
of $\Delta r^2$ vs. time, you should be able to spot the transition.
Incorporate this idea into your code and see if you can use it to spot
the solid-liquid transition.

\end{enumerate}

\end{enumerate}