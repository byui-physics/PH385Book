\chapter{Second-order Runge-Kutta}
\label{Lab:2}

\index{ODE} \index{Initial Value Problems} 
Last time we got our first taste of solving an ordinary differential
equation using Euler's method.  We admitted that Euler's method is not
very accurate but we were content to have something to work with.  Our goal today will
be to improve upon Euler's method and to explore just how good our
improvement is.

\labsection{Second Order Runge-Kutta}


Let's start by recalling how we got to Euler's method last time.
Consider a general single, first-order, ordinary differential

\begin{equation}\label{eq:generalFirst}
\frac{dx}{dt} = f(x,t)
\end{equation}

Where $f(x,t)$ is any function of $x$ and $t$.  Euler's method
consists of first writing the derivative in equation
\eqref{eq:generalFirst} with the forward-difference numerical
approximation to get:

\begin{equation}\label{eq:finiteVersion}
\frac{x_{n+1} - x_n}{\tau} = f(x_n,t_n)
\end{equation}

and rearranging to solve for $x_{n+1}$:

\begin{equation}
x_{n+1} = x_n + f(x_n,t_n) \tau
\end{equation}

One might suspect that a good way to improve the accuracy would be to
include more terms from the Taylor expansion when we write down the
numerical derivative on the left hand side of equation
\eqref{eq:finiteVersion} (see equation \eqref{eq:ForwardError} to
remind yourself).  However, these extra terms involve second and higher
derivatives ($x''(t)$, $x'''(t)$, etc) and it's most likely that we
won't have any information about those derivatives which means this
won't help much.

Recall that one reason for the inaccuracies in Euler's method is the
fact that the right hand side of equation \eqref{eq:finiteVersion} is
centered at $t_n$ and the left hand side is centered at $t_{n+{1\over
    2}}$.  Using the centered-difference version of the derivative
would fix this and equation \eqref{eq:finiteVersion} would become:

\begin{equation}\label{eq:finiteVersionCentered}
\frac{x_{n+1} - x_{n-1}}{2 \times \tau} = f(x_n,t_n)
\end{equation}

Solving for $x_{n+1}$:

\begin{equation}
x_{n+1} = x_{n-1}  + 2 \times \tau\times  f(x_n,t_n) 
\end{equation}

or, equivalently:

\begin{equation} \label{eq:RK2}
x_{n+1} = x_n + \tau\times  f(x_{n + {1\over 2}},t_{n + {1\over 2}}) 
\end{equation}

This looks exactly like Euler's method except that instead of
evaluating the right hand side of the differential equation at
$x_n,t_n$, we'd like to evaluate it at $x_{n + {1\over 2}}, t_{n +
  {1\over 2}}$.  We could have probably guessed this from the outset
since the objective was to have both sides of the differential
equation centered at the same point.  
\marginfig[-1in]{Figures/RK2}{\label{fig:RK2}Illustration of the
  second-order Runge Kutta algorithm.}

One question remains:  How will we calculate $x_{n + {1\over 2}}$ so
that we can use it to evaluate $f(x_{n + {1\over 2}},t_{n + {1\over 2}})$.
As contrary as it may sound, we choose to use an Euler's step for this calculation.  This means that
each time step is a two-step process:  (i) first we advance half a time
step using Euler's method and then (ii) use the result of that half
step to make the full time step by evaluating equation \eqref{eq:RK2} as depicted in
figure \ref{fig:RK2}.  In other words:

\begin{align}\label{RK2:methodOne}
x_{n + {1\over 2}} = x_n + {\tau\over 2} f(x_n,t_n)& ~~\mathrm{(Euler's~ step)}\\
x_{n + 1} = x_n + \tau f(x_{n + {1\over 2}},t_{n + {1\over 2}})&
~~\mathrm{(\nth{2} ~Order~ Runge-Kutta)}
\end{align}

We can write this algorithm in a way that turns out to be a little
easier to implement.  It looks like this.  \sidenote{Pay close attention during
practice problem 2 to see why.}

\begin{align}\label{RK2:methodTwo}
k_1 &= \tau f(x_n,t_n)\\
k_2 &= \tau f(x_n + {1\over 2} k_1,t_n + {1\over 2} \tau)\\
x_{n + 1} &= x_n + k_2
\end{align}

You should take a minute to verify that these equations are equivalent
to equations \eqref{RK2:methodOne}

\begin{enumerate}
  \probtwo You should
  recall that Euler's method did not conserve energy. (remember the
  bouncy ball that gained height the more it bounced).  Now that we
  have a new method, the first thing we ought to investigate is whether
  it conserves energy.  Let's investigate using a simple pendulum.
  The differential equation for the simple pendulum is:
\begin{equation}\label{eq:simplependulum}
\frac{d^2s}{dt^2} = -g \sin \theta
\end{equation}
Notice that the left hand side involves $x$, and the right hand side
involves $\theta$.  We need to pick one or the other and stick with
it. Let's choose to use angular quantities.
\begin{enumerate}
\item Rewrite the differential equation in eq \eqref{eq:simplependulum}
  so that the left hand side involves $\theta$, not $x$.
\item Express the second order differential equation as a set of two
  first order equations.
\item Now use \nth{2} order Runge-Kutta to make a plot of $\theta$
  vs. time for several cycles of the motion.  At $t=0$ set $\theta =
  {\pi \over 4}$ and $\omega = 0$.
\item Does \nth{2} order Runge-Kutta conserve energy?
\end{enumerate}
\ifsolutions
\textit{Solution:}\\
\begin{codeexample}
\begin{VerbatimOut}{\listingFile}
class pendulum():
    def __init__(self,l,thetaI, omegaI,tMax = 10,dt = 0.1,g=9.8):
        self.g = g
        self.l = l
        self.theta = [thetaI]
        self.omega = [omegaI]
        self.dt = dt
        self.tMax = tMax


    def Euler(self):
        from numpy import arange,sin
        self.time = arange(0,self.tMax,self.dt)

        for t in self.time:
            self.omega.append(self.omega[-1] - self.g/self.l * self.theta[-1] * self.dt)
            self.theta.append(self.theta[-1] + self.dt * self.omega[-2])
    
    def RK2(self):
        from numpy import arange,sin
        self.time = arange(0,self.tMax,self.dt)

        for t in self.time:
            thetaHalf = self.theta[-1] + self.dt/2 * self.omega[-1]
            omegaHalf = self.omega[-1] - self.dt/2 * self.g/self.l * self.theta[-1]
            self.theta.append(self.theta[-1] + self.dt * omegaHalf)
            self.omega.append(self.omega[-1] - self.g/self.l * thetaHalf * self.dt)


    def plot(self):
        from matplotlib import pyplot

        pyplot.plot(self.time,self.theta[:-1])
        pyplot.show()


from numpy import pi
l = 1.
thetaI =0.2
omegaI = 0.

myPend = pendulum(l,thetaI,omegaI,dt=0.05,tMax = 100)
myPend.RK2()
myPend.plot()
            \end{VerbatimOut}
\end{codeexample}
\else
\noindent\rule{4 in}{0.01 in}
\fi

\end{enumerate}

In the above problem you should have found that the \nth{2}
order Runge-Kutta method \sidenote{Note: Runge-Kutta is prounounced ``Roong-uh
  Koo-tah'', not ``Roonja-Cutta''} \ul{does not conserve energy} just
like Euler's method didn't.  

\labsection{Convergence}
If the Runge-Kutta algorithm doesn't conserve energy, then there has
to be something else about it that makes it better than Euler.  As you
will soon see, the RK algorithm converges more quickly than the
Euler algorithm.  This means that we can use a larger value of $\tau$
 with RK and achieve a higher level of accuracy than we could obtain
 with Euler.  
\begin{enumerate}
  \probtwo \label{prob:RK2andEuler}To compare the rate of convergence between RK and Euler,
  let's return to the batted-ball problem (problem
  \ref{prob:battedBall}).
\begin{enumerate}
\item Make a copy of your batted-baseball code from problem
  \ref{prob:battedBall}
\item Add a member function that use \nth{2} order Runge-Kutta to find
  the trajectory of the baseball.
\item To investigate the convergence behavior of this method,
  construct a loop over step sizes starting at $\tau = 1$, ending at
  $\tau = \num{1e-5}$.  Decrease the step size by a factor of $2$ at
each iteration.  Each time through the loop, use \nth{2} order
Runge-Kutta to calculate the range of the particle and save that range
to a list.  When you are done, plot the ranges vs. step size with the
horizontal axis scaled logarithmically.
\item Do the same thing using Euler's method and overlay the two plots.
\item How do the convergences compare?
\end{enumerate}
\ifsolutions
\textit{Solution:}\\
\begin{codeexample}
\begin{VerbatimOut}{\listingFile}
from numpy import sinh,arange,pi,cos,sin
from matplotlib import pyplot



class projectile():

    def __init__(self, g = 9.8, dt = 0.1):
        self.g = g
        self.dt = dt

        # For part d
    def C(self,v):
        return .198 + .295/(1 + exp((v - 35)/5) )

    def setDragSettings(self,A,C,m,rho,variableC = False):
        # Most of the logic here is for part d
        if not variableC:
            self.variableC = False
            self.dragFactor = 1/2. * A * C * rho/m  #Only include this line for previous parts
        else:
            self.variableC = True
            self.constant = 1/2. * A * rho/m

    def getDerivs(self,varsVec):
        from numpy.linalg import norm
        from numpy import array
        v = varsVec[2:]
        xDeriv = v[0]
        yDeriv = v[1]
        speed = norm([v[0],v[1]])
        if self.variableC:
            self.dragFactor = self.constant * self.C(speed)
        dragX = self.dragFactor * speed * v[0]
        dragY = self.dragFactor * speed * v[1]
        vxDeriv = - dragX
        vyDeriv = - (dragY + self.g)
        return array([xDeriv,yDeriv,vxDeriv,vyDeriv])
    
    def setInitialConditions(self,r,v):
        self.initialPos = r
        self.initialV = v

    def Euler(self):
        from numpy.linalg import norm

        self.x = [self.initialPos[0]]
        self.y = [self.initialPos[1]]
        vx = [self.initialV[0]]
        vy = [self.initialV[1]]

        while self.y[-1] > 0:

            speed = norm([vx[-1],vy[-1]])
            if self.variableC:
                self.dragFactor = self.constant * self.C(speed)
            dragX = self.dragFactor * speed * vx[-1]
            dragY = self.dragFactor * speed * vy[-1]
            self.y.append(self.y[-1] + vy[-1] * self.dt)
            self.x.append(self.x[-1] + vx[-1] * self.dt)
            vy.append(vy[-1] - (dragY + self.g) * self.dt)
            vx.append(vx[-1] - dragX * self.dt)
        self.x[-1] = self.x[-2] - self.y[-2] * (self.x[-2] - self.x[-1])/ (self.y[-2] - self.y[-1])
        self.y[-1] = 0



    def RK2(self):
        from numpy import array
        
        self.x = [self.initialPos[0]]
        self.y = [self.initialPos[1]]
        vx = [self.initialV[0]]
        vy = [self.initialV[1]]
        r = array([self.x[-1],self.y[-1],vx[-1],vy[-1]])
        while self.y[-1] > 0:
            k1 = self.dt * self.getDerivs(r)
            k2 = self.dt * self.getDerivs(r + 1/2. * k1)
            r += k2
            self.x.append(r[0])
            self.y.append(r[1])
        self.x[-1] = self.x[-2] - self.y[-2] * (self.x[-2] - self.x[-1])/ (self.y[-2] - self.y[-1])
        self.y[-1] = 0
            

        
    def plotTrajectory(self):
        from matplotlib import pyplot
        pyplot.plot(self.x,self.y)
        # pyplot.show()


from numpy import exp
speed = 49
angle = 35 * pi/180.
diameter = 0.075
radius = diameter /2
area = pi * radius**2 
mass = .145
C = 0.5
rho = 1.22

dt = 1
eulerRanges = []
rkRanges = []
dts = []
while dt > 1e-4:
    print(dt)
    dts.append(dt)
    trajectoryOne = projectile(dt=dt)
    trajectoryOne.setInitialConditions([0,1],[speed * cos(angle),speed * sin(angle)])
    trajectoryOne.setDragSettings(area,C,mass,rho,variableC = True)
    trajectoryOne.Euler()
    eulerRanges.append(trajectoryOne.x[-1])
    trajectoryOne.setInitialConditions([0,1],[speed * cos(angle),speed * sin(angle)])
    trajectoryOne.RK2Elegant()
    rkRanges.append(trajectoryOne.x[-1])
    dt /=2.
  
pyplot.plot(dts,eulerRanges,label='Euler')
pyplot.xscale('log')
pyplot.plot(dts,rkRanges,label = "RK")
pyplot.xscale('log')
pyplot.xlabel('Step size (s)')
pyplot.ylabel('Projectile Range')
pyplot.legend()
pyplot.show()
\end{VerbatimOut}
\end{codeexample}
\else
\noindent\rule{4 in}{0.01 in}
\fi
\end{enumerate}

\labsection{Error on \nth{2}-order Runge-Kutta \label{section:second-orderRK}}
It's important to quantify the error associated with any numerical
method.  To do this, Let's use a Taylor expansion to estimate a
$x(t + \tau)$.

\begin{equation}\label{eq:TaylorForward}
    x(t+\tau) = x(t) + \tau x'(t) + {1 \over 2} \tau^2 x''(t) +
     {1\over 6}\tau^3 x'''(t) + {1\over 24}\tau^4 x''''(t) ~\cdots~+ x^{(n)}(t) {\tau^n \over n!}
\end{equation}

Next, let's use the Taylor series to estimate $x(t - \tau)$
\sidenote{Mathematicians often have the benefit of hindsight when
  presenting a set of mathematical steps like this.  If you feel like
  you couldn't have done this from scratch, don't feel bad.  The
  mathematicians who derived it orignally no doubt tried several
  things before arriving at the desired result. }

\begin{equation}\label{eq:TaylorBackward}
    x(t-\tau) = x(t) - \tau x'(t) + {1 \over 2} \tau^2 x''(t)
    - {1\over 6}\tau^3 x'''(t) + {1\over 24}\tau^4 x''''(t) ~\cdots~+ x^{(n)}(t) {\tau^n \over n!}
\end{equation}

Next, let's subtract equation \eqref{eq:TaylorBackward} from equation
\eqref{eq:TaylorForward}:

\begin{equation}\label{eq:TaylorDiff}
   x(t+\tau) -  x(t-\tau) =  2 x'(t) \tau 
    - {1\over 3}\tau^3 x'''(t) \tau^3 + ~\cdots~
\end{equation}

and now let's solve for $x(t + \tau)$:

\begin{equation}\label{eq:TaylorDiff}
   x(t+\tau) =  x(t-\tau) +  2 x'(t) \tau 
    - {1\over 3}\tau^3 x'''(t) ~\cdots~
\end{equation}

which is equivalent to 

\begin{equation}\label{eq:TaylorDiff}
   x(t+\tau) =  x(t) +   x'(t + {1\over 2} \tau) \tau 
    - {1\over 3}\tau^3 x'''(t + {1\over 2} \tau) ~\cdots~
\end{equation}
or, in index notation, this equation can be written as:

\begin{equation}\label{eq:TaylorDiff}
   x_{n+1} =  x_n +   f(x_{n+{1\over 2}},t_{n+{1\over 2}}) \tau 
    - {1\over 3}\tau^3 f'(x_{n+{1\over 2}},t_{n+{1\over 2}}) ~\cdots~
\end{equation}
Comparing this with equation \eqref{eq:RK2}, we can see that the
\nth{2}-order Runge-Kutta is neglecting terms of order
$\tau^3$ and higher.  This is better than Euler, which neglects terms
of order $\tau^2$ and higher.  In other words, \nth{2}-order
Runge Kutta is good up to second order, which is why we call it a
\nth{2}-order method.   

\labsection{Homework}
\begin{enumerate}
\prob (\textbf{\LaTeX~ Problem}) It is well-known that baseballs and golfballs can be made to curve in
directions perpendicular to their direction of motion.  The force that
causes this deflection is called the Magnus force and is due to
differences in the force of air drag between opposite sides of the ball.  The
Magnus force can be calculated as:

\begin{equation}
F_M = S_0 \mathbf{\omega} \times \vec{v}
\end{equation}

where $S_0$ is $\num{6.109e-5}$ kg
\begin{enumerate}
\item Use second-order Runge Kutta to model a major-league baseball
  pitch.  You'll need to look up the mass of a baseball, and a typical
  throwing speed for an MLB pitcher.  Assume that the pitcher throws
  the pitch exactly horizontal.  You may assume that the air density
  remains constant during the flight of the ball.
\item Now incorporate the Magnus force into your simulation.  Assume
  that the pitcher throws the ball with side-spin with a rate of
  rotation equal to $\omega = 30 $ revolutions/second.  \\

  Please note that including the Magnus force requires that you track
  all three coordinates of the position and velocity. Let +x be the
  direction from pitcher to batter, +y be the vertical direction, and
  +z be perpendicular to both of them.  You'll
  need to add in the necessary code to calculate the z-coordinates of the ball's position and velocity.  \\

Hint: \texttt{numpy} has a
  function called \texttt{cross} that might help you calculate the
  Magnus force.
\item Make a plot of x vs. z for the baseball (horizontal
  plane) to see how big of a deflection from straight-line motion
  occurs.
\item Now assume that the pitcher gives the ball top spin.  In what
  direction will the Magnus force point now.  Make a
  plot of x vs y for the baseball (vertical plane) and investigate.  
\end{enumerate}
\ifsolutions
\textit{Solution:}\\
\begin{codeexample}
\begin{VerbatimOut}{\listingFile}
from numpy import sinh,arange,pi,cos,sin
from matplotlib import pyplot



class projectile():

    def __init__(self, g = 9.8, dt = 0.1):
        self.g = g
        self.dt = dt

        # For part d
    def C(self,v):
        from numpy import exp
        return .198 + .295/(1 + exp((v - 35)/5) )

    def setDragSettings(self,A,C,m,rho,variableC = False):
        # Most of the logic here is for part d
        if not variableC:
            self.variableC = False
            self.dragFactor = 1/2. * A * C * rho/m  #Only include this line for previous parts
        else:
            self.variableC = True
            self.constant = 1/2. * A * rho/m

    def turnOnMagnus(self, omega,m,S0 = 6.11e-5):
        from numpy import array
        self.Magnus = True
        self.magnusConstant = S0/m
        self.omega = array(omega)
        
    def getDerivs(self,varsVec):
        from numpy.linalg import norm
        from numpy import array,cross
        v = varsVec[3:]
        xDeriv = v[0]
        yDeriv = v[1]
        zDeriv = v[2]
        speed = norm([v[0],v[1],v[2]])
        if self.variableC:
            self.dragFactor = self.constant * self.C(speed)
        dragX = self.dragFactor * speed * v[0]
        dragY = self.dragFactor * speed * v[1]
        dragZ = self.dragFactor * speed * v[2]
        if self.Magnus:
            Magnus = self.magnusConstant * cross(self.omega, v)
        else:
            Magnus = [0,0,0]
        vxDeriv = - dragX + Magnus[0]
        vyDeriv = - (dragY + self.g - Magnus[1])
        vzDeriv = - dragZ + Magnus[2]


        return array([xDeriv,yDeriv,zDeriv,vxDeriv,vyDeriv,vzDeriv])

    def setInitialConditions(self,r,v):
        self.initialPos = r
        self.initialV = v

    def RK2(self):
        from numpy import array

        self.x = [self.initialPos[0]]
        self.y = [self.initialPos[1]]
        self.z = [self.initialPos[2]]
        vx = [self.initialV[0]]
        vy = [self.initialV[1]]
        vz = [self.initialV[2]]
        r = array([self.x[-1],self.y[-1],self.z[-1],vx[-1],vy[-1],vz[-1]])
        while self.y[-1] > 0:
            k1 = self.dt * self.getDerivs(r)
            k2 = self.dt * self.getDerivs(r + 1/2. * k1)
            r += k2
            self.x.append(r[0])
            self.y.append(r[1])
            self.z.append(r[2])


    def plotTrajectory(self):
        from matplotlib import pyplot
        from numpy import array
        pyplot.plot(array(self.x) * 3.28,array(self.y) * 3.28)
        # pyplot.show()


from numpy import pi, array
speed = 31.3
angle = 0 * pi/180.
diameter = 0.075
radius = diameter /2
area = pi * radius**2
mass = .145
C = 0.5
rho = 1.22
omega = array([0,1,0]) * 60 * 2 * pi

trajectoryOne = projectile(dt = 0.01)
trajectoryOne.setInitialConditions([0,2,0],[speed * cos(angle),speed * sin(angle),0])
trajectoryOne.setDragSettings(area,C,mass,rho,variableC = False)
trajectoryOne.turnOnMagnus(omega,mass)
trajectoryOne.RK2()
trajectoryOne.plotTrajectory()
pyplot.show()
\end{VerbatimOut}
\end{codeexample}
\else
\noindent\rule{4 in}{0.01 in}
\fi

\end{enumerate}

