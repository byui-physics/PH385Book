\chapter{A short introduction to \LaTeX}
As a scientist (or engineer), one of your most important tasks will be
to produce quality written documents detailing your work.  The quality
of the document that has your name attached to it speaks volumes about
the type of professional you are.  \LaTeX ~is your best friend in this
arena.  

You can think of \LaTeX ~as almost like a programming language for
generating documents.  You can create variables, loops, function, and
even use if/else statements, all in the context of document
production.  We won't focus on the most complicated parts of \LaTeX~
here but if you are interested, please come and talk to me.  At first
glance, \LaTeX ~may seem like a complicated version of Microsoft Word, but as
you gain experience using it you will find that it actually simplifies
many complicated tasks.  It is an especially useful tools when:
\begin{enumerate}
\item Your document contains a lot of math equations and you are
  referring to those equations frequently in the body of your document.
\item Your document has a lot of citations and a bibliography.
\item Your document has a lot of figures/graphics and you are
  frequently referencing them in the body of your document.
\item You'd like to use your programming skills to build nifty
  automations when building a document.  Here are a few examples
\begin{enumerate}
\item When I build an exam, I include the questions and the solutions
  in the same document.  By modifying a single variable I can turn
  those solutions on to build the key and off to build the exam.
\item When I build a schedule/calendar, I have a file that lists the
  correct dates and other date-specific information.  When the
  schedule is built, the dates are read from the file.  If I need to
  modify the schedule, I simply modify the list of dates and
  recompile.
\end{enumerate}
\end{enumerate}
\LaTeX~ is used heavily in research environments and is a skill
that will be an asset to you.  

\labsection{Getting Started}
The first thing we ought to learn is how to create a simple document
using \LaTeX.  

\begin{enumerate}
\probtwo Open your editor and type the following into the
window and press the \verb!Typeset! button.

\lstset{language=TeX,keywordstyle=\color{black}}
\begin{Verbatim}
\documentclass{article}
\begin{document}
This is a new document.  I can type whatever I want and it will appear
in the body of the text
\end{document}
\end{Verbatim}
\end{enumerate}
 

\noindent\rule{5 in}{0.01 in}

You probably wouldn't use \LaTeX ~to produce a document this simple,
but at least you can see how simple documents are produced. Please
note that every document you produce must have these lines:
\begin{Verbatim}
\documentclass{article}
\begin{document}

\end{document}
\end{Verbatim}


\labsection{Math} The fun starts when you need to add math to your
document.  \LaTeX ~ is great when it comes to producing great-looking,
numbered math equations that can easily be referenced from within the
body of the text.  Literally any math symbol you could want can be
produced if you know the correct syntax.  A full listing of the syntax
for all of the math symbols will not be provided here but can be
easily found with the help of Google.  

There are several different ways that you may want to incorporate math
into your document.  You may just want to add a math equation in the
middle of a sentence

\begin{enumerate}
\probtwo Type the following into your editor window and press the \verb!Typeset! button.
\lstset{language=TeX,keywordstyle=\color{black}}
\begin{Verbatim}
\documentclass{article}
\begin{document}
To find the electric field, we need to evaluate the integral:
$\frac{1}{\alpha} \int_0^{10} \ln(2 x) dx$ and then take the limit as
$\alpha \rightarrow \infty$.
\end{document}

\end{Verbatim}
Take a second to look at the output and to digest the code.  Ask any
questions that you may have.
\end{enumerate}

\noindent\rule{5 in}{0.01 in}

\noindent Let me highlight a few things that you should have noticed:
\begin{enumerate}
\item When typing math in the middle of a sentence, you must enclose
  the math expression in \$ symbols.
\item Sub- and super- scripts are done just as you would expect.  If
  the sub- or super-script is longer than one character, you must
  enclose it in curly braces (\{\}).
\item Common math symbols can be easily produced if you know the
  syntax.  Here we see that \verb!\int! is the syntax for the integral
  symbol, \verb!\ln! is the syntax for $\ln$, and \verb!\frac{}{}! is
  the symbol for making fractions (\verb!{num \over denom}! also works).  The syntax for other commonly-used
  math symbols is provided in table \ref{tab:LatexMathSymbols}.
\item Greek letters can also be produced if you know the correct syntax.
  In this example we see that \verb!\alpha! is the syntax for
  $\alpha$.  The syntax for a few of the other common greek letters is
  given in table \ref{tab:LatexGreekLetters} 
\end{enumerate}

Sometimes the math that you want to write down is a little longer and
you'd like it to be on it's own line.  That's no problem in \LaTeX

\begin{enumerate}
\probtwo  Type the following into your editor windown and push the
\verb!Typset! button:
\lstset{language=TeX,keywordstyle=\color{black}}
\begin{Verbatim}
\documentclass{article}
\usepackage{amsmath} % Needed to use \eqref
\begin{document}
To find the electric field, we need to evaluate the integral:
\begin{equation}\label{eq:integralEquation}
 \frac{1}{\alpha} \int_0^{10} \ln(2 x) dx
\end{equation}

When we evaluate equation \eqref{eq:integralEquation}, we find that it
equals $0$ because $\alpha = \infty$.

\end{document}
\end{Verbatim}

\marginpar{\footnotesize\captionsetup{type=table}
  \vspace{-2.5in}
\begin{tabular}{p{1.05in}p{1.05in}}
\texttt{\textbackslash{}frac\{1\}\{5\}} & $\frac{1}{5}$\\ \\
\texttt{\textbackslash{}sqrt\{5\}} & $\sqrt{5}$\\ \\
\texttt{\textbackslash{}int} & $\int$\\ \\
\texttt{\textbackslash{}ln} & $\ln$\\ \\
\texttt{\textbackslash{}oint} & $\oint$\\ \\
\texttt{\textbackslash{}sum} & $\sum$\\ \\
\texttt{\textbackslash{}infty} & $\infty$\\ \\
\texttt{\textbackslash{}nabla} & $\nabla$\\ \\
\texttt{\textbackslash{}partial} & $\partial$\\ \\
\end{tabular}
\captionof{table}{Commonly-used \LaTeX ~math symbols.\label{tab:LatexMathSymbols}}
}
 Take a second to look at the
output and to digest the code.  Ask any questions that you may have.

\noindent\rule{4.5 in}{0.01 in}
\end{enumerate}
\noindent Once again, let me highlight a few things that you should have noticed:
\begin{enumerate}
\item When you want an equation to be numbered and located on it's own
  line you can use the 
\begin{Verbatim}
\begin{equation} \label{myLabel}
<equation>
\end{equation}
\end{Verbatim} 
\item The \verb!\label! command can be used to assign a name to your
  equation.
\item You can use the name you gave to your equation to reference it
  in the text.  This means that you don't need to know the number that
  it was assigned.  This is done like this
  \verb!\eqref{equationLabel}!. Some commands require that you import
  an external package.  In this case, the \verb!\eqref! command needed
  the \verb!amsmath! package.
\end{enumerate}

Often you will have multiple lines of math and you want to be very
careful about how the lines line up.  Let's explore that a little
bit.

\marginpar{\footnotesize\captionsetup{type=table}
  \vspace{-2.5in}
\begin{tabular}{p{1.05in}p{1.05in}}
\texttt{\textbackslash{}alpha} & $\alpha$\\ \\
\texttt{\textbackslash{}beta} & $\beta$\\ \\
\texttt{\textbackslash{}gamma} & $\gamma$\\ \\
\texttt{\textbackslash{}chi} & $\chi$\\ \\
\texttt{\textbackslash{}Delta} & $\Delta$\\ \\
\end{tabular}
\captionof{table}{Commonly-used greek letters in \LaTeX.\label{tab:LatexGreekLetters}}
}

\begin{enumerate}
\probtwo  Type the folowing into the editor window and press the
\verb!Typeset! button \sidenote{A copy/paste may save you some time.}
\lstset{language=TeX,keywordstyle=\color{black}}
\begin{Verbatim}
\documentclass{article}
\usepackage{amsmath}
\begin{document}

\begin{equation}\label{kRadius}
k = \sqrt{\frac{2mE_f}{\hbar^2}}
\end{equation}

Now we can re-arrange to solve for $E_f$:

\begin{align}
3 \pi^2 \frac{N}{V} &= \left(\frac{2 m E_f}{\hbar^2}\right)^{3/2}\\
3 \pi^2 n &= \left(\frac{2 m E_f}{\hbar^2}\right)^{3/2}\\
\left(3 \pi^2 n\right)^{1/3} &= \left(\frac{2 m E_f}{\hbar^2}\right)^{1/2}\\
\left(3 \pi^2 n\right)^{1/3} &= \sqrt{\frac{2 m E_f}{\hbar^2}}\\
\end{align}

Comparing to equation \eqref{kRadius} we can conclude that:

\begin{equation}
k = \left(3 \pi^2 n\right)^{1/3}
\end{equation}


\end{document}
\end{Verbatim}
As before, digest what you see and ask any questions that you may
have.  What useful bits of information can you extract from this example
\end{enumerate}

\noindent\rule{5 in}{0.01 in}

These are just a few ways to produce math equations in your document.
As you progress in your abilities, more questions will undoubtedly
arise.  We'll handle those situations case by case in this class.
In conclusion, \LaTeX ~produces beautiful math that is formatted in
exactly the way that you want it and can be easily referenced from
within the text.  It should be your go-to tool anytime you need to
produce a document with math in it.

\labsection{Figures}
A key element of any scientific document are graphics.  This could be
a plot or a chart, or just an image.  Regardless, we need to figure
out how to include them in our \LaTeX ~ document.
\begin{enumerate}
\probtwo Download a picture of an elephant and save it to your
computer.  Then type the folowing into the editor window and press the
\verb!Typeset! button \sidenote{A copy/paste may save you some time.}
\sidenote{You may have to fiddle with the \texttt{scale=1.2} to get an
  appropriate size.}
\lstset{language=TeX,keywordstyle=\color{black}}
\begin{Verbatim}
\documentclass{article}
\usepackage{graphicx}  % You need this anytime you want to include graphics
\begin{document}

In figure \ref{figLabel} you will find an image of an elephant.
\begin{figure}
\includegraphics[scale = 1.2]{path/to/figure/of/elephant}
\caption{This is the caption to the figure \label{figLabel}}
\end{figure}

\end{document}
\end{Verbatim}
Look over the code and the output until things start to make sense.
Ask any questions that you may have.
\end{enumerate}

\noindent\rule{5 in}{0.01 in}
\noindent Once again, let me highlight a few things that you should have noticed:
\begin{enumerate}
\item Anytime you are including graphics in your document you will
  need the \verb!graphicx! package.
\item Including a graphic is done with the \verb!\includegraphics!
  command.  The \ul{required} argument to this function (found in curly
  braces) is the path to the image file.  One of the \ul{optional} arguments
  (found in square brackets) is \verb!scale = 1.2!, which allows you
  to specify the size of the image.
\item To label a figure (for referencing in the text) and creating
  captions you need to place your \verb!\includegraphics! statement in
\begin{Verbatim}
\begin{figure}

\end{figure}
\end{Verbatim}
\end{enumerate}


\labsection{Citations and Bibiliography}
A key element to any scientific paper are citations and a
bibliography page.  It is not uncommon for a published paper to have 30-40
citations.  Tracking and managing all of these citations manually
would be a mind-numbing task. Luckily, \LaTeX ~can handle all of this
for you.  There are two things that need to be discussed regarding
citations: i) finding the information for a source to be cited, ii)
citing a source and creating the bibliography.\\

\subsection*{Finding source information}
%\textbf{\Large Finding source information}
Google Scholar is a great place to do literature searches and it can
help you gather the bibliography information too.  But there are some
settings that need to be altered.
\begin{enumerate}
\probtwo  Follow the steps below to enable tex-friendly citation
information:
\begin{enumerate}
\item Go to www.google.com/scholar
\item In the upper left corner, click the drop down menu and click on
  \verb!Settings!.
\item Under Bibliography manager, click ``Show link to import
  citations into BibTeX'' and click ``Save''
\item Perform a search for ``Superalloys'', or some other interesting
  topic of your choosing.
\item Near the bottom of the first hit, there should be a link
  entitled ``Import into BibTex''.  Click it.  This is the source
  information that you will need
\end{enumerate}
\end{enumerate}
%\end{enumerate}
\subsection*{Citing a source and creating the bibliography}
For a scientist that is actively and frequently publishing papers, it
is quite common for them to cite the same source(s) in multiple
publications.  To simplify the citation process, a file that contains
all of his frequently-cited sources (commonly referred to as a ``bib''
file) is maintained.  

\begin{enumerate}
\probtwo Follow the steps below to create a simple bib file.
\begin{enumerate}
\item Create a new file named \texttt{refs.bib}.  It needs to
have the \texttt{.bib} postfix.  
\item Using Google Scholar, search for a few publications and copy
  their bibTex entry into the file refs.bib. (See previous section)  My refs.bib looks like
  this:
\lstset{language=TeX,keywordstyle=\color{black}}
\begin{Verbatim}
@article{caron2000high,
  title={High ysolvus new generation nickel-based superalloys for single crystal turbine blade applications},
  author={Caron, P},
  journal={Superalloys},
  volume={2000},
  pages={737--746},
  year={2000}
}

@article{pettit1984oxidation,
  title={Oxidation and hot corrosion of superalloys},
  author={Pettit, FS and Meier, GH and Gell, M and Kartovich, CS and Bricknel, RH and Kent, WB and Radovich, JF},
  journal={Superalloys},
  volume={85},
  pages={65},
  year={1984}
}

@article{pollock2006nickel,
  title={Nickel-based superalloys for advanced turbine engines: chemistry, microstructure, and properties},
  author={Pollock, Tresa M and Tin, Sammy},
  journal={Journal of propulsion and power},
  volume={22},
  number={2},
  pages={361--374},
  year={2006}
}

\end{Verbatim}
\item Save the file
\item Create another file in the same directory as the bib file.  Put
  the following into the file
\lstset{language=TeX,keywordstyle=\color{black}}
\begin{Verbatim}
\documentclass{article}
\begin{document}

As I discuss superalloys I may need to make a citation
\cite{pollock2006nickel}, or maybe even two at a time \cite{pettit1984oxidation,caron2000high}.

\bibliographystyle{ieeetr}  % There are various styles to choose from.
\bibliography{refs}
\end{document}
\end{Verbatim}
\end{enumerate}
\sidenote{The names of your citations will likely be different from
  mine.  You are also free to modify the names in the refs.bib file to
be whatever you want.}
Study the code and the output until things make sense.  Ask any
questions that you may have.
\end{enumerate}

\noindent\rule{5 in}{0.01 in}

\labsection{Miscellaneous}
The possible topics relating to \LaTeX~ functionality fills entire
books.  I will not try to be complete in my coverage.  Rather, let me
give you a few more handy tidbits that come up frequently.  Below you
will find 

\subsection*{Section Headings}
When organizing a document, you will want to use sections and
subsections. Here's how to do it.
\begin{Verbatim}
\section{Name of Section}  % Section with numbering
\subsection{Name of Subsection} % Subsection with numbering
\section*{Name of Section}  % Section with no numbering
\subsection*{Name of Subsection} % Subsection with no numbering

\end{Verbatim}

As mentioned in the comments, adding a * will supress the numbering
of the sections.
\pagebreak
\subsection*{Abstract and Title}
Every scientific document has an abstract, or short summary, of the
document.  Beneath the title of every paper is listed the authors
names and affiliations.  Here is how all of this is generated in
latex: 

\lstset{language=TeX,keywordstyle=\color{black}}
\begin{Verbatim}

\documentclass{article}
\usepackage{authblk}

            \title{The effect of variable air density on the trajectory of a cannon shell.}
            \author[1]{Lance J. Nelson}
            \affil[1]{Brigham Young University - Idaho}
            \author[2]{SECONDARY AUTHOR}
           \affil[2]{SECONDARY AUTHOR affiliation}
            \date{\today}
\begin{document}
    \maketitle
    \begin{abstract}
      Most projectile motion problems assume that the air density remains
      constant for the duration of the motion.  This is not a bad
      asssumption when the projectiles maximum altitude is relatively
      small.  However, for high altitude projectiles this may be a poor
      assumption.  In this work, we will investigate how a variable air
      density changes the trajectory of a high-altitude projectile.  We
      will also explore the effect of ground temperature on the range of
      these projectiles.
    \end{abstract}


This is my article


\end{document}
\end{Verbatim}
\pagebreak
\subsection*{Including code in your document}
In this class, you may want to include some or all of your code and
explain what you did.  Instead of copying your code into \LaTeX~
(ughh), \LaTeX~ can read your code file and place it into the
document, complete with text highlighting specific to the language you
are coding in.  Here is how you do it: \sidenote{Pay special attention
  to the comments for help understanding}

\begin{Verbatim}
\documentclass{article}
\pdfoutput=1

\usepackage{fancyvrb}
\usepackage{color}

\definecolor{purple}{rgb}{0.625,0.125,0.9375}
\usepackage{listings}
\lstset{
  frame=lines,  % top and bottom rule only
  framesep=2em, % separation between frame and text
    keepspaces=true,
    aboveskip=0in,
    belowskip=0.2in,
    language=Python, % What language are you coding in
    fancyvrb=true,
    breaklines=false,
    basicstyle=\footnotesize\ttfamily,
    numbers=left,
    stepnumber=1,
    keywordstyle=\color{blue},  %What color for keywords
    identifierstyle=,
    commentstyle=\color{red}, % What color do you want comments.
    stringstyle=\ttfamily\color{purple},
    columns=fullflexible,
    showstringspaces=False,
    caption = {  The following is an example code},  
    captionpos = b  % Where do you want the caption located
    }

\begin{document}


\lstinputlisting{testPython.py}  % This is where you specify
                                 % the location of your code file.
\end{document}

\end{Verbatim}

To say that we have only scratched the surface of what \LaTeX can do
would be a huge understatement.  However, what we have given you will
suffice for the requirements of this class.  If your curiosity
overwhelms you and you must have more, please feel free to come talk
with me one-on-one.


\labsection{Homework}



\begin{enumerate}
  \prob Recreate the document called exampleWriteUp.pdf found on
  iLearn.  The figures used in the document are also found on
  iLearn. You are free to use copy/paste to help speed up the process.
Note:  Use the document class
\verb!\documentclass[twocolumn]{revtex4-1}!  to make your document
look like mine.
Also note that all references to equations and figures must not be
hard-coded, but referenced using the assigned name.


\end{enumerate}
\lstset{language=Python,keywordstyle=\color{blue}}
