\chapter{Quantum Bound States}
\label{Lab:13} \index{Quantum Bound States}

\labsection{Quantum bound states}

\index{Schr\"{o}dinger equation!bound states}

Consider the problem of a particle in a 1-dimensional harmonic
oscillator well in quantum mechanics.\footnote{N.\ Asmar, {\it
Partial Differential Equations and Boundary Value Problems}
(Prentice Hall, New Jersey, 2000), p. 470-506.} Schr\"{o}dinger's
equation for the bound states in this well is
\begin{equation}
    - {\hbar^2 \over 2 m} {d^2 \psi \over dx^2} + {1 \over 2}  k x^2 \psi =
    E \psi
\end{equation}
with boundary conditions $\psi=0$ at $\pm \infty$.

The numbers that go into Schr\"{o}dinger's equation are so small
that it makes it difficult to tell what size of grid to use. For
instance, our usual trick of using lengths like 2, 5, or 10 would be
completely ridiculous for the bound states of an atom where the
typical size is on the order of $10^{-10}$~m. We could just set
$\hbar$, $m$, and $k$ to unity, but then we wouldn't know what
physical situation our numerical results describe. When
computational physicists encounter this problem a common thing to do
is to ``rescale'' the problem so that all of the small numbers go
away. And, as an added bonus, this procedure can also allow the
numerical results obtained to be used no matter what $m$ and $k$ our
system has.

\begin{enumerate}
\probtwo \label{P:12.3} This probably seems a little nebulous, so
    follow the recipe below to see how to rescale in this problem
    (write it out on paper).

\marginfig{Figures/f04p4}{\label{f04p4}The probability distributions for the ground
state and the first three excited states of the harmonic
oscillator.}

    (i) In Schr\"{o}dinger's equation use the substitution $x=a
    \xi$, where $a$ has units of length and $\xi$ is
    dimensionless. After making this substitution put the left
    side of Schr\"{o}dinger's equation in the form
    \begin{equation}
        C \left(  -{D \over 2}{d^2 \psi \over d \xi^2}
        +{1 \over 2} \xi^2 \psi \right)
        = E \psi
    \end{equation}
    where $C$ and $D$ involve the factors $\hbar$, $m$, $k$, and $a$.\\
\ifsolutions
\textit{Solution:}\\
The chain rule allows us to write:
\begin{equation}
\frac{\partial}{\partial \xi} = \frac{\partial x}{\partial \xi}
\frac{\partial }{\partial x}
\end{equation}
Now we can transform our derivatives:

\begin{align}
\frac{\partial \psi}{\partial \xi} &= \frac{\partial x}{\partial \xi}
\frac{\partial \psi}{\partial x}\\
&= a \frac{\partial \psi}{\partial x}
\end{align}

\begin{align}
\frac{\partial^2 \psi}{\partial \xi^2} &= \frac{\partial }{\partial
  \xi} \frac{\partial \psi }{\partial \xi}\\
&=\frac{\partial }{\partial
  \xi} \left( a \frac{\partial \psi}{\partial x}\right)\\
&= a\frac{\partial x}{\partial \xi}
\frac{\partial }{\partial x}\frac{\partial \psi}{\partial x}\\
&= a^2\frac{\partial^2 \psi}{\partial x^2}\\
\end{align}

Now we can begin to transform the differential equation:
\begin{align}
    - {\hbar^2 \over 2 m} {d^2 \psi \over dx^2} + {1 \over 2}  k x^2 \psi &=
    E \psi\\
    - {\hbar^2 \over 2 m a^2} {d^2 \psi \over d\xi^2} + {1 \over 2}  k a^2\xi^2 \psi &=
    E \psi\\
    k a^2 \left(- {\hbar^2 \over 2 k m a^4} {d^2 \psi \over d\xi^2} + {1
        \over 2}  \xi^2 \psi \right) &=
    E \psi\\
\end{align}

From this we can see that:

\begin{equation}
C = k a^2
\end{equation}

\begin{equation}
D = {\hbar^2 \over k m a^4}
\end{equation}


\fi
    (ii) Make the differential operator inside the parentheses
    (...) on the left be as simple as possible by choosing to
    make $D=1$. This determines how the characteristic length $a$
    depends on $\hbar$, $m$, and $k$. Once you have determined
    $a$ in this way, check to see that it has units of length.
    You should find
    \begin{equation}
        a = \left( { \hbar^2 \over k m} \right)^{1/4}
        = \sqrt{ \hbar \over m \omega}
        ~~~~{\rm where}~~~~ \omega = \sqrt{k \over m}
    \end{equation}\\
\ifsolutions
\textit{Solution:}\\
By setting $D=1$ we can solve for the characteristic length:
\begin{align}
1 &= {\hbar^2 \over k m a^4}\\
a^4 &= {\hbar^2 \over k m }\\
a &= \left({\hbar^2 \over k m }\right)^{1/4}\\
\end{align}
\fi

    (iii) Now rescale the energy by writing $E=\epsilon \bar{E}$,
    where $\bar{E}$ has units of energy and $\epsilon$ is
    dimensionless.  Show that if you choose $\bar{E}=C$ in the
    form you found above in (i) that Schr\"{o}dinger's equation
    for the bound states in this new dimensionless form is
    \begin{equation}\label{dimensionless}
        -{1 \over 2}{d^2 \psi \over d \xi^2} +{1 \over 2} \xi^2 \psi
        = \epsilon \psi
    \end{equation}
    You should find that
\begin{equation}
    \bar{E} = \hbar \sqrt{k \over m}
\end{equation}
Verify that $\bar{E}$ has units of energy.\\
\ifsolutions
\textit{Solution:}\\
Letting $ E = \epsilon \bar{E}$ where $\bar{E} = C = k a^2$
we find
\begin{align}
    k a^2 \left(- {\hbar^2 \over 2 k m a^4} {d^2 \psi \over d\xi^2} + {1
        \over 2}  \xi^2 \psi \right) &=
    \epsilon  k a^2 \psi\\
    - {\hbar^2 \over 2 k m a^4} {d^2 \psi \over d\xi^2} + {1
        \over 2}  \xi^2 \psi &=
    \epsilon \psi\\
\end{align}

\begin{align}
\bar{E} &= k a^2\\
&= k \sqrt{{\hbar^2\over k m}}\\
&= \sqrt{{\hbar^2 k^2\over k m}}\\
&= \sqrt{{\hbar^2 k\over m}}\\
&= \hbar\sqrt{{ k\over m}}\\
\end{align}
\fi
\end{enumerate}

Now that Schr\"{o}dinger's equation is in dimensionless form it makes
sense to choose a grid that goes from -4 to 4, or some other similar
pair of numbers. These numbers are supposed to approximate infinity
in this problem, so make sure (by looking at the eigenfunctions) that
they are large enough that the wave function goes to zero with zero
slope at the edges of the grid. As a guide to what you should find,
Figure~\ref{f04p4} displays the square of the wave function for
the first few excited states. (The amplitude has been appropriately
normalized so that $\int |\psi (x)|^2 = 1$

If you look in a quantum mechanics textbook you will find that
the bound state energies for the simple harmonic oscillator are
given by the formula
\begin{equation}
    E_n = (n + {1 \over 2}) \hbar \sqrt{k \over m} =
    (n + {1 \over 2}) \bar{E}
\end{equation}
so that the dimensionless energy eigenvalues $\epsilon_n$ are given by
\begin{equation}\label{eigenenergies}
    \epsilon_n = n + {1 \over 2}
\end{equation}
\marginfig{Figures/f04p5}{The probability distributions for the ground
state and the first three excited states for the potential in
Problem~\ref{P:4.5}.}
\labsection{Homework}
\begin{enumerate}
\prob \label{H:12.4}
(\textbf{\LaTeX~ Problem})Use Python's ability to do eigenvalue problems to solve equation
\eqref{dimensionless} and verify that equation \eqref{eigenenergies} is
the correct formula for the bound state energies.  Check for $n=0,1,2,3,4$.
Here are some hints that may help you:
\begin{enumerate}
\subprob When plotting $|\psi(x)|^2$, you'll need the
\texttt{conjugate} function inside of \texttt{numpy}.
\subprob  Don't forget to normalize the wavefunction.  The
\texttt{dot} command inside of numpy will be helpful.
\end{enumerate}
\ifsolutions
\textit{Solution:}\\
\begin{codeexample}
\begin{VerbatimOut}{\listingFile}
class HangingChain():

    def __init__(self,a,b,N):
        from numpy import linspace
        self.L = b
        self.N = N
        self.x,self.dx = linspace(a,b,N,retstep = True)

    def loadMatrices(self):
        from numpy import zeros,sin,cos,eye
        # Load A
        self.A = zeros([self.N,self.N])
        self.A[0,0] = 1
        self.A[-1,-1] = 1

        for i in range(1,self.N-1):
            self.A[i][i] = 1. /self.dx**2 + 1./2. * self.x[i]**2
            self.A[i][i+1] = -1./( 2 * self.dx**2)
            self.A[i][i-1] = -1./( 2 * self.dx**2)

        # Load b
        self.B = eye(self.N)
        self.B[0,0] = 0.
        self.B[-1,-1] = 0.

    def solveProblem(self):
        from scipy.linalg import eig
        from numpy import sqrt,pi
        self.eVals,self.eVecs = eig(self.A,self.B)
        self.key = sorted(range(len(self.eVals)), key=lambda k: self.eVals[k])
        #        self.omega = sorted(self.omega)
    def plot(self,mode):
        from numpy import real,conjugate,sqrt,dot
        from matplotlib import pyplot
        normalization = sqrt( dot( self.eVecs[:,self.key[mode]],conjugate(self.eVecs[:,self.key[mode]])) * self.dx)
        print normalization, 'here'
        pyplot.plot(self.x,self.eVecs[:,self.key[mode]] * conjugate(self.eVecs[:,self.key[mode]])/normalization**2,'r.-')
        pyplot.title("mode:  " + str(mode + 1)+ '\n' +  str(real(self.eVals[self.key[mode]])))
        pyplot.xlim(-5,5)
        pyplot.show()


from matplotlib import pyplot
a = -4.
b = 4.
N = 500

myBVP = HangingChain(a,b,N)
myBVP.loadMatrices()
myBVP.solveProblem()
print myBVP.eVals[1]
myBVP.plot(1)
\end{VerbatimOut}
\end{codeexample}
\else
\noindent\rule{5 in}{0.01 in}
\fi



\prob \label{P:4.5}
Now redo this entire problem, but with the harmonic oscillator
potential replaced by
\begin{equation}
    V(x) = \mu x^4
\end{equation}
so that we have
\begin{equation}
    - {\hbar^2 \over 2 m} {d^2 \psi \over dx^2} + \mu x^4 \psi =
    E \psi
\end{equation}
With this new potential you will need to find new formulas for the
characteristic length $a$ and energy $\bar{E}$ so that you can use
dimensionless scaled variables as you did with the harmonic
oscillator. Choose $a$ so that your scaled equation is
\begin{equation}
    -{1 \over 2}{d^2 \psi \over d \xi^2} + \xi^4 \psi  = \epsilon \psi
\end{equation}
with $E=\epsilon \bar{E}$. Use Mathematica and/or algebra
by hand to show that
\begin{equation}
    a=\left( {\hbar^2 \over m \mu} \right)^{1/6}~~~~~
    \bar{E} =\left( {\hbar^4 \mu \over m^2} \right)^{1/3}
\end{equation}
Find the first 5 bound state energies by finding the first 5 values
of $\epsilon_n$ in the formula $E_n = \epsilon_n \bar{E}$.
\end{enumerate}
