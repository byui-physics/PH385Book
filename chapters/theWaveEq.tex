\chapter{The Wave Equation: Steady State and Resonance}
\label{Lab:11}\index{Wave equation}

To see why we did so much work in Lab~\ref{Lab:10} on ordinary
differential equations when this is a course on partial differential
equations, let's look at the wave equation in one dimension. For a
string of length $L$ fixed at both ends with a force applied to it
that varies sinusoidally in time, the wave equation can be written as
\begin{equation} \label{waveeqn}
    \mu {\partial^2 y \over \partial t^2}
    = T {\partial^2 y \over \partial x^2} + f(x) \cos{\omega t}~~~~;
    ~~~~y(0,t)=0,~~~y(L,t)=0
\end{equation}
where $y(x,t)$ is the (small) sideways displacement of the string as
a function of position and time, assuming that $y(x,t) \ll L$.\
\footnote{N.\ Asmar, {\it Partial Differential Equations and Boundary
Value Problems} (Prentice Hall, New Jersey, 2000), p. 87-110.} This
equation may look a little unfamiliar to you, so let's discuss each
term. We have written it in the form of Newton's second law, $F=ma$.
The  ``$ma$'' part is on the left of Eq.~\eqref{waveeqn}, except that
$\mu$ is not the mass, but rather the linear mass density
(mass/length). This means that the right side should have units of
force/length, and it does because $T$ is the tension (force) in the
string and $\partial^2 y / \partial x^2$ has units of 1/length. (Take
a minute and verify that this is true.) Finally, $f(x)$ is the
amplitude of the driving force (in units of force/length) applied to
the string as a function of position (so we are not necessarily just
wiggling the end of the string) and $\omega$ is the frequency of the
driving force.

Before we start calculating, let's train our intuition to guess how
the solutions of this equation behave. If we suddenly started to push
and pull on a string under tension we would launch waves, which would
reflect back and forth on the string as the driving force continued
to launch more waves. The string motion would rapidly become very
messy. Now suppose that there was a little bit of damping in the
system (not included in the equation above, but in a future lab we
will add it). Then what would happen is that all of the transient
waves due to the initial launch and subsequent reflections would die
away and we would be left with a steady-state oscillation of the
string at the driving frequency $\omega$. (This behavior is the wave
equation analog of damped transients and the steady final state of a
driven harmonic oscillator.) \index{Damped transients}

\labsection{Steady state solution}
\index{Steady state}
You might remember from PH123 that there are two forms for the
solution to the wave equation.  They are:
\begin{equation}\label{traveling}
    y(x,t) = A \cos (k x - \omega t + \phi)
\end{equation}

and
\begin{equation}\label{standing}
    y(x,t) = g(x) \cos (\omega t)
\end{equation}
Take a second to notice the differences between these two functions.
Waves described by equation \eqref{traveling} are called traveling
waves and those described by equation \eqref{standing} are called
standing waves.  Let's look for the steady-state (or standing wave)
solution by guessing that the solution has the form of
equation \eqref{standing}.  Substituting this ``guess'' into the wave
equation to see if it works yields (after some rearrangement)
\begin{equation}
    T g''(x) + \mu \omega^2 g(x) = -f(x)~~~~;
    ~~~~g(0)=0,~~~g(L)=0
    \label{steady}
\end{equation}
This is just a two-point boundary value problem of the kind we
studied in Lab~\ref{Lab:10}, so we can solve it using our matrix
technique.

\marginfig{Figures/f03p1a}{Solution to \ref{P:11.1a}}

\begin{enumerate}
\probtwo \label{P:11.1}
\begin{enumerate}
\subprob \label{P:11.1a} Modify one of your Python scripts
    from Lab~\ref{Lab:10} to solve Eq.~(\ref{steady}) with
    $\mu=0.003$, $T=127$, $L=1.2$, and $\omega=400$. (All
    quantities are in SI units.) Find the steady-state
    amplitude associated with the driving force density:
    \begin{equation}
    \begin{array}{c}
     \\
    f(x) \\
     \\
    \end{array}
    =
    \left\{
    \begin{array}{ll}
    0.73 & {\rm if~~~} 0.8 \le x \le 1 \\
     & \\
    0 & {\rm if~~~} x<0.8 \textrm{ or } x>1 \\
    \end{array}
    \right.
    \end{equation}


\subprob \label{P:11.1b}Repeat the calculation in part (a)
for 100 different frequencies between $\omega=400$ and
$\omega=1200$ by putting a loop around your calculation in
(a) that varies $\omega$. Use this loop to load the maximum
amplitude as a function of $\omega$ and plot it to see the
resonance behavior of this system. Can you account
qualitatively for the changes you see in $g(x)$ as $\omega$
varies? (Use \texttt{pyplot.pause(.1)} after the plot of $g(x)$ and
watch what happens as $\omega$ changes.)

\marginfig[-0.75in]{Figures/f03p1b}{Solution to problem~\ref{P:11.1b}.}

\end{enumerate}
\end{enumerate}
\ifsolutions
\textit{Solution:}\\
\begin{codeexample}
\begin{VerbatimOut}{\listingFile}
class BVP():

    def __init__(self,a,b,N):
        from numpy import linspace
        self.L = b
        self.N = N
        self.x,self.dx = linspace(a,b,N,retstep=True)

    def initialize(self,T,mu,omega):
        self.T = T
        self.mu = mu
        self.omega = omega

    def f(self,x):
        if isinstance(x,ndarray):
            from numpy import array
            return array([self.f(var) for var in x])
        else:
            if  0.8 <= x <= 1:
                return 0.73
            else:
               return 0.

    def loadMatrices(self):
        from numpy import zeros,sin,cos
        # Load A
        self.A = zeros([self.N,self.N])
        self.A[0][0] = 1
        self.A[-1][-1] = 1

        for i in range(1,self.N-1):
            self.A[i][i] = -2./self.dx**2 + self.mu * self.omega**2/self.T
            self.A[i][i+1] = 1./self.dx**2
            self.A[i][i-1] = 1./self.dx**2

        # Load b
        self.b = -self.f(self.x)/self.T# [-self.f(var)/self.T for var in self.x]
        self.b[0] = 0
        self.b[-1] = 0

    def solveProblem(self):
        from numpy.linalg import solve
        self.solution = solve(self.A,self.b)

    def plot(self):
        from matplotlib import pyplot
        pyplot.plot(self.x,self.solution,'r.')
        #        pyplot.show()
    def plotExact(self):
        from numpy import cos,sin
        yExact = -0.223681 * ( cos(6 - self.x) + cos(2 + 3 * self.x) + 16 * sin(3*self.x) - cos(2 - 3 * self.x) - cos(6 + self.x))
        pyplot.plot(self.x,yExact)

from matplotlib import pyplot
from numpy import linspace
a = 0
b = 1.2
N = 100
T  = 127
mu = 0.003
omega = linspace(400,1200,100)
amplitudes = []
myBVP = BVP(a,b,N)
for om in omega:
    myBVP.initialize(T,mu,om)
    myBVP.loadMatrices()
    myBVP.solveProblem()
    amplitudes.append(max(abs(myBVP.solution)))
    myBVP.plot()
    pyplot.draw()
    print(omega)
    #myBVP.plotExact()
     pyplot.ylim(-.05,.05)
    pyplot.pause(.1)
    pyplot.clf()
    #    pyplot.show()
pyplot.plot(omega,amplitudes)
pyplot.show()
\end{VerbatimOut}
\end{codeexample}
\else
\noindent\rule{5 in}{0.01 in}
\fi

In problem~\ref{P:11.1b} you should have noticed an apparent
resonance behavior, with resonant frequencies near $\omega=550$ and
$\omega=1100$ (see Fig.~\ref{Figures/f03p1b}). Now we will learn how to use
Python to find these resonant frequencies directly (i.e.~without
solving the differential equation over and over again).

\labsection{Resonance and the eigenvalue problem}
\index{Resonance}

The essence of resonance is that at certain frequencies a large
steady-state amplitude is obtained with a very small driving force.
To find these resonant frequencies we seek solutions of
Eq.~(\ref{steady}) for which the driving force is zero. With
$f(x)=0$, Eq.~(\ref{steady}) takes on the form
\begin{equation}
    -\mu \omega^2 g(x)=  T  g''(x)  ~~~~;~~~~
    g(0)=0,~~~g(L)=0
    \label{eigen}
\end{equation}
\index{Eigenvalue problem}
If we rewrite this equation in the form
\begin{equation}
    g''(x) = -\left( \frac{\mu \omega^2}{T} \right) g(x)
    \label{eigen2}
\end{equation}
then we see that it is in the form of a classic eigenvalue problem:
\begin{equation}
    A g = \lambda g
    \label{simpleig}
\end{equation}
where $A$ is a linear operator (the second derivative on the left
side of Eq.~(\ref{eigen2})) and $\lambda$ is the eigenvalue
($-\mu \omega^2/T$ in Eq.~(\ref{eigen2}).)


Equation~(\ref{eigen2}) is easily solved analytically, and its
solutions are just the familiar sine and cosine functions. The
condition $g(0)=0$ tells us to try a sine function form,
$g(x)=g_0 \sin (kx)$. To see if this form works we substitute it
into Eq.~(\ref{eigen2} and find that it does indeed work,
provided that the constant $k$ is $k=\omega \sqrt{\mu / T}$. We
have, then,

\marginfig{Figures/f03ResPhotos}{Photographs of the first three resonant
modes for a string fixed at both ends.}

\begin{equation}\label{eq:Modes}
    g(x)=g_0 \sin{\left( \omega \sqrt{\mu \over T} x \right) }
\end{equation}
where $g_0$ is the arbitrary amplitude. But we still have one more
condition to satisfy: $g(L)=0$. This boundary condition tells us the
values that resonance frequency $\omega$ can take on.  When we apply
the boundary condition, we find that the resonant frequencies of the
string are given by
\begin{equation}\label{omegastring}
    \omega = n \frac{\pi}{L} \sqrt{ \frac{T}{\mu}}
\end{equation}
where $n$ is an integer.  Each value of $n$ gives a specific
resonance frequency from Eq.~\eqref{omegastring} and a
corresponding spatial amplitude $g(x)$ given by
Eq.~\eqref{eq:Modes}. Figure~\ref{f03ResPhotos} shows photographs
of a string vibrating for $n=1,2,3$.

For this simple example we were able to do the eigenvalue problem
analytically without much trouble. However, when the differential
equation is not so simple we will need to do the eigenvalue
calculation numerically, so let's see how it works in this simple
case. Rewriting Eq.~(\ref{eigen}) in matrix form, as we learned to do
by finite differencing the second derivative, yields
\begin{equation}
    {\bf A g} = \lambda {\bf g}
    \label{eq:eigenm}
\end{equation}
which is written out as
\begin{equation}
\left[
\begin{array}{cccccccc}
? & ? & ? & ? & ... & ? & ? & ? \\
{1 \over h^2} & -{2 \over h^2}  & {1 \over h^2} & 0 & ... & 0 & 0 & 0 \\
0 & {1 \over h^2} & -{2 \over h^2}  & {1 \over h^2} & ... & 0 & 0 & 0 \\
. & . & . & . & ... & . & . & . \\
. & . & . & . & ... & . & . & . \\
. & . & . & . & ... & . & . & . \\
0 & 0 & 0 & 0 & ... & {1 \over h^2} & -{2 \over h^2}  & {1 \over h^2} \\
? & ? & ? & ? & ... & ? & ? & ? \\
\end{array}
\right]
\left[
\begin{array}{r}
g_1 \\
g_2 \\
g_3  \\
.   \\
.   \\
.   \\
g_{N-1}  \\
g_N
\end{array}
\right]
= \lambda
\left[
\begin{array}{r}
?   \\
g_2 \\
g_3 \\
.  \\
.  \\
.  \\
g_{N-1} \\
?
\end{array}
\right]
\label{eigenmatrix1}
\end{equation}
where $\lambda = -\omega^2 {\mu \over T}$. The question marks in the
first and last rows remind us that we have to invent something to put
in these rows that will implement the correct boundary conditions.
Note that having question marks in the $g$-vector on the right is a
real problem because without $g_1$ and $g_N$ in the top and bottom
positions, we don't have an eigenvalue problem (i.e.~the vector ${\bf
g}$ on left side of Eq.~(\ref{eigenmatrix1}) is not the same as the
vector ${\bf g}$ on the right side).

\index{Generalized eigenvalue problem}
\index{Eigenvalue problem!generalized}

The simplest way to deal with this question-mark problem and to also
handle the boundary conditions is to change the form of
Eq.~(\ref{simpleig}) to the slightly more complicated form of a {\it
generalized eigenvalue problem}, like this:
\begin{equation}
    {\bf A g} = \lambda {\bf B g}
    \label{generaleig}
\end{equation}
where $\bf B$ is another matrix, whose elements we will choose to
make the boundary conditions come out right. To see how this is done,
here is the generalized modification of Eq.~(\ref{eigenmatrix1}) with
$\bf B$ and the top and bottom rows of $\bf A$ chosen to apply the
boundary conditions $g(0)=0$ and $g(L)=0$.
\marginfig{Figures/f03p2}{The first three eigenfunctions found in
\ref{P:11.2}.  The points are the numerical eigenfunctions
and the line is the exact solution.}
\small
\begin{align}
{\bf A}~~~~~~~~~~~~~~~~~~~~~~~~~~~~~~~~~~~~~~~~{\bf g} \qquad
& = \lambda ~~~~~~~~~~~~~~~~~~~~~~~~~~~{\bf B}
~~~~~~~~~~~~~~~~~~~~~~~~~~~~~~~~~{\bf g}
\nonumber \\
\left[
\begin{array}{cccccc}
1 & 0 & 0 &  ...  & 0 & 0 \\
{1 \over h^2} & -{2 \over h^2}  & {1 \over h^2} &  ...  & 0 & 0 \\
0 & {1 \over h^2} & -{2 \over h^2}  &  ...  & 0 & 0 \\
. & . & . &  ...  & . & . \\
. & . & . &  ...  & . & . \\
. & . & . &  ...  & . & . \\
0 & 0 & 0 &  ... &-{2 \over h^2}  & {1 \over h^2} \\
0 & 0 & 0 &  ...  & 0 & 1 \\
\end{array}
\right]
\left[
\begin{array}{r}
g_1 \\
g_2 \\
g_3  \\
.   \\
.   \\
.   \\
g_{N-1}  \\
g_N
\end{array}
\right]
&= \lambda
\left[
\begin{array}{cccccc}
0 & 0 & 0 &  ...  & 0 & 0 \\
0 & 1 & 0 &  ...  & 0 & 0 \\
0 & 0 & 1 &  ...  & 0 & 0 \\
. & . & . &  ...  & . & . \\
. & . & . &  ...  & . & . \\
. & . & . &  ...  & . & . \\
0 & 0 & 0 &  ...  & 1 &0 \\
0 & 0 & 0 &  ...  & 0 & 0 \\
\end{array}
\right]
\left[
\begin{array}{r}
g_1 \\
g_2 \\
g_3 \\
.  \\
.  \\
.  \\
g_{N-1} \\
g_N
\end{array}
\right]
\label{geneigenmatrix}
\end{align}

\normalsize Notice that the matrix $\bf B$ is very simple: it is just
the identity matrix (made in Python with {\tt eye(N,N)}) except that
the first and last rows are completely filled with zeros. \textbf{Take a
minute now and do the matrix multiplications corresponding the first
and last rows and verify that they correctly give $g_1=0$ and
$g_N=0$, no matter what the eigenvalue $\lambda$ turns out to be.}

\begin{enumerate}
\probtwo  \label{P:11.2} Let's solve equation \eqref{eigen} as an eignevalue problem
in Python.  
\begin{enumerate}
\subprob  Load the matrix $\mathbf{A}$ with the matrix on the left hand
side of equation \eqref{geneigenmatrix} and the matrix $\mathbf{B}$ on
the right side.
\subprob  Solve this problem using the \texttt{eig} function inside of
scipy.linalg
\subprob Convert the eigenvalues into frequencies via $\omega^2 = -
{T\over \mu} \lambda$, sort the squared frequencies in ascending
order.
\subprob  Now plot some of the eigenfunctions, displaying the
corresponding frequency as the title of the plot
\subprob  The exact frequencies are given by equation
\eqref{omegastring}.  Calculate the exact frequencies for $n = 1,2,3$ and
compare them to your numerical results.  Now calculate the exact
frequency for $n=20$ and compare to your numerical results. What trend
do you notice and can you explain why?
\subprob Now that you have the exact resonant frequencies, return to
your code from problem \ref{P:11.1} and make plots of the steady state
amplitude for driving frequencies near these resonant values.  You
should find very large amplitudes, indicating that you have found the
resonant modes.
\marginfig{Figures/f03p3a}{The first three
    eigenfunctions for \ref{P:11.3}.}\ifsolutions
\textit{Solution:}\\
\begin{codeexample}
\begin{VerbatimOut}{\listingFile}
class BVP():

    def __init__(self,a,b,N):
        from numpy import linspace
        self.L = b
        self.N = N
        self.x,self.dx = linspace(a,b,N,retstep=True)

    def initialize(self,T,mu):
        self.T = T
        self.mu = mu

    def loadMatrices(self):
        from numpy import zeros,sin,cos,eye
        # Load A
        self.A = zeros([self.N,self.N])
        self.A[0][0] = 1
        self.A[-1][-1] = 1

        for i in range(1,self.N-1):
            self.A[i][i] = -2./self.dx**2
            self.A[i][i+1] = 1./self.dx**2
            self.A[i][i-1] = 1./self.dx**2

        # Load b
        self.B = eye(self.N)
        self.B[0] = 0
        self.B[-1] = 0

    def solveProblem(self):
        from scipy.linalg import eig
        from numpy import sqrt,pi
        self.eVals,self.eVecs = eig(self.A,self.B)
        self.omega = sqrt(-self.T/self.mu * self.eVals)/(2 * pi)
        self.key = sorted(range(len(self.omega)), key=lambda k: self.omega[k])
        self.omega = sorted(self.omega)

    def plot(self,mode):
        from numpy import real
        from matplotlib import pyplot
        prefactor = 1/max(abs(self.eVecs[:,self.key[mode]]))
        pyplot.plot(self.x,prefactor * self.eVecs[:,self.key[mode]],'r.-')
        pyplot.title("mode:  " + str(mode + 1)+ '\n' +  str(real(self.omega[mode])) + " Hz")
        pyplot.show()

    def exactEigenVal(self,mode):
        from numpy import sqrt, pi
        return (mode + 1) * pi/self.L * sqrt(self.T/self.mu) /( 2* pi)

from matplotlib import pyplot
a = 0
b = 1.2
N = 20
T  = 127
mu = 0.003

myBVP = BVP(a,b,N)
myBVP.initialize(T,mu)
myBVP.loadMatrices()
myBVP.solveProblem()
print myBVP.omega[1]
print myBVP.exactEigenVal(1)
myBVP.plot(3)
\end{VerbatimOut}
\end{codeexample}
\else
\noindent\rule{5 in}{0.01 in}
\fi

\end{enumerate}
\end{enumerate}

\labsection{Homework}

\begin{enumerate}
\prob Let's explore what happens to the eigenmode shapes when we
change the boundary conditions.
\begin{enumerate}
\subprob \label{P:11.3} Change your program
    from problem~\ref{P:11.2} to implement the boundary
    condition
    \[
        g'(L) = 0
    \]
    Use the approximation you derived in problem~\ref{P:10.6}
    for the derivative $g'(L)$ to implement this boundary
    condition, i.e.
    \[
        g'(L) \approx
        \frac{1}{2h} g_{N-2} - \frac{2}{h} g_{N-1} + \frac{3}{2h} g_N
    \]
    Explain physically why the resonant frequencies change as
    they do.

\subprob In some problems mixed boundary conditions are
    encountered, for example
    \[
        g'(L) = 2 g(L)
    \]
    Find the first few resonant frequencies and
    eigenfunctions for this case. Look at your eigenfunctions
    and verify that the boundary condition is satisfied.
\end{enumerate}
\end{enumerate}
\ifsolutions
\textit{Solution:}\\
\begin{codeexample}
\begin{VerbatimOut}{\listingFile}
class BVP():

    def __init__(self,a,b,N):
        from numpy import linspace
        self.L = b
        self.N = N
        self.x,self.dx = linspace(a,b,N,retstep=True)

    def initialize(self,T,mu):
        self.T = T
        self.mu = mu

    def loadMatrices(self):
        from numpy import zeros,sin,cos,eye
        # Load A
        self.A = zeros([self.N,self.N])
        self.A[0][0] = 1
        self.A[-1][-1] = 3./(2 * self.dx) - 2.
        self.A[-1][-2] = -2./self.dx
        self.A[-1][-3] = 1./(2 * self.dx)

        for i in range(1,self.N-1):
            self.A[i][i] = -2./self.dx**2
            self.A[i][i+1] = 1./self.dx**2
            self.A[i][i-1] = 1./self.dx**2

        # Load b
        self.B = eye(self.N)
        self.B[0,0] = 0.
        self.B[-1,-1] = 0
        print self.B
        import sys
        #sys.exit()

    def solveProblem(self):
        from scipy.linalg import eig
        from numpy import sqrt,pi
        self.eVals,self.eVecs = eig(self.A,self.B)
        self.omega = sqrt(-self.T/self.mu * self.eVals)/(2 * pi)
        self.key = sorted(range(len(self.omega)), key=lambda k: self.omega[k])
        print self.omega, 'before'
        self.omega = sorted(self.omega)
        print self.omega, 'after'
        import sys
        # sys.exit()
        
    def plot(self,mode):
        from numpy import real
        from matplotlib import pyplot
        prefactor = 1./max(abs(self.eVecs[:,self.key[mode]]))
        print self.eVecs[:,self.key[mode]], 'here'
        print prefactor, 'pfac'
        print max(self.eVecs[:,self.key[mode]]), 'max'
        pyplot.plot(self.x,prefactor * self.eVecs[:,self.key[mode]],'r.-')
        pyplot.title("mode:  " + str(mode + 1)+ '\n' +  str(real(self.omega[mode])) + " Hz")
        pyplot.show()

    
from matplotlib import pyplot
a = 0
b = 1.2
N = 100
T  = 127
mu = 0.003

myBVP = BVP(a,b,N)
myBVP.initialize(T,mu)
myBVP.loadMatrices()
myBVP.solveProblem()
print myBVP.omega[3], 'omega'
myBVP.plot(2)
\end{VerbatimOut}
\end{codeexample}
\else
\noindent\rule{5 in}{0.01 in}
\fi
