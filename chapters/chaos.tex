\chapter{Chaos}
\label{Lab:9}

Today we'll not introduce a new numerical method.  Rather we'll use
the methods that we've already learned to investigate the topic of
chaos.  People often associate chaos or chaotic systems with
undeterminism, or a system (object) whose future state cannot be
determined from information about its past state.  \textbf{This is not
  what chaos is.}  As we see some examples in action, hopefully you
will start to get a feel for the true definition of chaos.  But just
in case you need a formal definition to fall back on, here it is:
\textbf{A chaotic system is one that is extremely sensitive to initial
  conditions}.  In fact, it's so sensitive to the I.C. that it's
virtually unpredicatable.  

Let me give an example to illustrate. Consider the following
conversation between two scientists:\\
\noindent\textbf{Scientist \#1:} I'm going to launch a cannon shell at a
$45^\circ$ angle at a speed of $400$ m/s.  Can you predict where it
will land?\\
\noindent\textbf{Scientist \#2:} Sure, I'll just use Euler's method and
calculate the trajectory.  Your cannon shell will land exactly $25.1$
kilometers away.\\
\noindent\textbf{Scientist \#1:} Great!  Thanks.  What if I changed the launch
angle to $45.01^\circ$.  How would that change the landing location?\\

This last question should have struck you as odd.  Surely this
scientist knows that changing the launch angle by $.01^\circ$ can't
affect the trajectory that much, right?  The landing location will
still be very close to $25.1$ km?  At least close enough that you
wouldn't need to recalculate the trajetory.  Suprisingly, the answer
to this question is a clear and definitive``\textbf{No}'' if the
system is chaotic.  An appropriate response from
scientist \#2 if the system is chaotic would be:\\
\noindent\textbf{Scientist \#2:} I have no idea where your cannon
shell will land.  It could land anywhere.  If you're going to change
the initial conditions
on me I have no idea what's going to happen.\\

Luckily for us, projectiles do not behave chaotically. However, there
are plenty of physical systems that do behave this way.  Let's
investigate further.
\begin{enumerate}
  \probtwo Our first example of a chaotic system is the driven, damped
  pendulum.  Let's investigate further

\begin{enumerate}
\item The differential equation for the driven, damped, pendulum is:
\begin{equation}
\frac{d^2\theta}{dt^2} = - {g\over l} \sin(\theta) - q
\frac{d\theta}{dt} + F_D \sin(\Omega_D t)
\end{equation}

Write this second-order differential equation as two coupled first
order equations.
\item Use fourth-order Runge-Kutta to determine $\theta(t)$  Use the
  following variable values:
\marginfig[-1in]{Figures/PSplot-noChaos.png}{\label{fig:PSnoChaos}Phase
space plot for the normal pendulum ($F_D = 0.5$)}

\begin{eqnarray}
g = 9.8 \mathrm{~m/s^2}& l = 9.8 \mathrm{~m} & q = {1\over 2} \mathrm{~s}^{-1}\\
\Omega_D = {2\over 3}\mathrm{~rads} & F_D = 0 \mathrm{~s}^{-2} & dt = 0.04 \mathrm{~s}\\
\theta(0) = 0.2 \mathrm{~rads}& \omega(0) = 0 \mathrm{~rads/s}& \\
\end{eqnarray}
Plot $\theta$ vs. time for $ 0 < t < 60$ s.  Play with different
values of $q$ to convince yourself that your simulation is behaving
correctly.
\item Now turn the driver on and set $F_D = 0.5$.  Observe the motion
  and convince yourself that it is behaving appropriately.
\item Increase the magnitude of the drive force to $F_D = 1.2$ and
  observe the motion. Explain the behavior.
\marginfig[-1in]{Figures/TwoChaoticPendulums.png}{\label{fig:TwoPendsOne}
  $\Delta \theta$ for two chaotic pendululms ($F_D = 1.2$).}
\item So far, nothing seems too out of the ordinary except that when
  we increased the driving force we got motion that wasn't periodic
  anymore.  Let's take a closer look.  Initialize a second pendulum
  with $\theta(0) = 0.201$ and plot $\theta$ vs. time for $ 0 < t <
  60$ s. Compare this plot, side by side, to the plot with $\theta(0)
  = 0.2$.  What do you notice?

\item Increase the run time of both pendulums to $150$ s, $300$ s, and
  $600$ s and compare.
\item Maybe our time step is too big and numerical errors are simply
  compounding over time.  Decrease the time step ($dt$) and plot
  $\theta$ vs. time to investigate this possibility.
\marginfig[-1in]{Figures/TwoNormalPendulums.png}{\label{fig:TwoPendsTwo}$\Delta \theta$ for two normal pendululms ($F_D = 0.5$).}
\item Finally, let's get a visual on how quickly the position of these
  two pendulums diverge from each other.  Make a plot of $\Delta
  \theta$ vs. time where $\Delta \theta$ is the difference in angular
  position of the two pendulums.  Do this for the normal pendulumm
  ($F_D = 0.5$) and for the chaotic pendulum ($F_D= 1.2$). Scale your
  vertical axis to a log scale.  Your
  graphs should look like figures  \ref{fig:TwoPendsOne} and \ref{fig:TwoPendsTwo}
\end{enumerate}
\end{enumerate}


\ifsolutions
\textit{Solution:}\\
\begin{codeexample}
\begin{VerbatimOut}{\listingFile}
class pendulum():
    def __init__(self,tMax = 60, dt = .01,g=9.8,l= 9.8, q = 1/2.,Fd = 1.2,omegaD = 2/3.,resets = False):
        self.dt = dt
        self.tMax = tMax
        self.g = g
        self.l = l
        self.q = q
        self.Fd = Fd
        self.omegaD = omegaD
        self.resets = resets

    def setInitialConditions(self,theta,omega):
        self.theta = [theta]
        self.omega = [omega]
    def getDerivs(self,varsVec,t):
        from numpy import array,pi,sin
        from numpy.linalg import norm
        # Varsvec [theta,omega]
        dtheta = varsVec[1]
        dOmega = - self.g/self.l * sin(varsVec[0]) - self.q * varsVec[1] + self.Fd * sin(self.omegaD * t)

        return array([dtheta,dOmega])


    def RK4(self):
        from numpy import array

        r = array([self.theta[-1],self.omega[-1]])
        t = 0
        self.time = [t]
        while t < self.tMax:
            k1 = self.dt * self.getDerivs(r,t)
            k2 = self.dt * self.getDerivs(r + 1/2. * k1,t + 1/2. * self.dt)
            k3 = self.dt * self.getDerivs(r + 1/2. * k2,t + 1/2. * self.dt)
            k4 = self.dt * self.getDerivs(r + k3,t +  self.dt)
            r += 1/6. * (k1 + 2 * k2 + 2 * k3 + k4 )
            if r[0] > pi and self.resets:
                r[0] -= 2 *pi
            elif r[0] < -pi and self.resets:
                r[0] += 2 *pi
            self.theta.append( r[0]  )
            self.omega.append(r[1])
            t += self.dt
            self.time.append(t)



    def plot(self):
        from matplotlib import pyplot
        pyplot.figure()
        pyplot.plot(self.time,self.theta)
        #pyplot.show()

    def phaseSpace(self,show = True):
        from matplotlib import pyplot
        pyplot.scatter(self.theta,self.omega)
        pyplot.xlabel(r"$\theta$ (rads)")
        pyplot.ylabel(r"$\omega$ (rads/s)")
        font = {'family':'serif',
                'color': 'darkred',
                'weight': 'normal',
                'size':28}
        pyplot.text(-0.6,0.68,r"$\omega$ versus $\theta$     $F_D$ = " + str(self.Fd),fontdict=font)
        if show:
            pyplot.show()
        else:
            pyplot.savefig('PSplot-noChaos.png')
from numpy import pi,array,abs
from matplotlib import pyplot
import sys
theta = 0.2
omega = 0
pendulumOne = pendulum(tMax = 150,dt = .04,Fd = 0.5,q=0.5,resets=True)
pendulumOne.setInitialConditions(theta,omega)
pendulumOne.RK4()
pendulumOne.phaseSpace(show=False)
#pendulumOne.plot()
sys.exit()

theta = 0.201
omega = 0
pendulumTwo = pendulum(tMax = 150,dt = .04,Fd = 0.5,q=0.5)
pendulumTwo.setInitialConditions(theta,omega)
pendulumTwo.RK4()
#pendulumTwo.plot()
pyplot.plot(pendulumOne.time,array(pendulumOne.theta) - array(pendulumTwo.theta))
pyplot.yscale('log')
pyplot.show()
\end{VerbatimOut}
\end{codeexample}
\else
\noindent\rule{4 in}{0.01 in}
\fi


As you can see, changing the initial position of the oscillator by a
very small amount ($.001 \mathrm{~rads} \approx {1\over 2}^\circ$) resulted in
very different motion.  That's how a chaotic system behaves!  The future motion (state of
the system) is unpredictable, but it's not indeterminable.  The
pendulum still obeys the deterministic laws of physics (which allows
us to use a numerical algorithm to calculate future motion) but is
unpredictable due to the extreme sensitivity to the initial
conditions.

\marginfig[-1in]{Figures/PSplot-Chaos.png}{\label{fig:PSChaos}Phase
space plot for the chaotic pendulum ($F_D = 1.2$)}

\begin{enumerate}
\probtwo A plot of $\theta$ vs. time for the chaotic pendulum looks
pretty structureless, with no clear pattern to the motion.  Scientists
like to look for patterns anyway they can.   
\begin{enumerate}
\item Construct a phase-space diagram for the driven,damped pendulum
  for $F_D = 1.2$ and $F_D = 0.5$.  Compare the plots.  A phase-space
  diagram is a plot of $\omega$ vs. $\theta$.  You probably don't have
  much intuition for this kind of plot but it is a useful way to spot
  structure and trends in a seemingly random and unpredictable system.
  Your graphs should look similar to figures \ref{fig:PSnoChaos} and
  \ref{fig:PSChaos}.
\item Instead of plotting every ($\omega$,$\theta$) pair, only plot those data points that are in-phase with
  the driving frequency.  In other words, only plot those points that
  satisfy
\begin{equation}
\Omega_D t = 2 \pi n
\end{equation}
where $n$ is an integer.  You can think of this plot as if you put a
strobe light on your phase-space diagram from part (a) and you only
plotted those points that you see when the strobe appears.  This graph
is called a \textit{Poincar\'{e} section} and is useful for spotting
structure in chaotic systems. Your graph should look like figure
\ref{fig:poincare}

\marginfig[-1in]{Figures/poincare.png}{\label{fig:poincare}\textit{Poincar\'{e}
  section} for the chaotic pendulum ($F_D = 1.2$).  The pendulum ran
for $15000$ s.}
\end{enumerate}
\end{enumerate}
\ifsolutions
\textit{Solution:}\\
\begin{codeexample}
\begin{VerbatimOut}{\listingFile}
class pendulum():
    def __init__(self,tMax = 60, dt = .01,g=9.8,l= 9.8, q = 1/2.,Fd = 1.2,omegaD = 2/3.,resets = False):
        self.dt = dt
        self.tMax = tMax
        self.g = g
        self.l = l
        self.q = q
        self.Fd = Fd
        self.omegaD = omegaD
        self.resets = resets

    def setInitialConditions(self,theta,omega):
        self.theta = [theta]
        self.omega = [omega]
    def getDerivs(self,varsVec,t):
        from numpy import array,pi,sin
        from numpy.linalg import norm
        # Varsvec [theta,omega]
        dtheta = varsVec[1]
        dOmega = - self.g/self.l * sin(varsVec[0]) - self.q * varsVec[1] + self.Fd * sin(self.omegaD * t)

        return array([dtheta,dOmega])


    def RK4(self):
        from numpy import array

        r = array([self.theta[-1],self.omega[-1]])
        t = 0
        self.time = [t]
        while t < self.tMax:
            k1 = self.dt * self.getDerivs(r,t)
            k2 = self.dt * self.getDerivs(r + 1/2. * k1,t + 1/2. * self.dt)
            k3 = self.dt * self.getDerivs(r + 1/2. * k2,t + 1/2. * self.dt)
            k4 = self.dt * self.getDerivs(r + k3,t +  self.dt)
            r += 1/6. * (k1 + 2 * k2 + 2 * k3 + k4 )
            if r[0] > pi and self.resets:
                r[0] -= 2 *pi
            elif r[0] < -pi and self.resets:
                r[0] += 2 *pi
            self.theta.append( r[0]  )
            self.omega.append(r[1])
            t += self.dt
            self.time.append(t)



    def plot(self):
        from matplotlib import pyplot
        pyplot.figure()
        pyplot.plot(self.time,self.theta)
        #pyplot.show()

    def phaseSpace(self,show = True):
        from matplotlib import pyplot
        pyplot.scatter(self.theta,self.omega)
        pyplot.xlabel(r"$\theta$ (rads)")
        pyplot.ylabel(r"$\omega$ (rads/s)")
        font = {'family':'serif',
                'color': 'darkred',
                'weight': 'normal',
                'size':28}
        pyplot.text(-2.5,2.5,r"$\omega$ versus $\theta$     $F_D$ = " + str(self.Fd),fontdict=font)
        pyplot.xticks(fontsize=30)
        pyplot.yticks(fontsize=30)
        if show:
            pyplot.show()
        else:
            pyplot.savefig('PSplot-Chaos.png')


    def poincare(self):
        from numpy import pi
        from matplotlib import pyplot
        
        period = 2 * pi/self.omegaD
        inPhaseTimes = [n * period for n in range(0,int(self.tMax/period))]
        indices = [abs((array(self.time) - n)).argmin() for n in inPhaseTimes]
        specialTheta = [self.theta[n] for n in indices]
        specialOmega = [self.omega[n] for n in indices]
        pyplot.scatter(specialTheta,specialOmega)
        pyplot.xlabel(r"$\theta$ (rads)",fontsize=30)
        pyplot.ylabel(r"$\omega$ (rads/s)",fontsize=30)
        pyplot.xticks(fontsize=30)
        pyplot.yticks(fontsize=30)

        pyplot.show()

from numpy import pi,array,abs
from matplotlib import pyplot
import sys
theta = 0.2
omega = 0
pendulumOne = pendulum(tMax = 15000,dt = .04,Fd = 1.2,q=0.5,resets=True)
pendulumOne.setInitialConditions(theta,omega)
pendulumOne.RK4()
#pendulumOne.phaseSpace(show=False)
pendulumOne.poincare()
#pendulumOne.plot()
\end{VerbatimOut}
\end{codeexample}
\else
\noindent\rule{4 in}{0.01 in}
\fi


\labsection{Homework}
\begin{enumerate}
  \prob Weather systems are well-known to behave chaotically.  Edward
  Lorenz was an atmospheric scientist in the mid 1900s that discovered
  this. He built atmospheric models and noticed that his models would
  predict very differently when the initial conditions were rounded
  (in an seemingly inconsequential fashion) compared to the results
  the emerged from the unrounded initial conditions. He famously
  hypothesized that a butterfly could flap his wings in Africa and the
  disturbance would cause a tornado in North America.  This is well
  known as the butterfly effect today.\sidenote{The term ``butterfly
    effect'' has come to be a more general term since Lorenz
    originally coined it in reference to weather systems. Today, the
    butterfly effect is used to denote any situation where a small
    cause can have large effects.}The differential equations that
  emerge from Lorenz's model are:

\begin{align}
\frac{dx}{dt} &= \sigma \left( y - x\right)\\
\frac{dy}{dt} &= - x z + rx - y\\
\frac{dz}{dt} &= xy - bz\\
\end{align}
The variables $x$, $y$, and $z$ can loosely be compared to
temperature, density, and velocity of a fluid although this is a
pretty drastic oversimplification

\begin{enumerate}
\item Solve this set of first-order differential equations using
fourth-order Runge-Kutta.  Set $ \sigma = 10$, $b = {8\over 3}$ and $r
= 10$, $x(0) = 1$, $y(0) = 0$, and $z(0)=0$.  Plot $z$ vs. time.
\item Now set $r = 25$ and plot $z$ vs. time.  Compare to the previous
  plot.
\item Plot $z$ vs. $x$.
\item Plot a full three dimensional plot of $x$, $y$, and $z$ (Look in
  the python book to see how to generate a 3D plot.)
\end{enumerate}
\end{enumerate}
\ifsolutions
\textit{Solution:}\\
\begin{codeexample}
\begin{VerbatimOut}{\listingFile}
class lorenz():
    def __init__(self,tMax = 60, dt = .01,sigma = 10, b = 8./3., r = 5):
        self.dt = dt
        self.tMax = tMax
        self.sigma = sigma
        self.b = b
        self.r = r
        
    def setInitialConditions(self,x,y,z):
        self.x = [x]
        self.y = [y]
        self.z = [z]
        
    def getDerivs(self,varsVec):
        from numpy import array,pi,sin
        from numpy.linalg import norm
        # Varsvec [theta,omega]
        x = varsVec[0]
        y = varsVec[1]
        z = varsVec[2]
        dx = self.sigma * (y - x) 
        dy = - x * z + self.r * x - y 
        dz = x * y - self.b * z
        return array([dx,dy,dz])


    def RK4(self):
        from numpy import array

        r = array([self.x[-1],self.y[-1],self.z[-1]])
        t = 0
        self.time = [t]
        while t < self.tMax:
            k1 = self.dt * self.getDerivs(r)
            k2 = self.dt * self.getDerivs(r + 1/2. * k1)
            k3 = self.dt * self.getDerivs(r + 1/2. * k2)
            k4 = self.dt * self.getDerivs(r + k3)
            r  +=1/6. * (k1 + 2. * k2 + 2. * k3 + k4 )

            self.x.append( r[0]  )
            self.y.append(r[1])
            self.z.append(r[2])
            t += self.dt
            self.time.append(t)


    def leapfrog(self):
        from numpy import arange,sin,array
        from numpy.linalg import norm

        # Use Euler to perform the first half time step
        r = array([self.x[-1],self.y[-1],self.z[-1]])
        k1 = self.dt * self.getDerivs(r)
        midPointVars = r + 1/2. * k1
        # Done with first half time step.

        t = 0
        self.time = [t]
        
        while t < self.tMax:
            # Using the midpoint, make a full step
            k2 = self.dt * self.getDerivs(midPointVars)
            r += k2
            # Step from the previous midPoint to the next midPoint.
            midPointVars = midPointVars + self.dt * self.getDerivs(r)
            self.x.append(r[0])
            self.y.append(r[1])
            self.z.append(r[2])
            t += dt
            self.time.append(t)
            


    def plot(self):
        from matplotlib import pyplot
        pyplot.figure()
        pyplot.plot(self.time,self.z)
        pyplot.show()

    def phaseSpace(self,show = True):
        from matplotlib import pyplot
        pyplot.scatter(self.x,self.z)
        if show:
            pyplot.show()

    def plot3D(self):
        from mpl_toolkits.mplot3d import Axes3D
        fig = pyplot.figure()
        fig.gca(projection= '3d')
        pyplot.plot(self.x,self.y,self.z)
        pyplot.axis('equal')
        pyplot.show()

from numpy import pi,array,abs
from matplotlib import pyplot
import sys
x = 1.
y = 0.
z = 0.

weatherOne = lorenz(r = 25)
weatherOne.setInitialConditions(x,y,z)
weatherOne.RK4()
#weatherOne.phaseSpace()
weatherOne.plot3D()
#weatherOne.plot()
sys.exit()
\end{VerbatimOut}
\end{codeexample}
\else
\noindent\rule{4 in}{0.01 in}
\fi
