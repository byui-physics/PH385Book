\chapter*{Review}
\addcontentsline{toc}{chapter}{Review}

If you are like most students, loops and logic gave you trouble in
330. We will be using these programming tools extensively this
semester, so you may want to review and brush up your skills a bit.
Here are some optional problems designed to help you remember your
loops and logic skills. You will probably need to use online help
(and you can ask a TA to explain things in class too).

\begin{enumerate}
\subprob \index{For loop} \index{Loops!for} Write a {\tt for}
    loop that counts by threes starting at 2 and ending at 101.
    Along the way, every time you encounter a multiple of 5 print
    a line that looks like this (in the printed line below it
    encountered the number 20.)
\begin{Verbatim}
fiver: 20
\end{Verbatim}
    You will need to use the commands {\tt for}, {\tt mod}, and
    {\tt fprintf}, so first look them up in online help.

\subprob Write a loop that sums the integers from 1 to $N$, where
    $N$ is an integer value that the program receives via the
    {\tt input} command. Verify by numerical experimentation that
    the formula
    \[
        \sum_{n=1}^N n = \frac{N(N+1)}{2}
    \]
    is correct

\subprob For various values of $x$ perform the sum
    \[
        \sum_{n=1}^{1000} n x^n
    \]
    with a {\tt for} loop and verify by numerical experimentation
    that it only converges for $|x| < 1$ and that when it does
    converge, it converges to $x/(1-x)^2$.

\index{While loop} \index{Loops!while}
\subprob Redo (c) using a {\tt while} loop (look it up in online
    help.) Make your own counter for $n$ by using $n=0$ outside
    the loop and $n=n+1$ inside the loop. Have the loop execute
    until the current term in the sum, $n x^n$ has dropped below
    $10^{-8}$. Verify that this way of doing it agrees with what
    you found in (c).

\subprob Verify by numerical experimentation with a {\tt while}
    loop that
    \[
        \sum_{n=1}^{\infty} \frac{1}{n^2} = \frac{\pi^2}{6}
    \]
    Set the {\tt while} loop to quit when the next term added to
    the sum is below $10^{-6}$.

\subprob Verify, by numerically experimenting with a {\tt for}
    loop that uses the {\tt break} command (see online help) to
    jump out of the loop at the appropriate time, that the
    following infinite-product relation is true:
    \[
        \prod_{n=1}^{\infty} \left( 1 + \frac{1}{n^2} \right)
        = \frac{ \sinh{\pi} }{ \pi }
    \]

\subprob \index{Iteration} \index{Successive substitution} Use a
    {\tt while} loop to verify that the following three iteration
    processes converge. (Note that this kind of iteration is
    often called successive substitution.) Execute the loops
    until convergence at the $10^{-8}$ level is achieved.
    \[
        x_{n+1} = e^{-x_n}~~~~;~~~~
        x_{n+1} = \cos{x_n}~~~~;~~~~
        x_{n+1} = \sin{2 x_n}
    \]
    Note: iteration loops are easy to write. Just give $x$ an
    initial value and then inside the loop replace $x$ by the
    formula on the right-hand side of each of the equations
    above. To watch the process converge you will need to call
    the new value of $x$ something like {\tt xnew} so you can
    compare it to the previous $x$.

    Finally, try iteration again on this problem:
    \[
        x_{n+1} = \sin{3 x_n}
    \]
    Convince yourself that this process isn't converging to
    anything. We will see in Lab~\ref{Lab:10} what makes the
    difference between an iteration process that converges and
    one that doesn't.
\end{enumerate}

\mainmatter

\pagestyle{fancy}
\renewcommand{\chaptermark}[1]{\markboth{Computational Physics 385}{\chaptername \ \thechapter \ \ #1}}
