\chapter{The Discrete Fourier Transform}
\label{Lab:16}

Before we move on from our study of the wave equation and
oscillations, we'll cover the topic of Fourier analysis.  Fourier
analysis is a powerful and widely-used technique.  It has applications
in signal processing, image processing, and many physics
applications.  Fourier techniques are helpful for breaking down,
anaylzing, smoothing, and filtering any osciallatory function.  They
are also helpful when solving differential equations.


\labsection{The Fourier Series} 

By now you should know that any function can be written as a sum of
sine and cosine functions.  More precisely, if $f(t)$ is defined on
the interval $0 \le x < \tau$, we can express this function as:
\begin{equation}\label{eq:FourierSeries}
f(t) = \sum_{k=0}^{\infty} \alpha_k \cos{ \frac{2 \pi k t}{\tau}} + \sum_{k=1}^{\infty} \beta_k \sin{ \frac{2 \pi k t}{\tau}} 
\end{equation}
We can compress the notation by using the following identities:
$\cos\theta ={1\over 2}\left( e^{-i\theta} + e^{i \theta}\right)$ and $\sin\theta
={1\over 2}i \left(e^{-i\theta} - e^{i \theta}\right)$.  Upon collecting like
terms, equation \ref{eq:FourierSeries} becomes:
\begin{equation}\label{eq:FourierComplex}
f(t) = \sum_{k=-\infty}^{\infty} \gamma_k e^{i \frac{2 \pi k t}{\tau}}
\end{equation}
with
\begin{equation}\label{eq:cases}
\gamma_k = 
\begin{cases}
{1\over 2} \left( \alpha_{-k} + i \beta_{-k}\right) & \mathrm{~if~~} k < 0\\
\alpha_0 & \mathrm{~if~~} k = 0\\
{1\over 2} \left( \alpha_{k} - i \beta_{k}\right) & \mathrm{~if~~} k > 0\\
\end{cases}
\end{equation}

\begin{enumerate}
\probtwo Derive equation \ref{eq:FourierComplex} by inserting the
identities given above into equation \ref{eq:FourierSeries}. 
\end{enumerate}


\labsection{The Fourier Transform}
Finding the coefficients ($\gamma_k$) in equation
\ref{eq:FourierComplex} is valuable because it helps us know which of
the sine and cosine functions should be included in the expansion.
Or, to put it another way, if we knew which coefficients were not
zero, we would know which frequencies were present in the function
$f(t)$.  

The standard approach to finding the $\gamma_k$ is to exploit the
orthogonality of the functions $e^{i \frac{2 \pi k t}{\tau}}$.  If we
multiply both sides of equation \ref{eq:FourierComplex} by $e^{-i
  \frac{2 \pi k' t}{\tau}}$ and integrate both sides, we get:
\begin{align}
\int_0^\tau f(t) e^{-i
  \frac{2 \pi k' t}{\tau}}dt &= \sum_{k=-\infty}^{\infty} \gamma_k
\int_0^\tau e^{i \frac{2 \pi k t}{\tau}}e^{-i
  \frac{2 \pi k' t}{\tau}} dt\\
\int_0^\tau f(t) e^{-i
  \frac{2 \pi k' t}{\tau}}dt &= \sum_{k=-\infty}^{\infty} \gamma_k
\int_0^\tau e^{i \frac{2 \pi (k-k') t}{\tau}} dt \label{eq:FourierTrick}
\end{align}

The integral on the right hand side of Eq. \ref{eq:FourierTrick} will
be zero unless $k = k'$.

\begin{enumerate}
\probtwo Use Mathematica to verify that the integral on the right hand
side of equation \ref{eq:FourierTrick} is zero if $k \ne k'$.  Just
pick a value for $\tau$.  What is the value of the integral when $k = k'$?
\end{enumerate}

Since all of the integrals in the sum evaluate to zero except when $k
= k'$, the sum essentially collapses down to one term and equation
\ref{eq:FourierTrick} becomes:
\begin{equation}
\int_0^\tau f(t) e^{-i
  \frac{2 \pi k' t}{\tau}}dt = \tau \gamma_k'
\end{equation}
 and we can solve for $\gamma_k$:
\begin{equation}\label{eq:gamma}
\gamma_k = {1 \over \tau} \int_0^\tau f(t) e^{-i
  \frac{2 \pi k t}{\tau}}dt 
\end{equation}

\labsection{The Discrete Fourier Transform}

Equation \ref{eq:gamma} is great if \textbf{you know the function
  $f(t)$}.  However, there are many situations when, instead of knowing
the function you simply have samples of the function \textbf{at regular intervals}:
\[f(t_n) \]
where 
\[t_n = \frac{n}{N} \tau\]

in this case, the integral in equation \ref{eq:gamma} becomes a sum
over the function samples:

\begin{equation}\label{eq:Discretegamma}
\gamma_k = {1 \over N} \sum_0^N f(t_n) e^{-i
  \frac{2 \pi k n}{N}}dt 
\end{equation}

\begin{enumerate}
\probtwo Let's try equation \ref{eq:Discretegamma} out on a few simple
examples to get a feel for how it works.  We'll start by working with
a simple one-frequency function:
\begin{equation}\label{eq:samplingFunction}
f(t) = \sin(5 \times 2 \pi t)
\end{equation}
\begin{enumerate}
\subprob Even though we have the analytic form of the function, let's
pretend that we don't know what it is and just extract some samples
from the function.  Sample this function at a rate of $15$ Hz from $t
= 0$ to $t = 1$ s.  
\subprob Now use these samples to calculate the discrete
Fourier transform using equation \ref{eq:Discretegamma}.  
\subprob Your probably noticed that your DFT had two peaks in it: one at $5$ Hz
and another at $10$ Hz.  What's going on here?   Let's try increasing
the sampling rate to $300$ Hz and see what happens.

You should have seen two peaks again, but this time one was at $5$ Hz
and the other at $295$
Hz.  You expected the one at $5$ Hz to be there but the other one is
clearly incorrect.  It turns out that only half (150) of the coefficients
that you calculated were meaningful.  The other half were simply
complex conjugates of the first $150$.  In other words, \textbf{even though
your sampling rate was $300$ Hz, you were only able to detect
frequencies up to $150$ Hz}. Lock that last statement away, it's very key.
\subprob  Investigate the validity of this last statement by modifying
your sampling function to be:
\begin{equation}\label{eq:samplingFunction}
f(t) = \sin(5 \times 2 \pi t) + \sin(170 \times 2 \pi t)
\end{equation}
and sampling at a rate of $300$ Hz.  Now you have two frequencies that
you'd like to detect.  Is your DFT able to correctly detect both
of them?

A simple statement will help you remember the key important rule:
\textit{A DFT can only detect frequencies up to half the sampling
  rate!} This is called Nyquist's theorem.
\subprob  What this all amounts to is that the sum in equation
\ref{eq:Discretegamma} shouldn't go to $N$ but rather to $N/2$
so that we don't calculate redundant coefficients.  Modify your DFT
code to incorporate this change and re-plot the DFT for the sampling
function in \ref{eq:samplingFunction}
\marginfig{Figures/aliasing}{Aliasing;  If your sampling rate is too
  low, you won't be able to distinguish the true signal from a
  lower-frequency version (the alias) of the signal.}

\subprob Now modify your sampling rate so that you can detect the
$170$ Hz signal and recalculate the DFT.
\subprob Calculate the DFT for the
following function:
\begin{equation}
f(t) = \sin(5 \times 2 \pi t) + 3 \sin(170 \times 2 \pi t)
\end{equation}
Are you able to detect that the $170$ Hz signal is stronger than the
$5$ Hz signal?
\subprob  Investigate what happens if you change the sampling time interval? 
\end{enumerate}
\end{enumerate}
\ifsolutions
\textit{Solution:}\\
\begin{codeexample}
\begin{VerbatimOut}{\listingFile}

def DFT(samples):
    from numpy import exp
    N = len(samples)
    gamma = []
    for k in range(N//2+1):
        gammaK = 0
        for n,yn in enumerate(samples):
            gammaK += yn * exp(-2j * pi * k * n/N )
        gamma.append(2 * gammaK/N)

    return gamma


from numpy import linspace,sin,pi,arange,abs
from matplotlib import pyplot

fS = 400  #Rate at which I sample the function
dt = 1/fS  # time between adjacent samples
samplingTime = 0.25  # How long do I sample my function
nSamples = samplingTime/dt  # How many samples am I going to take


t = linspace(0,samplingTime,nSamples)  #Location of samples in domain

f = sin(5 * 2 * pi * t) + 3 * sin(170 * 2 * pi * t) # Function samples
gamma = DFT(f)
k = linspace(0,fS//2,nSamples/2 + 1)
print(k)
pyplot.figure(1)
pyplot.scatter(t,f)
pyplot.figure(2)
pyplot.scatter(k,abs(gamma))
pyplot.show()
\end{VerbatimOut}
\end{codeexample}
\else
\noindent\rule{5 in}{0.01 in}
\fi

One key point from that last exercise was that the sampling frequency
is very important when performing a discrete Fourier transform.  To
help illustrate this, consider the $1.2$ Hz signal shown in black in
figure \ref{Figures/aliasing}.  Sampling this function regularly using
a sampling rate of $1$ Hz would extract the sample points shown by
the red dots.  Furthermore, this same $1$ Hz sampling rate would
\textbf{extract the same exact samples} from a $0.2$ Hz signal (shown
in orange). How is your DFT algorithm suppose to know which signal is
truly present?  The answer is that it doesn't know; it can't know.
Because the sampling rate ($1$ Hz) is not equal to twice the frequency
of the $1.2$ Hz signal, you cannot extract the higher-frequency
information.  The DFT algorithm will indicate the presence of the
$0.2$ Hz frequency even if the true signal was actually composed of a
$1.2$ Hz signal.  The ``fake'' signal is called an alias and this
problem is generally referred to as ``aliasing''.
\begin{enumerate}
\probtwo 
\begin{enumerate}
\subprob Perform a discrete Fourier transform on the function:
\begin{equation}
f(t) = \cos(5 \times 2 \pi t) + 3 \sin(170 \times 2 \pi t)
\end{equation}
Can you detect the $5$ Hz component of the signal?
\subprob Now look closer at the $\gamma_k$ coefficients and compare
to equation \ref{eq:cases}.  
\end{enumerate}
\end{enumerate}

\labsection{Homework}

\begin{enumerate}
\prob Let's incorporate our new knowledge about Fourier transforms
into some of the previous problems we have done.
\begin{enumerate}
  \subprob Return to your code from problem \ref{P:14.2} and add a
  member function that calculates the discrete Fourier transform.
  \subprob Add some code into your \texttt{animate} function that will
  collect samples in time at $x = 0.5$.  Note: Sampling at every
  iteraion will amount to a very high sampling rate and you will end
  up collecting way too much data. Your DFT function probably won't be
  able to handle it.  You should find a way to sample less frequently.\\

\subprob Now call the DFT routine from part (a) to calculate the
Fourier transform of the samples that you collected.

\subprob You probably noticed that if your dataset gets too big, your
DFT function crashes or takes a really long time.  Numpy has some
builtin Fourier transform functions that are optimized to be very
fast. You can import them like this:
\begin{Verbatim}
from numpy.fft import rfft
\end{Verbatim}
(\texttt{fft} stands for Fast Fourier Transform)
Try using the builtin function to see what kind of a speedup you get.

\subprob  Now sample the wave at a different spatial location and
compare the frequency components to the ones you got from part (c).
\end{enumerate}
\end{enumerate}
\ifsolutions
\textit{Solution:}\\
\begin{codeexample}
\begin{VerbatimOut}{\listingFile}


class animatedWave():

    def __init__(self,a,b,N,tau,mu,T,omegaD,tMax,gamma,stabilityCheck = False, c = 2):
        from numpy import arange,sqrt
        self.L = b
        self.N = N
        self.tMax = tMax
        self.tau = tau
        self.samplingRate = 1/tau/10
        self.mu = mu
        self.gamma = gamma
        self.T = T
        self.omegaD = omegaD
        self.dx = (b - a)/N
        self.x = arange(a - self.dx/2., b + self.dx,self.dx)

        self.c = sqrt(T/mu)

        if self.tau > self.dx/self.c and stabilityCheck:
            print('Beware: your algorithm will be unstable.')
            import sys
            sys.exit()

    def f(self,x):
        import numpy
        if type(x) is numpy.ndarray:
            from numpy import array
            return array([self.f(var) for var in x])
        else:
            if 0.8 <= x <= 1.0:
                return 0.73
            else:
                return 0
    def DFT(self,samples):
        from numpy import exp,pi
        N = len(samples)
        self.gamma = []
        for k in range(N//2 + 1):
            gammaK = 0
            for n,yn in enumerate(samples):
                gammaK += yn * exp(-2j * pi * k * n/N )
            self.gamma.append(2 * gammaK/N)
        

    def initializeWave(self):
        from numpy import exp,zeros
        # Gaussian initial displacement
        self.v = 2 * exp( -160/self.L**2 * (self.x - self.L/2.)**2 ) - exp( -160/self.L**2 * (0 - self.L/2.)**2 )
        # Zero initial velocity
        self.y = zeros(self.N+2)


    def animate(self):
        from numpy import zeros_like,copy,cos
        from matplotlib import pyplot
        constant = self.c**2 * self.tau**2/(2 * self.dx**2)

        constantOne = 2. + self.gamma * self.tau
        constantTwo = 2. * self.c**2 * self.tau**2/self.dx**2

        yOld = zeros_like(self.y)
        yOld[1:self.N + 1] = -self.v[1:self.N + 1]* self.tau + 1./constantOne * (4 * self.y[1:self.N + 1] + constantTwo * (self.y[2:self.N + 2] - 2 * self.y[1:self.N + 1] + self.y[0:self.N])) + self.f(self.x[1:self.N + 1])/self.mu  * 2 * self.tau**2/constantOne
        yOld[0] = -yOld[1]
        yOld[-1] = -yOld[-2]
        t = 0
        counter = 0
        self.samples = []
        while t < self.tMax:
            yNew = zeros_like(self.y)
            yNew[1:self.N + 1] = 1/(2 + self.gamma * self.tau) * (4 * self.y[1:self.N + 1] - 2 * yOld[1:self.N + 1] + self.gamma * self.tau * yOld[1:self.N + 1]+ 4 * constant * (self.y[2:self.N + 2] - 2 * self.y[1: self.N + 1] + self.y[0:self.N])) + self.f(self.x[1:self.N + 1])/self.mu * cos(self.omegaD * t) * 2 * self.tau**2/(2 + self.gamma * self.tau)
            yNew[0] =  -yNew[1]
            yNew[-1] = - yNew[-2]
            if counter % 10 == 0:
                self.samples.append(self.y[3* self.N//4])
#            if counter % 5000 == 0:
#                pyplot.plot(self.x,self.y,'r.-')
#                pyplot.ylim(-0.005,0.005)
#                pyplot.draw()
#                pyplot.pause(.000001)
#                pyplot.clf()


            yOld = copy(self.y)
            self.y = copy(yNew)

            print(t)
            t += self.tau
            counter += 1

    def plotDFT(self):
        from numpy import linspace,abs
        from matplotlib import pyplot
        nSamples = len(self.samples)
        k = linspace(0,self.samplingRate/2,nSamples/2 + 1)
        pyplot.figure(2)
        pyplot.plot(k,abs(self.gamma))
        print('here')
        pyplot.show()
            
            
a = 0
b = 1.2
N = 200
tau = 0.0001
gamma = 60
tMax = 2.
mu = .003
T = 2.7
omegaD = 1080.


myWave = animatedWave(a,b,N,tau,mu,T,omegaD,tMax,gamma,stabilityCheck = True)
myWave.initializeWave()
myWave.animate()
myWave.DFT(myWave.samples)
myWave.plotDFT()

\end{VerbatimOut}
\end{codeexample}
\else
\noindent\rule{5 in}{0.01 in}
\fi



