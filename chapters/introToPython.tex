\chapter{A short introduction to Python}
%\addcontentsline{toc}{chapter}{Review}
One goal of this class is that you become proficient with Python in a
scientific setting.  This will not happen all at once, but gradually
over the course of the entire semester.  You have already been exposed
to python in PH295 (previously PH291) but if you are like most
students, you might have struggled with loops, logic and functions.
These basic programming concepts are pretty crucial in this class.  In
this chapter, we'll review some python syntax and practice with some problems.

\labsection{Data Types} The most commonly used types of data that will
be used for this class are: integers, floats, strings, lists, and
arrays.  

\ul{Read sections 3.1 -- 3.6} in the Python manual to learn
  about the first four data types and \ul{sections 5.1--5.3} to learn
  about arrays.  If something you read is confusing or unfamiliar to
  you, I suggest that you try to recreate the calculation yourself.
  Take good notes so you can ask good questions during class.

Try the problem belwo to test your ability
\begin{enumerate}
\probtwo 

\item Build an array containing all multiples of 3 starting at 3 and
  ending at 2000.
\item Calculate $3 x^2 \sin(x) + \ln(x)$, where $x$ is the list you
  just created.  In other words, for every number in the array you
  just created, calculate the quantity above.  This will yield an
  array of calculated values.
\item Sum the elements of the list to get a single final answer.  
\end{enumerate}
Ans: -736905.545292\\
\ifsolutions
\textit{Solution:}\\
\begin{codeexample}
\begin{VerbatimOut}{\listingFile}
from numpy import arange,sin,log

x = arange(3,2000,3)
y = 3 *x**2 *sin(x) + log(x)
result = sum(y)

print result
\end{VerbatimOut}
\end{codeexample}
\else
\noindent\rule{5 in}{0.01 in}
\fi


\labsection{Loops and Logic} A loop is a set of instructions to be
performed until some pre-determined criteria is satisfied.  Loops are
used heavily in this class and you must grasp what they are as soon as
possible.  Logic refers to the act of checking the truthfulness of a
statement and is used heavily in conjunction with loops.

\ul{Read chapter 6 in the Python manual} to get a
better feel for what we mean. If something you read is confusing
  or unfamiliar to you, I suggest that you try to recreate the
  calculation yourself.  Take good notes so you can ask good questions
  during class.

Try the following problem to test your understanding of \texttt{for} loops
\begin{enumerate}
\probtwo On iLearn, you will find a file entitled ``massdata.txt''
which contains the mass of a collection of objects and their $(x,y,z)$
coordinates.
\begin{enumerate}
\item In Python, read in the data from the file (see chapter 8 in the python book).
\item Use a 'for' loop to loop over the data points and calculate the
  center of mass coordinates for the collection of particles.  You may
  remember that the equation find the center of mass is
\[x_\mathrm{cm} = {1\over M}\sum_i^N x_i m_i\]
\end{enumerate} 
\ifsolutions
\textit{Solution:}\\
\begin{codeexample}
\begin{VerbatimOut}{\listingFile}
with open('massdata.txt', 'r') as f:
    data = f.readlines()

masses = [x.split()[3] for x in data]

totalMass = sum(masses)
xCM = 0
yCM = 0
zCM = 0
for dPoint in data:
    x = float(dPoint.split()[0])
    y = float(dPoint.split()[1])
    z = float(dPoint.split()[2])
    m = float(dPoint.split()[3])
    xCM += x * m
    yCM += y * m
    zCM += z * m

xCM /= totalMass
yCM /= totalMass
zCM /= totalMass
print(xCM, yCM,zCM)
\end{VerbatimOut}
\end{codeexample}
\else
\noindent\rule{5 in}{0.01 in}
\fi
\end{enumerate}
Try the following problem to test your understanding of \texttt{while}
loops
\begin{enumerate}

    \probtwo Perform the summation below
    \[
        \sum_{n=1}^{\infty} n x^n
    \]
    using a {\tt while} loop.  Make your own counter for $n$ by using $n=0$
    outside the loop and $n=n+1$ inside the loop. Have the loop
    execute until the current term in the sum, $n x^n$ has dropped
    below $10^{-8}$. Verify that the sum only converges for $|x| < 1$
    and that when it does converge, it converges to
    $x/(1-x)^2$.\\
 Caution:  When building \texttt{while} loops, you can
    easily get caught in an infinite loop, one that never ends.  It's
    always wise to build in a fail safe: A variable to count the
    number of iterations and that breaks out of that loop if the
    number gets too high.\\
\ifsolutions
\textit{Solution:}\\
\begin{codeexample}
\begin{VerbatimOut}{\listingFile}
x = 0.75
thesum = 0
n = 1
while n * x**n > 1e-9:
    thesum += n * x**n 
    n = n + 1
    if n > 10000:
        break

checkVal = x /(1 -x)**2
print thesum, checkVal
\end{VerbatimOut}
\end{codeexample}
\else
\noindent\rule{5 in}{0.01 in}
\fi
\end{enumerate}

%Write a {\tt for}
%    loop that counts by threes starting at 2 and ending at 101.
%    Along the way, every time you encounter a multiple of 5 print
%    a line that looks like this (in the printed line below it
%    encountered the number 20.)\\
%\begin{Verbatim}
%fiver: 20
%\end{Verbatim}
%    You will need to use the modulo opertor({\tt \%}) to check for multiples of 5.
%\ifsolutions
%\textit{Solution:}\\
%\begin{codeexample}
%\begin{VerbatimOut}{\listingFile}
%for i in range(2,101,3):
%    if i % 5 == 0:
%        print "fiver:", i
%\end{VerbatimOut}
%\end{codeexample}
%\else
%\noindent\rule{5 in}{0.01 in}
%\fi


\labsection{Functions} A function is like a black box.  The user
passes some needed information into the box and the box uses that
information to perform some useful calculation and spits a result
out.  

\ul{Read chapter 4} in the Python manual to see how to
  build functions in Python. If something you read is confusing or
  unfamiliar to you, I suggest that you try to recreate the
  calculation yourself.  Take good notes so you can ask good questions
  during class.

Try the following problem to test your abilities
\begin{enumerate}

    \probtwo Build a function that contains a loop that sums the integers
    from 1 to $N$, where $N$ is an integer value that is passed into
    the function. By calling the function at least 5 times, verify
    that the formula
    \[
        \sum_{n=1}^N n = \frac{N(N+1)}{2}
    \]
    is correct.\\
\ifsolutions
\textit{Solution:}\\
\begin{codeexample}
\begin{VerbatimOut}{\listingFile}
def loopFunction(N):
    thesum = 0
    for i in range(1,N+1):
        thesum += i

    return thesum

N = 100
loopVal = loopFunction(N)
checkVal = N * (N + 1)/2.
print loopVal, checkVal
\end{VerbatimOut}
\end{codeexample}
\else
\noindent\rule{5 in}{0.01 in}
\fi
\end{enumerate}
\labsection{Plotting} \textbf{Computers can't plot
  functions}. Computers can only evaluate functions and plot data
points.  If you choose to evaluate a function on a very dense grid of
points, plot those points, and connect them with a line, the resulting
plot will resemble the function.  This way of
thinking will be increasingly important as the semester progresses.
\ul{Read chapter 7} in the Python manual to get a better
  idea of how to construct a plot in Python. If something you read is
  confusing or unfamiliar to you, I suggest that you try to recreate
  the calculation yourself.  Take good notes so you can ask good
  questions during class.

\begin{enumerate}
    \probtwo Plot the function 
    \[
        y(x) = e^{-0.25 x} \cos(x)
    \]

from $ x = 0$ to $x = 10$.\\
\ifsolutions
\textit{Solution:}\\
\begin{codeexample}
\begin{VerbatimOut}{\listingFile}
from numpy import arange, e, cos
from matplotlib import pyplot
x = arange(0,10,.1)
y = e**(-0.25 * x) * cos(x)
pyplot.plot(x,y)
pyplot.show()
\end{VerbatimOut}
\end{codeexample}
\else
\noindent\rule{5 in}{0.01 in}
\fi
\end{enumerate}
\labsection{Classes} You can think of a class as an object that
contains both data and functions.  Since you are already familiar with
variables and functions, the best way to learn about classes is to see
one and then ask questions. Here is a simple class used to
analyze ideal gas processes:
\begin{Verbatim}
class idealGas():
    def __init__(self,R=8.314,kB=1.38e-23):
        self.R = R
        self.kB = kB
    
    #Define what kind of ideal gas process we are dealing with
    def setProcessType(self,process, n,cV):
        self.n = n  #Set the number of moles of the gas
        self.process = process  #Indicate what kind of process it is:
                                # Isothermal, Isobaric, Adiabatic, Isochoric
        self.cV = cV
        self.cP = self.cV + self.R
        # If the process is adiabatic, we might need to know what gamma is.
        if process == 'Adiabatic':  
             self.gamma = self.cP/self.cV

    # Set the initial and final conditions for the process.
    def setConditions(self,pi=None,Vi=None,Ti=None,pf=None,Vf=None,Tf=None):
        self.pi = pi
        self.Vi = Vi
        self.Ti = Ti
        self.pf = pf
        self.Vf = Vf
        self.Tf = Tf

    def isothermal(self):
        from math import log
        self.work = -self.n * self.R * log(self.Vf/self.Vi)
        self.deltaE = 0
        self.heat = - self.work
        
    def isochoric(self):
        self.work = 0
        self.heat = self.n * self.cV * (self.Tf - self.Ti)
        self.deltaE = heat
        
    def isobaric(self):
        self.work = - self.pI * (self.Vf - self.Vi)
        self.heat = self.n * self.cP * (self.Tf - self.Ti)
        self.deltaE = self.n * self.cV * (self.Tf - self.Ti)
        
    def adiabatic(self):
        self.heat = 0
        self.deltaE = self.n * self.cV * (self.Tf - self.Ti)
        self.work = self.deltaE

    # Depending on what type of process it is, calculate the work
    # done, heat absorbed, and change in internal energy for the process.
    def calculate(self):

        
        # These three lines can replace the lines below and its a really
        # cool use of dictionaries. Try it.
        #         processes = {'Isothermal':self.isothermal, 'Adiabatic':
        #   self.adiabatic, 'Isochoric':self.isochoric, 'Isobaric':self.isobaric}
        # processes[self.process]()

         if self.process == 'Isothermal':
             self.isothermal()
         elif self.process == 'Isochoric':
             self.isochoric()
         elif self.process == 'Isobaric':
             self.isobaric()
         else:  #Adiabatic
             self.adiabatic()

# Class definition done. What follows below illustrates how to use the class.
gasOne = idealGas()  #Initialize the class.
gasOne.setProcessType('Isothermal',3,20.8)  # Set the type of process and number of moles
gasOne.setConditions(pi = 2.0e5, pf = 1.0e5,Ti = 200, Tf = 200, Vi = 2, Vf = 5 )  # Set initial and final conditions
gasOne.calculate()  # Calculate thermodynamic properties.
print(gasOne.work)  # Print results
print(gasOne.deltaE)

\end{Verbatim}
\labsection{Homework}

\begin{enumerate}
    \prob Build a function that takes argument $x$ and performs the sum
    \[
        \sum_{n=1}^{1000} n x^n
    \]
    using a {\tt for} loop.  By calling the function with different
    values of $x$, verify that the sum only converges for $|x| < 1$
    and that when it does converge, it converges to $x/(1-x)^2$.\\
\ifsolutions
\textit{Solution:}\\
\begin{codeexample}
\begin{VerbatimOut}{\listingFile}
def loopFunction(x):
    thesum = 0
    for n in range(1,1001):
        thesum += n * x**n 

    return thesum

x = 0.75
loopVal = loopFunction(x)
checkVal = x /(1 -x)**2
print loopVal, checkVal
\end{VerbatimOut}
\end{codeexample}
\else
\noindent\rule{5 in}{0.01 in}
\fi

%    \index{While loop} \index{Loops!while} 
%
%    \prob Redo the previous exercise using a {\tt while} loop instead
%    of a {\tt for} loop.  Make your own counter for $n$ by using $n=0$
%    outside the loop and $n=n+1$ inside the loop. Have the loop
%    execute until the current term in the sum, $n x^n$ has dropped
%    below $10^{-8}$. Verify that this way of doing it agrees with what
%    you found in the previous exercise.\\
%\ifsolutions
%\textit{Solution:}\\
%\begin{codeexample}
%\begin{VerbatimOut}{\listingFile}
%def loopFunction(x):
%    thesum = 0
%    n = 1
%    while n * x**n > 1e-9:
%        thesum += n * x**n 
%        n = n + 1
%    return thesum
%
%x = 0.75
%loopVal = loopFunction(x)
%checkVal = x /(1 -x)**2
%print loopVal, checkVal
%\end{VerbatimOut}
%\end{codeexample}
%\else
%\noindent\rule{5 in}{0.01 in}
%\fi

%\prob Verify by numerical experimentation with a {\tt while}
%    loop that
%    \[
%        \sum_{n=1}^{\infty} \frac{1}{n^2} = \frac{\pi^2}{6}
%    \]
%    Set the {\tt while} loop to quit when the next term added to
%    the sum is below $10^{-6}$.

\prob Verify, by numerically experimenting with a 
    loop that uses the {\tt break} command to
    exit the loop at the appropriate time, that the
    following infinite-product relation is true:
    \[
        \prod_{n=1}^{\infty} \left( 1 + \frac{1}{n^2} \right)
        = \frac{ \sinh{\pi} }{ \pi }
    \]\\
\ifsolutions
\textit{Solution:}\\
\begin{codeexample}
\begin{VerbatimOut}{\listingFile}
from numpy import sinh,pi

n = 1.
theproduct = 1.
while 1/n**2 > 1e-10:
    theproduct *= 1 + 1/n**2
    n += 1

print theproduct
print sinh(pi)/pi
\end{VerbatimOut}
\end{codeexample}
\else
\noindent\rule{5 in}{0.01 in}
\fi


\prob \index{Iteration} \index{Successive substitution}(\textbf{Transcendental
Equations})  A transcendental equation is one that cannot be solved
analytically.  Try solving the following equation for
$x$:
\[{\sin(x)\over x} = 1\]
to see what I mean.  One numerical method for solving this kind of
problem involves first rearranging the equation to look like this:
\[x = \sin(x)\] and then repeatedly evaluating the r.h.s using the
result of the previous evalution until subsequent evaluations differ
very little. \sidenote{This method is called successive evaluation}

Use a {\tt while} loop to verify that the following three iteration
    processes converge. Execute the loops
    until convergence at the $10^{-8}$ level is achieved.
    \[
        x_{n+1} = e^{-x_n}~~~~;~~~~
        x_{n+1} = \cos{x_n}~~~~;~~~~
        x_{n+1} = \sin{2 x_n}
    \]
    Note: iteration loops are easy to write. Just give $x$ an initial
    value (any number will do) and then inside the loop replace $x$ by
    the formula on the right-hand side of each of the equations
    above. To watch the process converge you will need to call the new
    value of $x$ something like {\tt xnew} so you can compare it to
    the previous $x$.

    Finally, try iteration again on this problem:
    \[
        x_{n+1} = \sin{3 x_n}
    \]
    Convince yourself that this process isn't converging to
    anything. We will see in Lab~\ref{Lab:19} what makes the
    difference between an iteration process that converges and
    one that doesn't.\\
\ifsolutions
\textit{Solution:}\\
\begin{codeexample}
\begin{VerbatimOut}{\listingFile}
from numpy import arange,sin,log,exp,cos

x = 100   # Set initial value
xNew = sin(3 * x)  # Find next value

while abs(x - xNew) > 1e-8:
    x = xNew  # Update old value to be what you just found
    xNew = sin(3 * x)  # Find the next value
    print xNew

\end{VerbatimOut}
\end{codeexample}
\else
\noindent\rule{5 in}{0.01 in}
\fi


\prob Write a class to calculate the trajectory of a projectile.  You should
have at least three member functions to: i) set the launch conditions (initial
velocity, position, etc.  ii) calculate the landing location and iii)
plot the trajectory.
\end{enumerate}

%\mainmatter

%\pagestyle{fancy}
%\renewcommand{\chaptermark}[1]{\markboth{Computational Physics 385}{\chaptername \ \thechapter \ \ #1}}
