\chapter{Euler's Method}
\label{Lab:Euler's Method}

\index{ODE} \index{Initial Value Problems} Today we get our first
taste of what it means to solve a differential equation numerically.
In fact, most of our time for the remainder of the semester will be
some form of this task: finding a numerical solution to all kinds of
differential equations.  We begin with an ordinary differential
equation and we'll use a familiar (albeit inaccurate) technique to
solve it: Euler's method.  You should have seen this method in PH150
and possibly in other classes.

\labsection{Solving Differential Equations Numerically}

Last time we mastered the art of taking derivatives on a grid.  Now,
let's see how we can use that knowledge to numerically solve
differential equations. Consider the motion of a projectile near the
surface of the earth with no air resistance. The differential
equations that describe the projectile are
\begin{equation}\label{eq:SimpleProjectile}
    \begin{aligned}
    \frac{d^2 x}{dt^2} &= 0
    & \qquad
    \frac{d^2 y}{dt^2} &= -g
    \end{aligned}
\end{equation}
along with some initial conditions, $x(0)$, $y(0)$, $v_x(0)$, and
$v_y(0)$. \sidenote{Remember, since they are second-order differential
  equations, I must have two extra pieces of information to solve
  them.} These are two, second-order, ordinary differential
equations. \sidenote{It's really important that you are able to
  identify the type of differential equation that you are solving.
  The numerical method that is most appropriate depends on it.}  Of
course, we can always write a second-order differential equation as a
set of two, coupled, first-order differential equations.  Let's do
that now:


\begin{equation}\label{eq:SimpleProjectile}
    \begin{aligned}
    \frac{d x}{dt} &= v_x
    & \qquad
    \frac{d y}{dt} &= v_y \\
    \frac{d v_x}{dt} &= 0
    & \qquad
    \frac{d v_y}{dt} &= -g
    \end{aligned}
\end{equation}
This set of equations is easily solved analytically, but
imagine that we didn't have an analytic solution.  How could we
numerically model the motion of the projectile?

The basic idea in a numerical solution for a system like this is to
think of time as being a discrete grid rather than a continuous
quantity.  It is easiest to have an \ul{evenly spaced time grid} $[t_0,
t_1, t_2, ...]$ with $t_0=0$, $t_1=\tau$, $t_2=2\tau$, etc. We
label the variables on this time grid using the notation $x_0
\equiv x(0)$, $x_1 \equiv x(\tau)$, $x_2 \equiv x(2\tau)$, etc.
To formulate a numerical solution to these equations, we first must
write the derivatives using the (inaccurate) forward
difference approximation of the derivative that you learned about
in the reading:
\begin{equation}\label{eq:Eul1}
    \begin{aligned}
    \frac{x_{n+1} - x_n}{\tau} &= v_{x,n}
    & \qquad
    \frac{y_{n+1} - y_n}{\tau} &= v_{y,n} \\
    \frac{v_{x,n+1} - v_{x,n}}{\tau} &= 0
    & \qquad
    \frac{v_{y,n+1} - v_{y,n}}{\tau}  &= - g
    \end{aligned}
\end{equation}
Notice that the left sides of these equations are centered on the time
$t_{n+1/2}$, but the right sides are centered at time $t_n$. This makes this
approach inaccurate\sidenote{Fixing this problem will be our topic for
next time.}, but if we make $\tau$ small enough it can work well
enough to see the principles involved.

By solving the equations in \eqref{eq:Eul1} we can obtain a simple
algorithm for stepping our solution forward in time:
\begin{equation}\label{eq:Eul2}
    \begin{aligned}
    x_{n+1} &= x_n + v_{x,n} \tau
    & \qquad
    y_{n+1} &= y_n + v_{y,n} \tau \\
    v_{x,n+1} &= v_{x,n}
    & \qquad
    v_{y,n+1} &= v_{y,n}  - g \tau
    \end{aligned}
\end{equation}
\reminder{\lefthand}{The name Euler does not rhyme with ``cooler''; it
  rhymes with ``boiler''. You will impress your fellow students and
  your professors if you give this important name from the history of
  mathematics its proper pronunciation.}This method of approximating
solutions is called Euler's method. In general, it's not very good,
especially over many time steps. However, it provides a foundation for
learning other better methods.

\begin{enumerate}
\probtwo Make a program in Python to model the motion of a ball bouncing on
    the floor using Euler's method.  In your script, define the initial
    position of the ball with \texttt{x=0} and \texttt{y=1}, and the
    initial velocity with \texttt{vx=1} and \texttt{vy=0}. Then write a
    while loop to step the position and velocity forward in time using
    Eq.~\eqref{eq:Eul2}. Have your while
    loop exit when the ball has traveled $10$ meters horizontally.  

\begin{enumerate}
    \item To simulate bouncing, put an \texttt{if}
        statement in your loop that checks if \texttt{y} is
        less than zero. When it is, make \texttt{vy}
        positive like this
\begin{Verbatim}
vy=abs(vy)
\end{Verbatim}
    Make a movie by plotting the position of the ball as a dot
    each time the loop iterates, like this:
\begin{Verbatim}
pyplot.scatter(x[-1],y[-1])
pyplot.xlim(0,10)
pyplot.ylim(0,1.2)
pyplot.draw()
pyplot.pause(0.01)\end{Verbatim}

\item Our bouncing condition in part (a) is lousy.  Make it
    better by adding some more logic that does the following:

    (i) Test to see if \texttt{y} will go less than zero on
    this time step, but don't actually change \texttt{y} yet.

    (ii) If \texttt{y} won't go less than zero this step,
    just do a regular Euler step.

    (iii) If it will go negative this time step, determine a smaller time
    step $\tau_1$ such that an Euler step will take the ball to $y=0$.
    Then take an Euler step with $\tau_1$. After taking this small step,
    make the $y$-velocity positive as before
\begin{Verbatim}
vy=abs(vy)
\end{Verbatim}
    and then take an Euler step of $\tau_2 = \tau-\tau_1$
    to finish off the time interval.

    Play with different values of $\tau$ and notice the behaviour of
    the simulation.

    \item Make your model look more realistic by adding some loss to the
        code that represents the bounce process, like this
\begin{Verbatim}
vy=0.95*abs(vy)
\end{Verbatim}
    This damping will mask the growth of Euler's method for a
    suitably small $\tau$.

\ifsolutions
\textit{Solution:}\\
\begin{codeexample}
\begin{VerbatimOut}{\listingFile}
from matplotlib import pyplot

class bouncing():

    def __init__(self,v,r):
        
        self.vx = v[0]
        self.vy = v[1]
        self.x = r[0]
        self.y = r[1]
        self.tau = 0.001
        self.g = 9.8



    def EulerBetterBounce(self):

        counter = 1
        while self.x < 10:
            self.x = self.x + self.vx * self.tau
            if self.y + self.vy * self.tau < 0:
                tempTau = -self.y/self.vy
                finishTau = self.tau - tempTau
                self.y = self.y + self.vy * tempTau
                self.vy = 0.95 *abs(self.vy)
                self.y = self.y + self.vy * finishTau
            else:
                self.y = self.y + self.vy * self.tau
                self.vy = self.vy - self.g * self.tau
            if counter % 10 == 0:
                pyplot.plot(self.x,self.y,'k.')
                pyplot.xlim(0,10)
                pyplot.ylim(0,1.2)
                pyplot.draw()
                pyplot.pause(0.01)
            counter += 1
            

    def Euler(self):

        counter = 1
        while self.x < 10:
            self.x = self.x + self.vx * self.tau
            self.y = self.y + self.vy * self.tau
            self.vy = self.vy - self.g * self.tau
            if self.y < 0:
                self.vy = abs(self.vy)
            if counter % 10 == 0:
                pyplot.plot(self.x,self.y,'k.')
                pyplot.xlim(0,10)
                pyplot.ylim(0,1.2)
                pyplot.draw()
                pyplot.pause(0.01)
            counter += 1

bouncer = bouncing([1,0],[0,1])
bouncer.EulerBetterBounce()
\end{VerbatimOut}
\end{codeexample}
\else
\noindent\rule{4 in}{0.01 in}
\fi

\end{enumerate}
\end{enumerate}


You should have noticed that no matter what we tried, Euler's method is
always unstable (i.e.  the amplitude of the bounce continues to grow
with each subsequent bounce).  This is an inadequacy of Euler's method
and must be addressed at some point. For now, you should remember that
\ul{Euler's method does not conserve energy}.  Stay tuned for a method
that does conserve energy.


\begin{enumerate}
  \probtwo \label{prob:battedBall} You may have heard that it's easier to hit a homerun in
  Denver than in Atlanta.  Let's use Euler's method to simulate the
  motion of a batted baseball and investigate the claim.
\begin{enumerate}
 \item We'll need to include the effect of air drag for this situation
   so let's think about this force for a second.  The force of drag is given by:
\begin{equation}
\mathbf{F} = {1 \over 2} \rho A C v^2 {\mathbf{v} \over |\mathbf{v}|}
\end{equation}
The first thing we need to do is decompose this force into it's
components.  Look at figure \ref{fig:Drag} to convince yourself that
the equations below are the correct components.
\begin{equation}
F_x = {1\over 2} \rho A C v v_x
\end{equation}
\begin{equation}
F_y = {1\over 2} \rho A C v v_y
\end{equation} \\

\marginfig[-1in]{Figures/Drag}{\label{fig:Drag}Illustration of the
  decomposition of the drag force.}
\item Add the force of air drag into your Euler's model and plot the
  trajectory of a baseball($C= 0.5$) batted at an angle of $35^\circ$
  with an initial speed of $49 $ m/s ($110$ mph). Use google to lookup
  the dimensions and mass of a baseball and choose the
  air density corresponding to sea level ($\rho = 1.29$). Play with
  the air density and cross-sectional area to convince yourself that
  you did it correctly.
\item Now make two plots\sidenote{This should be easy if you have
    designed your class structure well}: one trajectory for the ball
  batted in Denver ($\rho = 0.96$) and another for the ball batted in
  Atlanta ($ \rho = 1.22$).  Determine the difference in ranges for
  these two batted balls.  Would you rather bat in Denver or Atlanta?
\item It turns out that the drag coefficient (C) of a baseball varies
  with the speed of the ball.  At low speed $C \approx 0.5$ but at
  higher speed as the air flow becomes turbulent, the drag coefficient
  decreases.  The following formula is a good approximation \footcite{giordano1997computational}
\begin{equation}
C(v) = .198 + \frac{.295}{1 + e^{\frac{v - 35}{5}}}
\end{equation}
Incorporate this change into your code and determine how it affects
the difference in ranges between Atlanta and Denver.
 \end{enumerate}
\ifsolutions
\textit{Solution:}\\
\begin{codeexample}
\begin{VerbatimOut}{\listingFile}
from numpy import sinh,arange,pi,cos,sin
from matplotlib import pyplot



class projectile():

    def __init__(self, g = 9.8, dt = 0.1):
        self.g = g
        self.dt = dt

        # For part d
    def C(self,v):
        return .198 + .295/(1 + exp((v - 35)/5) )

    def setDragSettings(self,A,C,m,rho,variableC = False):
        # Most of the logic here is for part d
        if not variableC:
            self.variableC = False
            self.dragFactor = 1/2. * A * C * rho/m  #Only include this line for previous parts
        else:
            self.variableC = True
            self.constant = 1/2. * A * rho/m

        
    def setInitialConditions(self,r,v):
        self.initialPos = r
        self.initialV = v

    def Euler(self):
        from numpy.linalg import norm
        
        self.x = [self.initialPos[0]]
        self.y = [self.initialPos[1]]
        vx = self.initialV[0]
        vy = self.initialV[1]
        
        while self.y[-1] > 0:

            speed = norm([vx,vy])
            if self.variableC:
                self.dragFactor = self.constant * self.C(speed)
            dragX = self.dragFactor * speed * vx
            dragY = self.dragFactor * speed * vy
            self.y.append(self.y[-1] + vy * self.dt)
            self.x.append(self.x[-1] + vx * self.dt)
            vy = vy - (dragY + self.g) * self.dt
            vx = vx - dragX * self.dt

    def plotTrajectory(self):
        from matplotlib import pyplot
        pyplot.plot(self.x,self.y)
        # pyplot.show()


from numpy import exp
speed = 49
angle = 35 * pi/180.
diameter = 0.075
radius = diameter /2
area = pi * radius**2
mass = .145
C = 0.5
rhoDenver = 0.96
rhoAtlanta = 1.22

trajectoryOne = projectile(dt=0.01)
trajectoryOne.setInitialConditions([0,1],[speed * cos(angle),speed * sin(angle)])
trajectoryOne.setDragSettings(area,C,mass,rhoDenver,variableC = False)
trajectoryOne.Euler()
trajectoryOne.plotTrajectory()

trajectoryTwo = projectile(dt=0.01)
trajectoryTwo.setInitialConditions([0,1],[speed * cos(angle),speed * sin(angle)])
trajectoryTwo.setDragSettings(area,C,mass,rhoAtlanta,variableC = False)
trajectoryTwo.Euler()
trajectoryTwo.plotTrajectory()
print abs(trajectoryTwo.x[-1] - trajectoryOne.x[-1]) * 3.28 # conversion from feet to meters
pyplot.show()
\end{VerbatimOut}
\end{codeexample}
\else
\noindent\rule{4 in}{0.01 in}
\fi



\end{enumerate}
\labsection{Homework}
\begin{enumerate}
\prob Let's use Euler's method to model the motion of a cannon shell fired
to a very high altitude.  
\begin{enumerate}
\item Use your code from the previous problem as a starting point for
  this problem. \sidenote{You'll want to make a copy so there are two
    seperate files.}  Choose a launch angle $0^\circ < \theta <
  90^\circ$ and set the launch speed to be $750$ m/s. Choose $C= 0.5$,
  $A = 0.007$ m$^2$, $\rho = 1.29$ kg/m$^3$, and $m = 50$ kg. Plot the trajectory.
\item  In part (a) you should have found that the maximum altitude of
  the cannon shell is on the order of kilometers.  At these altitudes,
  the density of the air changes appreciably.  We can model the change
  in air density with the following function:
\begin{equation}
\rho = \rho_0 \left( 1 - {a y\over T_0} \right)^\alpha
\end{equation} 
with $a \approx \num{6.5e-3}$ K/m, $T_0$ is the sea-level temperature
and the exponent $\alpha \approx 2.5$.  Plot this function and verify
that it behaves as you would expect.
\item Now incorporate the changing air density into your Euler's model
  and plot the resulting trajectory.
\item Plot the constant-air density trajectory on top of the
  trajectory that accounts for the variation to see how big of an
  affect it is. (Hint: Since you coded this using a class, this should
  be no big deal.  Just initiate a second instance of this class and
  turn off the density correction function.)

\end{enumerate}

\ifsolutions
\textit{Solution:}\\
\begin{codeexample}
\begin{VerbatimOut}{\listingFile}
from numpy import sinh,arange,pi,cos,sin
from matplotlib import pyplot



class projectile():

    def __init__(self, g = 9.8, dt = 0.1):
        self.g = g
        self.dt = dt


    def setDragSettings(self,A,C,m,rho,variablerho = False,rho0 = 1.29, a = 6.5e-3,T0 = 320,alpha = 2.5):
        # Most of the logic here is for part d
        if not variablerho:
            self.variablerho = False
            self.dragFactor = 1/2. * A * C * rho/m  #Only include this line for previous parts
        else:
            self.variablerho = True
            self.rho0 = rho0
            self.a = a
            self.T0 = T0
            self.alpha = alpha
            self.constant = 1/2. * A * C/m


    def rho(self,y):
        return self.rho0 * ( 1 - self.a * y /self.T0)**self.alpha
    
    def setInitialConditions(self,r,v):
        self.initialPos = r
        self.initialV = v

    def Euler(self):
        from numpy.linalg import norm
        
        self.x = [self.initialPos[0]]
        self.y = [self.initialPos[1]]
        vx = self.initialV[0]
        vy = self.initialV[1]
        
        while self.y[-1] > 0:

            speed = norm([vx,vy])
            if self.variablerho:
                self.dragFactor = self.constant * self.rho(self.y[-1])
            dragX = self.dragFactor * speed * vx
            dragY = self.dragFactor * speed * vy
            self.y.append(self.y[-1] + vy * self.dt)
            self.x.append(self.x[-1] + vx * self.dt)
            vy = vy - (dragY + self.g) * self.dt
            vx = vx - dragX * self.dt

    def plotTrajectory(self):
        from matplotlib import pyplot
        pyplot.plot(self.x,self.y)
        # pyplot.show()


from numpy import exp
speed = 700
angle = 45 * pi/180.
area = 0.007
mass = 50
C = 0.5
rho = 1.29

# Without density correction
trajectoryOne = projectile(dt=0.01)
trajectoryOne.setInitialConditions([0,1],[speed * cos(angle),speed * sin(angle)])
trajectoryOne.setDragSettings(area,C,mass,rho,variablerho = False)
trajectoryOne.Euler()
trajectoryOne.plotTrajectory()

# With density correction.
trajectoryTwo = projectile(dt=0.01)
trajectoryTwo.setInitialConditions([0,1],[speed * cos(angle),speed * sin(angle)])
trajectoryTwo.setDragSettings(area,C,mass,rho,variablerho = True)
trajectoryTwo.Euler()
trajectoryTwo.plotTrajectory()
print abs(trajectoryTwo.x[-1] - trajectoryOne.x[-1]) 
pyplot.show()

\end{VerbatimOut}
\end{codeexample}
\else
\noindent\rule{4 in}{0.01 in}
\fi

\end{enumerate}