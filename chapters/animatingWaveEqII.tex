\chapter{The driven, damped wave equation}
\label{Lab:14}
Today we will continue solving the wave equation but we will add more
physics to the equation, making the simulation more realistic.  This
will come at the cost of algorithmic complexity.  Let's start by
adding some damping to the wave equation using a linear damping term, like this:
\begin{equation}\label{eq:WaveDamped}
    \frac{\partial^2 y}{\partial t^2}
    + \gamma \frac{\partial y}{\partial t}
    - c^2 \frac{\partial^2 y}{\partial x^2} = 0
\end{equation}

 with $c$ constant.  The staggered leapfrog method can be used to
solve Eq.~\eqref{eq:WaveDamped} also.  To do this, we use the
approximate first derivative formula
\begin{equation}
    \frac{\partial y}{\partial t} \approx \frac{y_j^{n+1} -
    y_j^{n-1}}{2 \tau}
\end{equation}


along with the second derivative formulas in
Eqs.~\eqref{eq:TimeDeriv} and \eqref{eq:SpaceDeriv} and find an
expression for the values one step in the future:
\begin{equation}\label{eq:stlfDamped}
    y_j^{n+1} = \frac{1}{2+\gamma \tau} \left (
    4 y_j^n - 2y_j^{n-1} + \gamma \tau y_j^{n-1}
    + \frac{2 c^2 \tau^2}{h^2} \left( y_{j+1}^n - 2 y_j^n + y_{j-1}^n
    \right) \right)
\end{equation}


\begin{enumerate}
\probtwo \label{P:14.1}
\begin{enumerate}

\subprob Derive Eq.~\eqref{eq:stlfDamped}.

\subprob Find a new formula for {\tt yold} using
Eqs.~\eqref{eq:leapfrogIV} and \eqref{eq:stlfDamped}, similar to what
we did to arrive at Eq.~\eqref{eq:yold}.\\
\marginfig[.2in]{Figures/f05p4}{The maximum amplitude
      of oscillation decays exponentially for the damped wave
      equation. (Problem~\ref{P:14.1d}) } Your answer should look like
    this:
\begin{equation}
  y_j^{-1} = -v_0(x_j) \frac{2 \tau(2+\gamma
    \tau) }{4 } + \left ( y_j^0 + \frac{c^2 \tau^2}{2h^2} \left( y_{j+1}^0 - 2 y_j^0 +
      y_{j-1}^0
    \right) \right) \\
\end{equation}

\ifsolutions
\textit{Solution:}
We will plug in this equation:
\begin{equation}
y_j^1 = \frac{1}{2+\gamma \tau} \left (
    4 y_j^0 - 2y_j^{-1} + \gamma \tau y_j^{-1}
    + \frac{2 c^2 \tau^2}{h^2} \left( y_{j+1}^0 - 2 y_j^0 + y_{j-1}^0
    \right) \right)
\end{equation}
into this one:
\begin{align}
    {y_j^1-y_j^{-1} \over 2 \tau} &= v_0(x_j)\\
\frac{1}{2 \tau(2+\gamma \tau)} \left (
    4 y_j^0 - 2y_j^{-1} + \gamma \tau y_j^{-1}
    + \frac{2 c^2 \tau^2}{h^2} \left( y_{j+1}^0 - 2 y_j^0 + y_{j-1}^0
    \right) \right) - {y_j^{-1} \over 2 \tau} = v_0(x_j)\\
\end{align}
collecting $y_j^{-1}$ terms:
\begin{align}
  \frac{-2 y_j^{-1} + \gamma \tau y_j^{-1}}{2 \tau(2+\gamma \tau)} -
  {y_j^{-1} \over 2 \tau} = v_0(x_j) - \frac{1}{2 \tau(2+\gamma \tau)}
  \left ( 4 y_j^0 + \frac{2 c^2 \tau^2}{h^2} \left( y_{j+1}^0 - 2
      y_j^0 + y_{j-1}^0
    \right) \right) \\
  \frac{-2 y_j^{-1} + \gamma \tau y_j^{-1} - (2 + \gamma \tau)
    y_j^{-1}}{2 \tau(2+\gamma \tau)} = v_0(x_j) - \frac{1}{2
    \tau(2+\gamma \tau)} \left ( 4 y_j^0 + \frac{2 c^2 \tau^2}{h^2}
    \left( y_{j+1}^0 - 2 y_j^0 + y_{j-1}^0
    \right) \right) \\
  y_j^{-1}\frac{-2 + \gamma \tau - (2 + \gamma \tau) }{2 \tau(2+\gamma
    \tau)} = v_0(x_j) - \frac{1}{2 \tau(2+\gamma \tau)} \left ( 4
    y_j^0 + \frac{2 c^2 \tau^2}{h^2} \left( y_{j+1}^0 - 2 y_j^0 +
      y_{j-1}^0
    \right) \right) \\
  y_j^{-1} = -v_0(x_j) \frac{2 \tau(2+\gamma
    \tau) }{4 } + \left ( y_j^0 + \frac{c^2 \tau^2}{2h^2} \left( y_{j+1}^0 - 2 y_j^0 +
      y_{j-1}^0
    \right) \right) \\
\end{align}

\fi
    \subprob \label{P:14.1b} Modify your staggered leapfrog code to
    include damping with $\gamma=0.2$. Then run your animation with
    the initial conditions in Problem~\ref{P:13.3c} and verify that
    the waves damp away.  You will need to run for about 25~s and you
    will want to use a big skip factor so that you don't have to wait
    forever for the run to finish.  

    \subprob \label{P:14.1d}Include some code to record the maximum value of $y(x)$
    over the entire grid as a function of time and then plot it as a
    function of time at the end of the run so that you can see the
    decay caused by $\gamma$. 

    \subprob The decay of a simple harmonic oscillator is exponential,
    with amplitude proportional to $e^{-\gamma t/2}$. Scale this time
    decay function properly and lay it over your maximum $y$ plot to
    see if it fits. Can you explain why the fit is as good as it is?
    (Hint: think about doing this problem via separation of
    variables.)  


\end{enumerate}
\ifsolutions
\textit{Solution:}\\
\begin{codeexample}
\begin{VerbatimOut}{\listingFile}

class animatedWave():

    def __init__(self,a,b,N,tau,tMax,gamma,stabilityCheck = False, c = 2):
        from numpy import arange
        self.L = b
        self.N = N
        self.tMax = tMax
        self.tau = tau
        self.gamma = gamma
        self.dx = (b - a)/N
        self.x = arange(a - self.dx/2., b + self.dx,self.dx)

        self.c = c

        print stabilityCheck, 'here'
        from numpy import any
        if any(self.tau > self.dx/self.c) and stabilityCheck:
            print 'Beware: your algorithm will be unstable.'
            import sys
            sys.exit()


    def initializeWave(self):
        from numpy import exp,zeros
        # Gaussian initial displacement
        self.v = 2 * exp( -160/self.L**2 * (self.x - self.L/2.)**2 )# - exp( -160/self.L**2 * (0 - self.L/2.)**2 )
        # Zero initial velocity
        self.y = zeros(self.N+2)


    def animate(self):
        from numpy import zeros_like,copy
        from matplotlib import pyplot
        constant = self.c**2 * self.tau**2/(2 * self.dx**2)
        constantTwo = 2. * self.tau * (2 + self.gamma * self.tau) 
        
        yOld = zeros_like(self.y)
        yOld[1:self.N + 1] = -self.v[1:self.N + 1] * constantTwo/4. +( self.y[1:self.N + 1] + constant *(self.y[2:self.N + 2]- 2 * self.y[1:self.N + 1] + self.y[0:self.N])) 
        yOld[0] = -yOld[1]
        yOld[-1] = -yOld[-2]
        t = 0
        counter = 0
        self.maxes = []
        while t < self.tMax:
            yNew = zeros_like(self.y)
            yNew[1:self.N + 1] = 1/(2 + self.gamma * self.tau) * (4 * self.y[1:self.N + 1] - 2 * yOld[1:self.N + 1] + self.gamma * self.tau * yOld[1:self.N + 1] + 4 * constant * (self.y[2:self.N + 2] - 2 * self.y[1: self.N + 1] + self.y[0:self.N]))
            yNew[0] =  -yNew[1]
            yNew[-1] = - yNew[-2]
            self.maxes.append(max(yNew))
            if counter % 50 == 0:
                pyplot.plot(self.x,self.y,'r.-')
                pyplot.ylim(-0.1,0.1)
                pyplot.draw()
                pyplot.pause(.000001)
                pyplot.clf()


            yOld = copy(self.y)
            self.y = copy(yNew)


            t += self.tau
            counter += 1
    def plotMaxes(self):
        from matplotlib import pyplot
        from numpy import linspace,exp
        t = linspace(0,self.tMax,len(self.maxes))
        pyplot.figure(2)
        pyplot.plot(t,self.maxes,'r.-')
        pyplot.plot(t,0.035 * exp(-self.gamma * t/2),'b-')
        pyplot.show()
a = 0
b = 1.
c = 2.
N = 200
tau = 0.001
gamma = 0.2
tMax = 25


myWave = animatedWave(a,b,N,tau,tMax,gamma,stabilityCheck = True)
myWave.initializeWave()
myWave.animate()
myWave.plotMaxes()
\end{VerbatimOut}
\end{codeexample}
\else
\noindent\rule{5 in}{0.01 in}
\fi

\end{enumerate}

\labsection{The damped and driven wave equation}

Finally, let's look at what happens when we add an oscillating
driving force to our string, so that the wave equation becomes
\begin{equation}\label{eq:WaveDampedDriven}
    \frac{\partial^2 y}{\partial t^2} + \gamma \frac{\partial
    y}{\partial t} - c^2 \frac{\partial^2 y}{\partial x^2} =
    \frac{f(x)}{\mu} \cos(\omega t)
\end{equation}
At the beginning of Lab~\ref{Lab:11} we discussed the
qualitative behavior of this system.  Recall that if we have a
string initially at rest and then we start to push and pull on
a string with an oscillating force/length of $f(x)$, we launch
waves down the string.  These waves reflect back and forth on
the string as the driving force continues to launch more waves.
The string motion is messy at first, but the damping in the
system causes the the transient waves from the initial launch
and subsequent reflections to eventually die away.  In the end,
we are left with a steady-state oscillation of the string at
the driving frequency $\omega$.

\marginfig[-2in]{Figures/f05p5}{Snapshots of the evolution a driven and
damped wave with $\omega=400$.  As the transient behavior dies
out, the oscillation goes to the resonant mode.  To make the
pictures more interesting, the string was not started from rest
in these plots. (In Problem~\ref{P:14.2} you start from rest for
easier coding.)}


\begin{enumerate}
  \probtwo \label{P:14.2} Model the motion of a damped, driven guitar
  string, clamped at both ends.  Since we are driving the oscillation with
  an external force, we can start with an undisplaced string with zero
  velocity.  This simplifies the initial conditions (just set $y=0$
  and $y_\mathrm{old}=0$ and enter the time-stepping loop). Use the
  following values for the string parameters: $T = 127$ N , $\mu =
  0.003$ kg/m, and $L = 1.2$ m and remember that $c=\sqrt{T/
    \mu}$. Use the same driving force as in Problem~\ref{P:11.1a}
\begin{equation}
\begin{array}{c}
 \\
f(x) \\
 \\
\end{array}
=
\left\{
\begin{array}{ll}
0.73 & {\rm if~~~} 0.8 \le x \le 1 \\
 & \\
0 & {\rm otherwise} \\
\end{array}
\right.
\end{equation}
and set the driving frequency at $\omega=400$.  
\begin{enumerate}
\subprob Derive an equation similar to Eq.~\eqref{eq:stlfDamped} but
that includes damping and driving as in
Eq.~\eqref{eq:WaveDampedDriven}.  Your answer should look like this:
\begin{multline}
    y_j^{n+1} = \frac{1}{2+\gamma \tau} \left (
    4 y_j^n - 2y_j^{n-1} + \gamma \tau y_j^{n-1}
    + \frac{2 c^2 \tau^2}{h^2} \left( y_{j+1}^n - 2 y_j^n + y_{j-1}^n
    \right) \right)\\ + \frac{f(x_j)}{\mu} \cos(\omega t) 
  \frac{2\tau^2}{(2 + \gamma \tau)}
\end{multline}

\subprob Modify your code from Problem~\ref{P:14.1} to use this new
algorithm.

\subprob Choose a damping constant $\gamma$ that is the proper size to
make the system settle down to steady state after 20 or 30 bounces of
the string. (You will have to think about the value of $\omega$ that
you are using and about your damping rate result from
problem~\ref{P:14.1} to decide which value of $\gamma$ to use to make
this happen.)

\subprob Run the model long enough that you can see the transients die
away and the string settle into the steady oscillation at the
driving frequency. You may find yourself looking at a flat-line
plot with no oscillation at all. If this happens look at the
vertical scale of your plot and remember that we are doing real
physics here. If your vertical scale goes from $-1$ to $1$, you
are expecting an oscillation amplitude of 1 meter on your
guitar string. Compare the steady state mode to the shape found
in Problem~\ref{P:11.1a} (see Fig.~\ref{Figures/f03p1a}).

Then run again with $\omega=1080$, which is close to a
resonance (see Fig.~\ref{Figures/f03p1b}), and again see the system
come into steady oscillation at the driving frequency.
\end{enumerate}
\end{enumerate}
\ifsolutions
\textit{Solution:}\\
\begin{codeexample}
\begin{VerbatimOut}{\listingFile}


class animatedWave():

    def __init__(self,a,b,N,tau,mu,T,omegaD,tMax,gamma,stabilityCheck = False, c = 2):
        from numpy import arange,sqrt
        self.L = b
        self.N = N
        self.tMax = tMax
        self.tau = tau
        self.mu = mu
        self.gamma = gamma
        self.T = T
        self.omegaD = omegaD
        self.dx = (b - a)/N
        self.x = arange(a - self.dx/2., b + self.dx,self.dx)

        self.c = sqrt(T/mu)

        if self.tau > self.dx/self.c and stabilityCheck:
            print 'Beware: your algorithm will be unstable.'
            import sys
            sys.exit()

    def f(self,x):
        import numpy
        if type(x) is numpy.ndarray:
            from numpy import array
            return array([self.f(var) for var in x])
        else:
            if 0.8 <= x <= 1.0:
                return 0.73
            else:
                return 0

    def initializeWave(self):
        from numpy import exp,zeros
        # Gaussian initial displacement
        self.v = 2 * exp( -160/self.L**2 * (self.x - self.L/2.)**2 ) - exp( -160/self.L**2 * (0 - self.L/2.)**2 )
        # Zero initial velocity
        self.y = zeros(self.N+2)


    def animate(self):
        from numpy import zeros_like,copy,cos
        from matplotlib import pyplot
        constant = self.c**2 * self.tau**2/(2 * self.dx**2)
        
        constantOne = 2. + self.gamma * self.tau
        constantTwo = 2. * self.c**2 * self.tau**2/self.dx**2

        yOld = zeros_like(self.y)
        yOld[1:self.N + 1] = -self.v[1:self.N + 1]* self.tau + 1./constantOne * (4 * self.y[1:self.N + 1] + constantTwo * (self.y[2:self.N + 2] - 2 * self.y[1:self.N + 1] + self.y[0:self.N])) + self.f(self.x[1:self.N + 1])/self.mu  * 2 * self.tau**2/constantOne
        yOld[0] = -yOld[1]
        yOld[-1] = -yOld[-2]
        t = 0
        counter = 0
        while t < self.tMax:
            yNew = zeros_like(self.y)
            yNew[1:self.N + 1] = 1/(2 + self.gamma * self.tau) * (4 * self.y[1:self.N + 1] - 2 * yOld[1:self.N + 1] + self.gamma * self.tau * yOld[1:self.N + 1]+ 4 * constant * (self.y[2:self.N + 2] - 2 * self.y[1: self.N + 1] + self.y[0:self.N])) + self.f(self.x[1:self.N + 1])/self.mu * cos(self.omegaD * t) * 2 * self.tau**2/(2 + self.gamma * self.tau)
            yNew[0] =  -yNew[1]
            yNew[-1] = - yNew[-2]
            if counter % 50 == 0:
                pyplot.plot(self.x,self.y,'r.-')
                pyplot.ylim(-0.005,0.005)
                pyplot.draw()
                pyplot.pause(.000001)
                pyplot.clf()


            yOld = copy(self.y)
            self.y = copy(yNew)


            t += self.tau
            counter += 1
a = 0
b = 1.2
N = 200
tau = 0.00001
gamma = 60
tMax = 25.
mu = .003
T = 127.
omegaD = 1080.


myWave = animatedWave(a,b,N,tau,mu,T,omegaD,tMax,gamma,stabilityCheck = True)
myWave.initializeWave()
myWave.animate()

\end{VerbatimOut}
\end{codeexample}
\else
\noindent\rule{5 in}{0.01 in}
\fi

\labsection{Homework}

\begin{enumerate}
  \prob Use the staggered leapfrog algorithm to study how pulses
  propagate along a hanging chain. 
The wave equation for transverse waves on a chain with a varying
tension ($T(x)$), constant linear mass density ($\mu$), including
damping and an external driving force is given by:
\begin{equation}\label{eq:WaveDampedDrivenVariableTension}
    \frac{\partial^2 y}{\partial t^2} + \gamma \frac{\partial
    y}{\partial t} - \frac{1}{\mu}\frac{\partial }{\partial x}\left(T(x)
    \frac{\partial y }{\partial x} \right) =
    \frac{f(x)}{\mu} \cos(\omega t)
\end{equation}
Here are some steps that you can follow to help you accomplish this:
\begin{enumerate}
\subprob \label{P:14.3}Recall that $T(x) = \mu g x$ for a hanging chain.  Insert
this into equation \eqref{eq:WaveDampedDrivenVariableTension} and
evaluate the third term on the left.\\
\ifsolutions
\textit{Solution:}
\begin{align}
\frac{\partial ^2y}{\partial t^2} - g \frac{\partial}{\partial
  x} \left( x \frac{\partial y}{\partial x}\right) + \gamma
\frac{\partial y}{\partial t} &= \frac{f(x)}{\mu} \cos(\omega t)\\
\frac{\partial ^2y}{\partial t^2} - g \left( \frac{\partial
    y}{\partial x} + x \frac{\partial^2 y}{\partial x^2}\right) + \gamma
\frac{\partial y}{\partial t} &= \frac{f(x)}{\mu} \cos(\omega t)\\
\end{align}
\fi
\subprob Now discretize the equation and rearrange to solve for
$y_j^{n+1}$ just as we did in equation \eqref{leapfrog} (Don't lose
track of your superscripts and subsripts).  Your answer should look
like this:
\begin{multline}
  y_j^{n+1} = \frac{2\tau^2}{\left( 2 + \gamma \tau\right)} \frac{f(x_j)}{\mu}
  \cos (\omega t_j) + \frac{2\tau^2 g }{\left( 2 + \gamma
      \tau\right)} \left( \frac{y_{j+1}^n - y_{j-1}^n}{2 dx}\right)\\ +
  \frac{2\tau^2 g }{\left( 2 + \gamma \tau\right)} x_j\left(
    \frac{y_{j+1}^n - 2 y_j^n + y_{j-1}^n}{dx^2}\right) +
  \frac{2\tau^2}{\left( 2 + \gamma \tau\right)}\left(\frac{ 2
      y_j^n - y_j^{n-1}}{\tau^2}\right) +
  \frac{ 2\tau^2\gamma}{\left( 2 + \gamma \tau\right)} \left(
    \frac{ y_j^{n-1}}{2 \tau}\right) 
\end{multline}
\ifsolutions
\textit{Solution:}
\begin{multline}
  \left(\frac{y_j^{n+1} - 2 y_j^n + y_j^{n-1}}{\tau^2}\right) - g
  \left( \frac{y_{j+1}^n - y_{j-1}^n}{2 dx}\right) - g x_j\left(
    \frac{y_{j+1}^n - 2 y_j^n + y_{j-1}^n}{dx^2}\right)\\ + \gamma
  \left( \frac{y_j^{n+1} - y_j^{n-1}}{2 \tau}\right) = \frac{f(x_j)}{\mu} \cos
  (\omega t_j)
\end{multline}
\begin{multline}
\left(\frac{y_j^{n+1} - 2 y_j^n + y_j^{n-1}}{\tau^2}\right) +
  \gamma \left( \frac{y_j^{n+1} - y_j^{n-1}}{2 \tau}\right) = \frac{f(x_j)}{\mu}
  \cos (\omega t_j)\\ + g \left( \frac{y_{j+1}^n - y_{j-1}^n}{2
      dx}\right) + g x_j\left( \frac{y_{j+1}^n  - 2 y_j^n +
      y_{j-1}^n}{dx^2}\right) 
\end{multline}
\begin{multline}
  \frac{y_j^{n+1}}{\tau^2} + \frac{\gamma y_j^{n+1}}{2 \tau} -
  \left(\frac{ 2 y_j^n - y_j^{n-1}}{\tau^2}\right) - \gamma \left(
    \frac{ y_j^{n-1}}{2 \tau}\right) = \frac{f(x_j)}{\mu} \cos (\omega t_j)\\ + g
  \left( \frac{y_{j+1}^n - y_{j-1}^n}{2 dx}\right) + g
  x_j\left( \frac{y_{j+1}^n - 2 y_j^n +
      y_{j-1}^n}{dx^2}\right) 
\end{multline}
\begin{multline}
  \frac{ y_j^{n+1}}{\tau^2} + \frac{\gamma y_j^{n+1}}{2 \tau} =
  \frac{f(x_j)}{\mu} \cos (\omega t_j) + g \left( \frac{y_{j+1}^n -
      y_{j-1}^n}{2 dx}\right) + g x_j\left( \frac{y_{j+1}^n - 2
      y_j^n + y_{j-1}^n}{dx^2}\right)\\ + \left(\frac{ 2 y_j^n -
      y_j^{n-1}}{\tau^2}\right) +
  \gamma \left( \frac{ y_j^{n-1}}{2 \tau}\right) 
\end{multline}
\begin{multline}
  \frac{2 y_j^{n+1} + \gamma \tau y_j^{n+1}}{2\tau^2} = \frac{f(x_j)}{\mu} \cos
  (\omega t_j) + g \left( \frac{y_{j+1}^n - y_{j-1}^n}{2
      dx}\right) + g x_j\left( \frac{y_{j+1}^n - 2 y_j^n +
      y_{j-1}^n}{dx^2}\right)\\ + \left(\frac{ 2 y_j^n -
      y_j^{n-1}}{\tau^2}\right) +
  \gamma \left( \frac{ y_j^{n-1}}{2 \tau}\right) 
\end{multline}
\begin{multline}
  y_j^{n+1}\left( 2 + \gamma \tau\right) = 2\tau^2 \frac{f(x_j)}{\mu} \cos
  (\omega t_j) + 2\tau^2 g \left( \frac{y_{j+1}^n - y_{j-1}^n}{2
      dx}\right)\\ + 2\tau^2 g x_j\left( \frac{y_{j+1}^n - 2 y_j^n +
      y_{j-1}^n}{dx^2}\right) + 2\tau^2\left(\frac{ 2 y_j^n -
      y_j^{n-1}}{\tau^2}\right) +
  2\tau^2\gamma \left( \frac{ y_j^{n-1}}{2 \tau}\right) 
\end{multline}
\begin{multline}
  y_j^{n+1} = \frac{2\tau^2}{\left( 2 + \gamma \tau\right)} \frac{f(x_j)}{\mu}
  \cos (\omega t_j) + \frac{2\tau^2 g }{\left( 2 + \gamma
      \tau\right)} \left( \frac{y_{j+1}^n - y_{j-1}^n}{2 dx}\right)\\ +
  \frac{2\tau^2 g }{\left( 2 + \gamma \tau\right)} x_j\left(
    \frac{y_{j+1}^n - 2 y_j^n + y_{j-1}^n}{dx^2}\right) +
  \frac{2\tau^2}{\left( 2 + \gamma \tau\right)}\left(\frac{ 2
      y_j^n - y_j^{n-1}}{\tau^2}\right) +
  \frac{ 2\tau^2\gamma}{\left( 2 + \gamma \tau\right)} \left(
    \frac{ y_j^{n-1}}{2 \tau}\right) 
\end{multline}
% \restoregeometry
\fi \subprob The initial conditions are easy enough to include (we'll
just set the initial displacement and velocity to 0 and let the
driving force create the disturbance) but we'll have to think a bit
harder about the boundary conditions.  At $x=L$, the rope is fixed to
the ceiling so $y(L,t) = 0$.  The harder one is at $x = 0$ where the
chain is allowed to flap around as it pleases.  To get the boundary
condition at $x = 0$ start with the equation that you arrived at in
part (a) and evaluate it at $x=0$.  Then discretize this equation and
solve it for $ y_0^{n+1}$ to get the desired boundary condition. Your answer should look
like this:
\begin{multline}
  y_0^{n+1}   = \left(\frac{4 \tau^2 }{2 + \gamma \tau}\right)\frac{f(x_j)}{\mu} \cos
  (\omega t_j) - \left(\frac{2 - \gamma \tau }{2 + \gamma
      \tau}\right)\left(  y_1^{n - 1} + y_0^{n - 1}\right) \\ +
  \left(\frac{4 }{2 + \gamma \tau}\right)\left(y_1^n + y_0^n\right) +
  \left(\frac{4 \tau^2g }{2 + \gamma \tau}\right)\left( \frac{y_1^n -
      y_0^n}{ dx}\right) - y_1^{n + 1}
\end{multline}

\ifsolutions
\textit{Solution:}
%\newgeometry{left = -40 cm,right = 0 cm}
When $x = 0$, the discrete version of the differential equation is
\begin{multline}
  \left(\frac{y_j^{n+1} - 2 y_j^n + y_j^{n-1}}{\tau^2}\right) - g
  \left( \frac{y_{j+1}^n - y_{j-1}^n}{2 dx}\right) \\ + \gamma
  \left( \frac{y_j^{n+1} - y_j^{n-1}}{2 \tau}\right) = \frac{f(x_j)}{\mu} \cos
  (\omega t_j)
\end{multline}
Let's collect like terms:
\begin{multline}
  y_j^{n+1} \left( {1\over \tau^2} + {\gamma \over 2 \tau}\right)  +
  y_j^{n-1}\left( {1\over \tau^2} - {\gamma \over 2 \tau}\right) - {2\over
      \tau^2} y_j^n - g \left( \frac{y_{j+1}^n - y_{j-1}^n}{2 dx}\right)\\ = \frac{f(x_j)}{\mu} \cos
  (\omega t_j)
\end{multline}
Now we need to evaluate this equation at $x = 0$.  The spatial
derivative term (4th term on l.h.s) is easily evaluated at this point:
\begin{multline}
  y_j^{n+1} \left( {1\over \tau^2} + {\gamma \over 2 \tau}\right)  +
  y_j^{n-1}\left( {1\over \tau^2} - {\gamma \over 2 \tau}\right) - {2\over
      \tau^2} y_j^n - g \left( \frac{y_1^n - y_0^n}{ dx}\right)\\ = \frac{f(x_j)}{\mu} \cos
  (\omega t_j)
\end{multline}
but for the other terms we can't just set $j = 0$ because that's not
the end point of our domain.  Instead we need to set 
\begin{multline}
\left(  \frac{y_1^{n + 1} + y_0^{n+1}}{2}\right) \left( {1\over \tau^2} + {\gamma \over 2 \tau}\right)  +
  \left(\frac{y_1^{n - 1} + y_0^{n-1}}{2}\right)\left( {1\over \tau^2} - {\gamma \over 2 \tau}\right) - {2\over
      \tau^2} \left(\frac{y_1^n + y_0^n}{2}\right) - g \left( \frac{y_1^n - y_0^n}{ dx}\right)\\ = \frac{f(x_j)}{\mu} \cos
  (\omega t_j)
\end{multline}
Now we need to solve for $y_0^{n+1}$:
\begin{multline}
  \left(  \frac{y_1^{n + 1} + y_0^{n+1}}{2}\right) \left( {1\over \tau^2} + {\gamma \over 2 \tau}\right)  \\ = \frac{f(x_j)}{\mu} \cos
  (\omega t_j) - \left(  \frac{y_1^{n - 1} + y_0^{n - 1}}{2}\right)\left( {1\over \tau^2} - {\gamma \over 2 \tau}\right) + {2\over
      \tau^2} \left(  \frac{y_1^n + y_0^n}{2}\right) + g \left( \frac{y_1^n - y_0^n}{ dx}\right)
\end{multline}
\begin{multline}
  \left(  \frac{y_1^{n + 1} + y_0^{n+1}}{2}\right) \left(\frac{2 + \gamma \tau}{2 \tau^2}\right)   = \frac{f(x_j)}{\mu} \cos
  (\omega t_j) - \left(  \frac{y_1^{n - 1} + y_0^{n - 1}}{2}\right) \left(\frac{2 - \gamma \tau}{2 \tau^2}\right)\\ + \left(\frac{y_1^n + y_0^n}{\tau^2}\right) + g \left( \frac{y_1^n - y_0^n}{ dx}\right)
\end{multline}
\begin{multline}
  \left(  y_1^{n + 1} + y_0^{n+1}\right)   = \left(\frac{4 \tau^2 }{2 + \gamma \tau}\right)\frac{f(x_j)}{\mu} \cos
  (\omega t_j) - \left(\frac{4 \tau^2 }{2 + \gamma \tau}\right)\left(
    \frac{y_1^{n - 1} + y_0^{n - 1}}{2}\right) \left(\frac{2 - \gamma \tau}{2 \tau^2}\right)\\ + \left(\frac{4 \tau^2 }{2 + \gamma \tau}\right)\left(\frac{y_1^n + y_0^n}{\tau^2}\right) + \left(\frac{4 \tau^2 }{2 + \gamma \tau}\right) g \left( \frac{y_1^n - y_0^n}{ dx}\right)
\end{multline}
\begin{multline}
  \left(  y_1^{n + 1} + y_0^{n+1}\right)   = \left(\frac{4 \tau^2 }{2 + \gamma \tau}\right)\frac{f(x_j)}{\mu} \cos
  (\omega t_j) - \left(\frac{2 - \gamma \tau }{2 + \gamma
      \tau}\right)\left(  y_1^{n - 1} + y_0^{n - 1}\right) \\ + \left(\frac{4 }{2 + \gamma \tau}\right)\left(y_1^n + y_0^n\right) + \left(\frac{4 \tau^2g }{2 + \gamma \tau}\right)\left( \frac{y_1^n - y_0^n}{ dx}\right)
\end{multline}
\begin{multline}
  y_0^{n+1}   = \left(\frac{4 \tau^2 }{2 + \gamma \tau}\right)\frac{f(x_j)}{\mu} \cos
  (\omega t_j) - \left(\frac{2 - \gamma \tau }{2 + \gamma
      \tau}\right)\left(  y_1^{n - 1} + y_0^{n - 1}\right) \\ +
  \left(\frac{4 }{2 + \gamma \tau}\right)\left(y_1^n + y_0^n\right) +
  \left(\frac{4 \tau^2g }{2 + \gamma \tau}\right)\left( \frac{y_1^n -
      y_0^n}{ dx}\right) - y_1^{n + 1}
\end{multline}

\fi

\subprob With the two initial condition, the two boundary conditions
and the differential equation all discretized, we are ready to code.
Code up the staggered leapfrog for this situation.  Let the animation
run for several periods of the motion to verify that it is behaving in
a way that you would expect.
\end{enumerate}
\end{enumerate}
\ifsolutions
\textit{Solution:}\\
\begin{codeexample}
\begin{VerbatimOut}{\listingFile}
class animatedWave():

    def __init__(self,a,L,N,tau,tMax,gamma, mu,stabilityCheck = False):
        from numpy import arange
        self.L = L
        #self.c = c
        self.N = N
        self.tMax = tMax
        self.tau = tau
        self.gamma = gamma
        self.mu = mu
        self.g = 9.8
        self.omegaD = 15.
        self.dx = (b - a)/N

        
        self.x = arange(a - self.dx/2, b + self.dx,self.dx)

        cTop = self.mu * self.L * self.g
        if self.tau > self.dx/cTop and stabilityCheck:
            print "Beware: You are numerically unstable"
            import sys
            sys.exit()

    def f(self,x):
        import numpy
        if type(x) is numpy.ndarray:
            from numpy import array
            return array([self.f(var) for var in x])
        else:
            if   0.45 < x < 0.5:
                return 1.73
            else:
                return 0
    def initializeWave(self):
        from numpy import exp,zeros

        self.v = zeros(self.N + 2)#exp(-160/self.L**2 * (self.x - self.L/2.)**2) - exp(-160/self.L**2 * (0 - self.L/2.)**2)
        self.y = zeros(self.N + 2)

    def animate(self):
        from numpy import zeros_like,copy,cos
        #        constant = self.c**2 * self.tau**2/(2 * self.dx**2)

        constant = 2 * self.tau**2/(2 + self.gamma * self.tau)

        constantFive = 4 * self.tau**2/(2 + self.gamma * self.tau)
        yOld = zeros_like(self.y)

        t = 0
        counter = 0
        while t < self.tMax:
            yNew = zeros_like(self.y)
            yNew[1:self.N+1] = constant * self.f(self.x[1:self.N + 1])/self.mu * cos(self.omegaD * t) + constant * self.g * (self.y[2:self.N + 2] - self.y[0:self.N])/(2 * self.dx) + constant * self.g * self.x[1:self.N + 1] * (self.y[2:self.N + 2] - 2 * self.y[1:self.N + 1] + self.y[0:self.N])/self.dx**2 + constant * (2 * self.y[1:self.N + 1] - yOld[1:self.N + 1])/self.tau**2 + constant * self.gamma *yOld[1:self.N +1]/(2 * self.tau)
#            print 'first term', - constantFive * (2 - self.gamma * self.tau)/(4 * self.tau**2) *(yOld[1] + yOld[0])
#            print 'second term',  constantFive /self.tau**2 *(self.y[1] + self.y[0])
#            print 'third term', - yNew[1]
#            print 'fourth term', self.g /self.dx * constantFive *(self.y[1] - self.y[0])
#            print 'fifth term',   self.f(0)/self.mu * constantFive * cos(self.omegaD * t)
            
            yNew[0] =    - constantFive * (2 - self.gamma * self.tau)/(4 * self.tau**2) *(yOld[1] + yOld[0])  +   constantFive /self.tau**2 *(self.y[1] + self.y[0]) - yNew[1] + self.g /self.dx * constantFive *(self.y[1] - self.y[0]) +  self.f(0)/self.mu * constantFive * cos(self.omegaD * t)
            #            print yNew[0], 't = ', t
            #input('press a key')
            yNew[-1] = -yNew[-2]

            if counter %100 == 0:
                from numpy import pi
                from matplotlib import pyplot
                pyplot.plot(self.y,self.x,'r.-')
                pyplot.xlim(-1.25,1.25)
                pyplot.ylim(-0.05,1.25)
                pyplot.title("T = " + str(2 * pi/self.omegaD)+  ' s\n time:'+ str(t))
                pyplot.draw()
                pyplot.pause(.001)
                pyplot.clf()



            yOld = copy(self.y)
            self.y = copy(yNew)
            t += self.tau
            counter += 1

a = 0
b = 1.
N = 200
tau = .0001
gamma = 20
mu = 0.003
tMax = 10
myWave = animatedWave(a,b,N,tau,tMax,gamma,mu,stabilityCheck=True)
myWave.initializeWave()
myWave.animate()
\end{VerbatimOut}
\end{codeexample}
\else
\noindent\rule{5 in}{0.01 in}
\fi

%Challenge Problem.
%Use the staggered leapfrog algorithm to study how pulses propagate along
%a hanging chain.
%You will have to work at making the boundary condition at the bottom of
%the chain work properly.
%Derive this boundary condition from the wave equation on the chain by looking
%at the wave equation at $x=0$.
