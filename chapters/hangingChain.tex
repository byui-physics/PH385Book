\chapter{The Hanging Chain}
\label{Lab:12} \index{Hanging chain}

The resonance modes that we studied in Lab~\ref{Lab:11} were simply
sine functions.  We can also use these techniques to analyze more
complicated systems.  In this lab we first study the problem of
standing waves on a hanging chain. This  problem was first solved in
the 1700's by Johann Bernoulli and is the first time that the
function that later became known as the $J_0$ Bessel function showed
up in physics.  Then we will jump forward several centuries in
physics history and study bound quantum states using the same
techniques.


\labsection{Resonance for a hanging chain}

\marginfig{Figures/f04Mode1}{The first normal mode for a hanging chain.}

Consider the chain hanging from the ceiling in the
classroom.\footnote{For more analysis of the hanging chain, see N.\
Asmar, {\it Partial Differential Equations and Boundary Value
Problems} (Prentice Hall, New Jersey, 2000), p. 299-305.} We are
going to find its normal modes of vibration using the method of
Problem~\ref{P:11.2}.  The wave equation for transverse waves on a
chain with varying tension $T(x)$ and constant linear mass density
$\mu$\footnote{Equation~(\ref{eq:ChainEq}) also works for systems
with varying mass density if you replace $\mu$ with a function
$\mu(x)$, but the equations derived later in the lab are more
complicated with a varying $\mu(x)$.} is given by
\begin{equation}\label{eq:ChainEq}
    \mu \frac{\partial^2 y}{\partial t^2} - \frac{\partial}{\partial x}
    \left(  T(x) \frac{\partial y}{\partial x} \right) = 0
\end{equation}
Let's use a coordinate system that starts at the bottom of the chain
at $x=0$ and ends on the ceiling at $x=L$.


\begin{enumerate}
\probtwo \label{P:12.1}
By using the fact that the stationary chain is in vertical
equilibrium, show that the tension in the chain as a function of $x$
is given by
\begin{equation}
    T(x) = \mu g x
\end{equation}
where $\mu$ is the linear mass density of the chain and
where $g=9.8~{\rm m/s}^2$ is the acceleration of gravity.
Then show that Eq.~\eqref{eq:ChainEq} reduces to
\begin{equation}\label{eq:ChainEq2}
    \frac{\partial^2 y}{\partial t^2} - g \frac{\partial}{\partial x}
    \left(  x \frac{\partial y}{\partial x} \right) = 0
\end{equation}
for a freely hanging chain.
\end{enumerate}

As in Lab~\ref{Lab:11}, we now look for normal modes by separating the
variables: $y(x,t) = f(x) \cos{(\omega t)}$. We then substitute this
form for $y(x,t)$ into \eqref{eq:ChainEq2} and simplify to obtain
\begin{equation}
    x {d^2 f \over dx^2} +
    {df \over dx} = -{\omega^2 \over g} f
    \label{modeq}
\end{equation}
which is in eigenvalue form with $\lambda=-\omega^2/g$. 

\marginfig[0.5in]{Figures/f04Grid}{A cell-centered grid with ghost
points. (The open circles are the ghost points.)}

The boundary condition at the ceiling is $f(L)=0$ while the
boundary condition at the bottom is obtained by taking the
limit of Eq.~(\ref{modeq}) as $x \rightarrow 0$ to find
\begin{equation}
    f'(0)=-{\omega^2 \over g} f(0) = \lambda f(0)
    \label{bottom}
\end{equation}
In the past couple labs we have dealt with derivative boundary
conditions by fitting a parabola to the last three points on
the grid and then taking the derivative of the parabola (see
Problem~\ref{P:10.6}).  This time we'll handle
the derivative boundary condition by using a cell-centered grid
with ghost points, as discussed in Lab~\ref{Lab:grids}.

 \index{Cell-center grid}
\index{Grids!cell-center}

\marginfig[0.75in]{Figures/f04Mode2}{The shape of the second mode of a hanging
chain}

Recall that a cell-center grid divides the computing region from $0$
to $L$ into $N$ cells with a grid point at the center of each cell.
We then add two more grid points outside of $[0,L]$, one at
$x_1=-h/2$ and the other at $x_{N+2}=L+h/2$. \index{Ghost points} The
ghost points are used to apply the boundary conditions. Notice that
by defining $N$ as the number of interior grid points (or cells), we
have $N+2$ total grid points, which may seem weird to you. We prefer
it, however, because it reminds us that we are using a cell-centered
grid with $N$ physical grid points and two ghost points. You can do
it any way you like, as long as your counting method works.

Notice that there isn't a grid point at either endpoint, but
rather that the two grid points on each end straddle the
endpoints. If the boundary condition specifies a value, like
$f(L)=0$ in the problem at hand, we use a simple average like
this:
\begin{equation}
    {f_{N+2}+f_{N+1} \over 2} = 0~~,
    \label{topbdy}
\end{equation}
and if the condition were $f'(L)=0$ we would use
\begin{equation}
    {f_{N+2}-f_{N+1} \over h} = 0~~.
\end{equation}

\index{Generalized eigenvalue problem} \index{Eigenvalue
problem!generalized} When we did boundary conditions in the
eigenvalue calculation of Problem~\ref{P:11.2} we used a $\bf B$
matrix with zeros in the top and bottom rows and we loaded the
top and bottom rows of $\bf A$ with an appropriate boundary
condition operator. Because the chain is fixed at the ceiling
($x=L$) we use this technique again in the bottom rows of $\bf
A$ and $\bf B$, like this (after first loading $\bf A$ with zeros
and $\bf B$ with the identity matrix):
\begin{equation}
    A[-1,-2]={1 \over 2}~~~~A[-1,-1]={1 \over 2} ~~~~B[-1,-1]=0
\end{equation}

\begin{enumerate}
\probtwo \label{P:12.2}
\begin{enumerate}
\subprob
Verify that these choices for the bottom rows of $A$ and $B$ in the
generalized eigenvalue problem
\begin{equation}
    {\bf A} f = \lambda {\bf B} f
\end{equation}
give the boundary condition in Eq.~(\ref{topbdy})
at the ceiling no matter what
$\lambda$ turns out to be.


\subprob

\marginfig{Figures/f04Mode3}{The shape of the third mode of a hanging chain}

    Now let's do something similar with the derivative
    boundary condition at the bottom, Eq.~(\ref{bottom}).
    Since this condition is already in eigenvalue form we
    don't need to load the top row of $\bf B$ with zeros.
    Instead we load $\bf A$ with the operator on the left
    ($f'(0)$) and $\bf B$ with the operator on the right
    ($f(0)$), leaving the eigenvalue $\lambda=-\omega^2/g$
    out of the operators so that we still have ${\bf A} f =
    \lambda {\bf B} f$. Verify that the following choices for
    the top rows of $\bf A$ and $\bf B$ correctly produce
    Eq.~(\ref{bottom}).
    \begin{equation}
        A[0,0]=-{1 \over h} ~~~~
        A[0,1]={1 \over h} ~~~~
        B[0,0]={1 \over 2} ~~~~
        B[0,1]={1 \over 2}
    \end{equation}


\subprob

Write the finite difference form of Eq.~(\ref{modeq})
and use it to load the matrices $\bf A$ and $\bf B$ for
a chain with $L=2$~m. (Notice that for the interior points the
matrix $\bf B$ is just the identity matrix with 1 on the
main diagonal and zeros everywhere else.) Use Python to
solve for the normal modes of vibration of a hanging
chain. As in Lab~\ref{Lab:11}, some of the eigenvectors
are unphysical because they don't satisfy the boundary
conditions; ignore them.

\subprob Solve Eq.~(\ref{modeq}) analytically using
Mathematica without any boundary conditions. You will
encounter the Bessel functions $J_0$ and $Y_0$, but
because $Y_0$ is singular at $x=0$ this function is not
allowed in the problem. Apply the condition $f(L)=0$ to
find analytically the mode frequencies $\omega$ and
verify that they agree with the frequencies you found
in part (c).
\end{enumerate}
\end{enumerate}
\ifsolutions
\textit{Solution:}\\
\begin{codeexample}
\begin{VerbatimOut}{\listingFile}

class HangingChain():

    def __init__(self,a,b,N):
        from numpy import arange,linspace
        self.L = b
        self.N = N
        self.dx = (b - a)/N
        self.x = linspace(a - self.dx/2., b + self.dx/2., N + 2)
        #        self.x = arange(a - self.dx/2.,b + self.dx,self.dx)
        self.g = 9.8

    def loadMatrices(self):
        from numpy import zeros,eye
        
        self.A = zeros([self.N + 2,self.N + 2])
        # Fill in top and bottom rows of A later
        self.A[-1,-1] = 1/2.
        self.A[-1,-2] = 1/2.
        self.A[0,0] = -1./self.dx
        self.A[0,1] = 1./self.dx
        
        
        for i in range(1,self.N + 1):
            self.A[i,i] = - 2 * self.x[i]/self.dx**2
            self.A[i,i-1] = self.x[i]/self.dx**2 - 1./(2 * self.dx)
            self.A[i,i+1] =  self.x[i]/self.dx**2 + 1./(2 * self.dx)

        self.B = eye(self.N + 2)
        self.B[-1,-1] = 0
        self.B[0,0] = 1./2
        self.B[0,1] = 1./2
        

    def solveProblem(self):
        from scipy.linalg import eig
        from numpy import pi, sqrt
        
        self.eVals,self.eVecs = eig(self.A,self.B)
        self.omega = sqrt(-self.eVals * self.g) /(2 * pi)
        self.key = sorted(range(self.N + 2), key=lambda k: self.omega[k])


    def plot(self,mode):
        from matplotlib import pyplot
        from numpy import real
        pyplot.plot(self.eVecs[:,self.key[mode]],self.x,'r.-')
        pyplot.title('mode: ' + str(mode) + '\n' + str(real(self.omega[self.key[mode]])) + ' Hz')
        pyplot.axes().set_aspect('equal')
        pyplot.show()

a = 0.
b = 2.
N = 500
myChain = HangingChain(a,b,N)
myChain.loadMatrices()
myChain.solveProblem()
myChain.plot(10)


\end{VerbatimOut}
\end{codeexample}
\else
\noindent\rule{5 in}{0.01 in}
\fi


