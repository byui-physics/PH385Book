\chapter{Fourth-order Runge-Kutta and LeapFrog}
\label{Lab:8}

So far we have used Euler's method and second-order Runge-Kutta.
Neither of these methods conserve energy, but we have seen that
second-order RK is more accurate than Euler.  Today we will improve upon the
second-order method and finally learn a method that
conserves energy.

\labsection{Fourth-Order Runge-Kutta} Recall that one way to arrive at
the equations for second-order Runge-Kutta is by performing a Taylor
expansion to estimate forward in time and subtracting from it a Taylor
expansion that estimates backwards in time (see section
\ref{section:second-orderRK}).  This results in the cancelation of
higher order terms ($h^2$ in the case of second-order RK) and hence
leads to a more accurate method. By performing more Taylor expansions
about different points and strategically adding/subtracting them we
can force even higher order terms to cancel out and hence arrive at
better and better numerical methods.  This comes at the expense of
greater and greater algorithmic complexity.

The fourth-order Runge-Kutta algorithm is widely accepted to be the best
balance between numerical accuracy and algorithmic complexity.  We
won't give all of the mathematical detail here although I hope my
description in the preceding paragraph gives you a good idea of how we
might do it.  For us it will suffice to state the equations without
proof:
\noindent
\begin{align}
k_1 &= \tau f(x_n,t_n)\\
k_2 &= \tau f(x_n + {1\over 2} k_1,t_{n+{1\over 2}})\\
k_3 &= \tau f(x_n + {1\over 2} k_2,t_{n+{1\over 2}})\label{eq:RK4-k3}\\
k_4 &= \tau f(x_n + k_3,t_{n+1}) \label{eq:RK4-k4}\\
x_{n+1} &= x_n + {1\over 6}\left( k_1 + 2 k_2 + 2 k_3 + k_4)\right) \label{eq:RK4-update}
\end{align}

If your implementation of second-order Runge-Kutta from last time was the more
elegant version, then modifying your code for this fourth-order method
is very simple.  You'll only need to add lines \eqref{eq:RK4-k3} and
\eqref{eq:RK4-k4} and modify line \eqref{eq:RK4-update}
\begin{enumerate}
  \probtwo Let's investigate how much better this approach is compared
  to Euler's and second-order RK.
\begin{enumerate}
\item Make a copy of your code from problem \ref{prob:RK2andEuler}.
\item Add a function that uses the fourth-order Runge-Kutta method to
  calculate the trajectory of the batted ball.  This function will be
  nearly identical to your elegant version of second-order RK, with
  only a few modifications/additions.
\item In the loop over step sizes add a few lines to call the
  fourth-order function and save the range that it predicts.\\
\textbf{Note}: To obtain an exact replica of figure \ref{fig:Euler-RK2-RK4},
let $dt = 0.7$ initially and decrease by factors of $1.5$ in the loop.
Also, use the following initial conditions:\\
\begin{minipage}{0.5\linewidth}
\[v = 400 \mathrm{~m/s}\]
\[\theta = 45^\circ\] 
\[d = 0.075 \mathrm{~m}\]
\end{minipage}
\begin{minipage}{0.5\linewidth}
\[m = 0.145\mathrm{~kg}\]
\[C = 0.5\] 
\[\rho = 1.22 \mathrm{~kg/m}^3\] 
\end{minipage}

\item Plot range vs. step size for all three methods that we have
  learned so far (Euler, Second-order RK, and Fourth-order RK).  Use
  Logarithmic scales for the horizontal axis and modify the vertical size of the
  plot so as to capture a $20$ meter window centered on the range.
  Your plot should look something like figure \ref{fig:Euler-RK2-RK4}
\marginfig[-1in]{Figures/Euler-RK2-RK4}{\label{fig:Euler-RK2-RK4}Comparison
of the rate of convergence for Euler's method, second-order
Runge-Kutta, and fourth-order Runge-Kutta.}
\end{enumerate}
\ifsolutions
\textit{Solution:}\\
\begin{codeexample}
\begin{VerbatimOut}{\listingFile}
from numpy import sinh,arange,pi,cos,sin
from matplotlib import pyplot



class projectile():

    def __init__(self, g = 9.8, dt = 0.1):
        self.g = g
        self.dt = dt

        # For part d
    def setInitialConditions(self,r,v):
        self.initialPos = r
        self.initialV = v

    def C(self,v):
        return .198 + .295/(1 + exp((v - 35)/5) )

    def setDragSettings(self,A,C,m,rho,variableC = False):
        # Most of the logic here is for part d
        if not variableC:
            self.variableC = False
            self.dragFactor = 1/2. * A * C * rho/m  #Only include this line for previous parts
        else:
            self.variableC = True
            self.constant = 1/2. * A * rho/m

    def getDerivs(self,varsVec):
        from numpy.linalg import norm
        from numpy import array
        v = varsVec[2:]
        xDeriv = v[0]
        yDeriv = v[1]
        speed = norm([v[0],v[1]])
        if self.variableC:
            self.dragFactor = self.constant * self.C(speed)
        dragX = self.dragFactor * speed * v[0]
        dragY = self.dragFactor * speed * v[1]
        vxDeriv = - dragX
        vyDeriv = - (dragY + self.g)
        return array([xDeriv,yDeriv,vxDeriv,vyDeriv])


    def Euler(self):
        from numpy.linalg import norm

        self.x = [self.initialPos[0]]
        self.y = [self.initialPos[1]]
        vx = [self.initialV[0]]
        vy = [self.initialV[1]]

        while self.y[-1] > 0:

            speed = norm([vx[-1],vy[-1]])
            if self.variableC:
                self.dragFactor = self.constant * self.C(speed)
            dragX = self.dragFactor * speed * vx[-1]
            dragY = self.dragFactor * speed * vy[-1]
            self.y.append(self.y[-1] + vy[-1] * self.dt)
            self.x.append(self.x[-1] + vx[-1] * self.dt)
            vy.append(vy[-1] - (dragY + self.g) * self.dt)
            vx.append(vx[-1] - dragX * self.dt)

    def RK4(self):
        from numpy import array

        self.x = [self.initialPos[0]]
        self.y = [self.initialPos[1]]
        vx = [self.initialV[0]]
        vy = [self.initialV[1]]
        r = array([self.x[-1],self.y[-1],vx[-1],vy[-1]])
        while self.y[-1] > 0:
            k1 = self.dt * self.getDerivs(r)
            k2 = self.dt * self.getDerivs(r + 1/2. * k1)
            k3 = self.dt * self.getDerivs(r + 1/2. * k2)
            k4 = self.dt * self.getDerivs(r + k3)
            r += 1/6. * (k1 + 2 * k2 + 2 * k3 + k4 )
            self.x.append(r[0])
            self.y.append(r[1])


    def RK2(self):
        from numpy import array

        self.x = [self.initialPos[0]]
        self.y = [self.initialPos[1]]
        vx = [self.initialV[0]]
        vy = [self.initialV[1]]
        r = array([self.x[-1],self.y[-1],vx[-1],vy[-1]])
        while self.y[-1] > 0:
            k1 = self.dt * self.getDerivs(r)
            k2 = self.dt * self.getDerivs(r + 1/2. * k1)
            r += k2
            self.x.append(r[0])
            self.y.append(r[1])

    def plotTrajectory(self):
        from matplotlib import pyplot
        pyplot.plot(self.x,self.y)
        # pyplot.show()


from numpy import exp
speed = 400#49
angle = 45 * pi/180.
diameter = 0.075
radius = diameter /2
area = pi * radius**2
mass = .145
C = 0.5
rho = 1.22

dt = .7
eulerRanges = []
rkRanges = []
rkFourRanges = []
dts = []
while dt > 1e-3:
    dts.append(dt)
    trajectoryOne = projectile(dt=dt)
    trajectoryOne.setInitialConditions([0,1],[speed * cos(angle),speed * sin(angle)])
    trajectoryOne.setDragSettings(area,C,mass,rho,variableC = True)
    trajectoryOne.Euler()
    eulerRanges.append(trajectoryOne.x[-1])
    trajectoryOne.setInitialConditions([0,1],[speed * cos(angle),speed * sin(angle)])
    trajectoryOne.RK2()
    rkRanges.append(trajectoryOne.x[-1])

    trajectoryOne.setInitialConditions([0,1],[speed * cos(angle),speed * sin(angle)])
    trajectoryOne.RK4()
    rkFourRanges.append(trajectoryOne.x[-1])
    dt /=1.5

pyplot.plot(dts,eulerRanges,label='Euler',linewidth=4.0)
pyplot.xscale('log')
#pyplot.yscale('log')
pyplot.plot(dts,rkRanges,label = "2nd order RK",linewidth=4.0)
pyplot.xscale('log')
pyplot.plot(dts,rkFourRanges,label = "4th order RK",linewidth=4.0)
pyplot.xticks(fontsize=30)
pyplot.yticks(fontsize=30)
pyplot.tight_layout()
pyplot.xscale('log')
pyplot.xlabel('Step size (s)')
pyplot.ylabel('Projectile Range')
pyplot.ylim(rkFourRanges[-1] - 10,rkFourRanges[-1] + 10)
pyplot.legend()
pyplot.savefig('Euler-RK2-RK4.png')
#pyplot.show()
\end{VerbatimOut}
\end{codeexample}
\else
\noindent\rule{4 in}{0.01 in}
\fi

\end{enumerate}
As you can see in figure \ref{fig:Euler-RK2-RK4}, each method gets
successively more accurate.  For example, you would have to use
$dt=0.05$ with second-order RK to get results of the same accuracy as
you would get with $dt = 0.5$ with fourth-order Runge-Kutta.  In other
words, a fourth-order Runge-Kutta approach could achieve
higher-quality results using less computational resources and time.
It's just better.

\labsection{The leapfrog method}
It's now time to learn a method that conserves energy.  It's worth
mentioning that even though the methods that we have learned do not
conserve energy this doesn't mean that they aren't useful.  There are
plenty of situations where the rate at which the energy increases will
be so small that it won't affect your results appreciable.  Generally
speaking, you'll want your numerical method to conserve energy for
motion that is repeating or that runs for long periods of time.

The leapfrog method is very similar to second-order Runge-Kutta.  We
start by stepping forward one time step using the RK2 algorithm:

\begin{align}
x_{n + {1\over 2}} = x_n + {\tau\over 2} f(x_n,t_n)& ~~\mathrm{(Euler's~ step)}\label{eq:leapfrogOne}\\
x_{n + 1} = x_n + \tau f(x_{n + {1\over 2}},t_{n + {1\over 2}}) \label{eq:leapfrogTwo}&
~~\mathrm{(\nth{2} ~Order~ Runge-Kutta)}
\end{align}

At this point, the Runge-Kutta method uses $x_{n+1}$ with an Euler step to perform the
next half step.

\begin{align}
x_{n + {3\over 2}} = x_{n+1} + {\tau\over 2} f(x_{n+1},t_{n+1})&\label{eq:leapfrogThree}\\
\end{align}

\marginfig[-1in]{Figures/leapfrog}{\label{fig:leapfrog}Diagram of the
  leapfrog method for solving first-order differential equations.} 
The algorithm continues this pattern, using an Euler step followed by full step at
each iteration.  We've already learned that Euler steps should be
avoided if we can help it.  We obviously needed an Euler step to 
make our first half step but it might be possible to avoid using Euler to
make the remaining half steps.  What if instead we used the previous
half step to calculate the next half step.  In other words:

\begin{align}
x_{n + {3\over 2}} = x_{n+{1\over 2}} + \tau f(x_{n+1},t_{n+1})&\\
\end{align}

This would be a better approach since the derivative is
being evaluated at the halfway point (i.e.  Both sides of this
equation are centered at the same point.) The remainder of the steps
are all full time steps alternating between the midpoint calculations
and the on-grid calculations. A diagramatic depiction of
the leapfrog method is given in figure \ref{fig:leapfrog}



\begin{enumerate}
\probtwo Let's use leapfrog to study the orbit of the earth around the
sun.
\begin{enumerate}
\item Use figure \ref{fig:Earth-Sun} to convince yourself that the
  components of gravity on the earth are given by:
\begin{equation}
\frac{d^2x}{dt^2} = -\frac{G M_s r_x}{r^3}
\end{equation}
\begin{equation}
\frac{d^2y}{dt^2} = -\frac{G M_s r_y}{r^3}
\end{equation}
\marginfig[-1in]{Figures/Earth-Sun}{\label{fig:Earth-Sun}Decomposition
of the graviational forces on the Earth}
\item The fourth-order Runge-Kutta algorithm is for first-order
  differential equations, not second order ones.  So you'll need to
  write each of these second-order differential equations as two,
  coupled, first-order equations.  Convince  yourself that the
  equations below are the correct set:
\begin{eqnarray}
\frac{dv_x}{dt} &= -\frac{G M_s r_x}{r^3}& \hspace{2cm}\frac{dv_y}{dt} = -\frac{G M_s r_y}{r^3}\\
\frac{dx}{dt} &= v_x & \hspace{2cm}\frac{dy}{dt} = v_y
\end{eqnarray}
\item Use the leapfrog method to calculate the trajectory of Earth.
  Appropriate length and time scales for this problem are AU
  (astronomical units) and years.  It turns out that $G M_S = 4 \pi^2$
  in these units.  Note also that the distance from the sun to the
  earth is $1 $ AU.
\item Add some code that will calculate the total energy of earth
  throughtout it's orbit.  Make a plot of total energy vs. time.  What
  did you learn?
\item  What if gravity was not an inverse square law?  What if the
  force of gravity was:
\begin{equation}
F_g = {G M_s M_e\over r^{2.5}}
\end{equation}
Modify your code to investigate this.  To see the effect you should
modify your intial conditions so that the orbit is eliptical.  You can
do this by decreasing your initial velocity by a small amount.
\item Investigate what Earth's orbit would look like if gravity was an
  inverse-cube law.
\end{enumerate}
\end{enumerate}
\ifsolutions
\textit{Solution:}\\
\begin{codeexample}
\begin{VerbatimOut}{\listingFile}
class orbit():
    def __init__(self,tMax = 1,dt = 0.01, mEarth = 5.98e24):
        self.dt = dt
        self.tMax = tMax
        self.mEarth = mEarth
    def setInitialConditions(self,r,v):
        self.r = [r]
        self.v = [v]


    def getDerivs(self,varsVec):
        from numpy import sin,array
        from numpy.linalg import norm
        v = varsVec[2:]
        r = array(varsVec[:2])
        xDeriv = v[0]
        yDeriv = v[1]

        vxDeriv = - 4 * pi**2 * varsVec[0]/norm(r)**3.5
        vyDeriv = - 4 * pi**2 * varsVec[1]/norm(r)**3.5

        return array([xDeriv,yDeriv,vxDeriv,vyDeriv])

    def leapfrog(self):
        from numpy import arange,sin,array
        from numpy.linalg import norm

        # Use Euler to perform the first half time step
        r = array([self.r[0][0],self.r[0][1],self.v[0][0],self.v[0][1]])
        k1 = self.dt * self.getDerivs(r)
        midPointVars = r + 1/2. * k1
        # Done with first half time step.
        
        self.E = [1/2. * self.mEarth * norm(r[2:])**2 - 4 * pi**2 * self.mEarth/norm(r[:2])]
        self.t = [0]
        while self.t[-1] < self.tMax:
            # Using the midpoint, make a full step
            k2 = self.dt * self.getDerivs(midPointVars)
            r += k2
            # Step from the previous midPoint to the next midPoint.
            midPointVars +=  self.dt * self.getDerivs(r)
            self.r.append([r[0],r[1]])
            self.v.append([r[2],r[3]])
            self.t.append(self.t[-1] + self.dt)
            self.E.append(1/2. * self.mEarth * norm(r[2:])**2 - 4 * pi**2 * self.mEarth/norm(r[:2]))


    def plot(self):
        from matplotlib import pyplot
        x = [var[0] for var in self.r]
        y = [var[1] for var in self.r]
        pyplot.plot(x,y)
        pyplot.show()

    def plotEnergy(self):
        from matplotlib import pyplot
        print len(self.t), len(self.K)
        pyplot.plot(self.t,self.E)
        pyplot.show()

from numpy import pi
r = [1,0]
v = [0,2 * pi - 1.2]

myOrbit = orbit(tMax = 5,dt = 0.001)
myOrbit.setInitialConditions(r,v)
myOrbit.leapfrog()
myOrbit.plot()
\end{VerbatimOut}
\end{codeexample}
\else
\noindent\rule{4 in}{0.01 in}
\fi


You should have noticed that although the total energy does
fluctuate during the orbit, it always returns to the same value once
the orbit returns to it's starting point.  An algorithm like this
would be a better choice if you need a solution over long time
intervals.  In contrast, the energy would slowly increase if we used a
method like Runge-Kutta.

\labsection{Homework}

\begin{enumerate}
\prob Investigate the effect of Jupiter on Earth's orbit.  Here are
some steps to follow:
\begin{enumerate}
\item You'll need to track the motion of both Earth and Jupiter.  This
  means that you'll need to consider more forces:
\begin{enumerate}
\item Gravity of Sun on Earth
\item Gravity of Sun on Jupiter
\item Gravity of Jupiter on Earth
\item Gravity of Earth on Jupiter
\end{enumerate}
Add the necessary forces into your function \texttt{getDervis} (or
whatever you called it)
\item Previously, when you calculated the force due to the Sun on the
  earth, you did:
\begin{align}
\frac{G M_s M_e}{r^2} &= M_e a\\ 
\frac{G M_s}{r^2} &= a\\ 
\frac{4 \pi^2}{r^2} &= a\\ 
\end{align}
When you calculate the force between Earth and Jupiter, the mass of
the sun does not appear.  However, so that you can continue to use the $G M_s = 4 \pi^2$ you can do
something like:
\begin{align}
\frac{G M_J M_e}{r^2} &= M_e a\\ 
\frac{G M_J}{r^2} &= a\\ 
\frac{G M_S}{r^2}\frac{M_J}{M_S} &= a\\ 
\frac{4 \pi^2}{r^2}\frac{M_J}{M_S} &= a\\ 
\end{align}

\item The period of Jupiter's orbit is $11.86$ years and it's distance
  from the sun is $\approx 5.2$ AU.  Jupiter's mass is
  $\num{1.898e27}$ kg. Use this information to initialize the position
  and velocity of Jupiter.
\item Plot the orbit of both planets for $12$ years and observe the
  effect on Earth's orbit.  To really see the effect you'll have to zoom
  in on Earth's orbit.  You can turn the effect of Jupiter on and off
  by changing the mass of Jupiter to a small number
\item How would Earth's orbit change if Jupiter's mass was bigger by a
  factor of $10$?, $100$?, $1000$?
\end{enumerate}

\end{enumerate}
\ifsolutions
\textit{Solution:}\\
\begin{codeexample}
\begin{VerbatimOut}{\listingFile}


class orbit():
    def __init__(self,tMax = 60,mEarth = 5.98e24,mJupiter = 1.898e27,mSun = 1.989e30,dt = .01):
        self.dt = dt
        self.tMax = tMax
        self.mEarth = mEarth
        self.mJupiter = mJupiter
        self.mSun = mSun

    def setInitialConditions(self,rE,vE,rJ,vJ):
        self.rE = [rE]
        self.vE = [vE]
        self.rJ = [rJ]
        self.vJ = [vJ]
        

    def getDerivs(self,varsVec):
        from numpy import array,pi,sin
        from numpy.linalg import norm

        rE = array(varsVec[:2])
        vE = varsVec[2:4]
        rJ = array(varsVec[4:6])
        vJ = varsVec[6:]

        xEDeriv = vE[0]
        yEDeriv = vE[1]

        xJDeriv = vJ[0]
        yJDeriv = vJ[1]

        rEJ = rJ - rE
        
        vxEDeriv = - 4 * pi**2 * rE[0]/norm(rE)**3 + 4 * pi**2 * self.mJupiter/self.mSun * rEJ[0]/norm(rEJ)**3
        vyEDeriv = - 4 * pi**2 * rE[1]/norm(rE)**3 + 4 * pi**2 * self.mJupiter/self.mSun * rEJ[1]/norm(rEJ)**3

        vxJDeriv = - 4 * pi**2 * rJ[0]/norm(rJ)**3 - 4 * pi**2 * self.mEarth/self.mSun * rEJ[0]/norm(rEJ)**3
        vyJDeriv = - 4 * pi**2 * rJ[1]/norm(rJ)**3 - 4 * pi**2 * self.mEarth/self.mSun * rEJ[1]/norm(rEJ)**3

        return array([xEDeriv,yEDeriv,vxEDeriv,vyEDeriv,xJDeriv,yJDeriv, vxJDeriv,vyJDeriv])


    def RK4(self):
        from numpy import array

        r = array([self.rE[0][0],self.rE[0][1],self.vE[0][0],self.vE[0][1],self.rJ[0][0],self.rJ[0][1],self.vJ[0][0],self.vJ[0][1]])
        t = 0
        self.time = [t]
        while t < self.tMax:
            k1 = self.dt * self.getDerivs(r)
            k2 = self.dt * self.getDerivs(r + 1/2. * k1)
            k3 = self.dt * self.getDerivs(r + 1/2. * k2)
            k4 = self.dt * self.getDerivs(r + k3)
            r += 1/6. * (k1 + 2 * k2 + 2 * k3 + k4 )
            self.rE.append( [r[0],r[1]]  )
            self.vE.append( [r[2],r[3]]  )
            self.rJ.append( [r[4],r[5]]  )
            self.vJ.append( [r[6],r[7]]  )
            t += self.dt
            self.time.append(t)


    def leapfrog(self):
        from numpy import array
        # Use 2nd order RK to perform the first time step
        t = 0
        self.time = [t]
        r = array([self.rE[0][0],self.rE[0][1],self.vE[0][0],self.vE[0][1],self.rJ[0][0],self.rJ[0][1],self.vJ[0][0],self.vJ[0][1]])
        k1 = self.dt * self.getDerivs(r)
        midPointVars = r + 1/2. * k1
        k2 = self.dt * self.getDerivs(r + 1/2. * k1)
        r += k2
        # First time step is done
        self.rE.append( [r[0],r[1]]  )
        self.vE.append( [r[2],r[3]]  )
        self.rJ.append( [r[4],r[5]]  )
        self.vJ.append( [r[6],r[7]]  )
        t += self.dt
        self.time.append(t)
        while t < self.tMax:
            # Find next
            midPointVars = midPointVars + self.dt * self.getDerivs(r)
            r = r + self.dt * self.getDerivs(midPointVars)

            self.rE.append( [r[0],r[1]]  )
            self.vE.append( [r[2],r[3]]  )
            self.rJ.append( [r[4],r[5]]  )
            self.vJ.append( [r[6],r[7]]  )

            t += self.dt
            self.time.append(t)
            
    def plot(self):
        from matplotlib import pyplot
        pyplot.figure()
        xE = [vars[0] for vars in self.rE]
        yE = [vars[1] for vars in self.rE]
        xJ = [vars[0] for vars in self.rJ]
        yJ = [vars[1] for vars in self.rJ]
        pyplot.plot(xE,yE)
        pyplot.plot(xJ,yJ)
        pyplot.show()


from numpy import pi,array,abs
from matplotlib import pyplot
import sys

rE = [1,0]
vE = [0,2 * pi]
rJ = [5.2,0]
vJ = [0, 2 * pi * 5.2/11.86]

myOrbits = orbit()
myOrbits.setInitialConditions(rE,vE,rJ,vJ)
myOrbits.leapfrog()
myOrbits.plot()
\end{VerbatimOut}
\end{codeexample}
\else
\noindent\rule{4 in}{0.01 in}
\fi
