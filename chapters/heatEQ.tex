\chapter{The Diffusion, or Heat, Equation}
\label{Lab:17}

\index{Diffusion equation}

Now let's attack the diffusion equation \footnote{N.\
Asmar, {\it Partial Differential Equations and Boundary Value
Problems} (Prentice Hall, New Jersey, 2000), p. 110-129.}
\begin{equation}
    \frac{\partial T}{\partial t} = D \frac{\partial ^2 T}{\partial x^2}~~~.
    \label{diffusioneq}
\end{equation}
This equation describes how the distribution $T$ (often
associated with temperature) diffuses through a material with
diffusion coefficient $D$. We'll study the diffusion equation
analytically first and then we'll look at how to solve it
numerically.

\labsection{Analytic approach to the diffusion equation}

The diffusion equation can be approached analytically by
assuming that $T$ is of the form $T(x,t) = g(x)f(t)$.  If we
plug this form into the diffusion equation, we can use
separation of variables to find that $g(x)$ must satisfy
\begin{equation}\label{eq:DiffSpace}
    g''(x) + a^2 g(x) = 0
\end{equation}
where $a$ is a separation constant.  If we specify the boundary
conditions that $T(x=0,t) = 0$ and $T(x=L,t) = 0$ then the
solution to Eq.~\eqref{eq:DiffSpace} is simply $g(x) =
\sin(ax)$ and the separation constant can take on the values
$a=n \pi / L$, where $n$ is an integer.  Any initial
distribution $T(x,t=0)$ that satisfies these boundary
conditions can be composed by summing these sine functions with
different weights using Fourier series techniques.

\begin{enumerate}
\probtwo \label{P:17.1}
\begin{enumerate}
\subprob \label{P:17.1a} Find how an initial temperature
    distribution $T_n(x,0)=T_0 \sin{(n \pi x/L)}$ decays with
    time, by substituting the form $T(x,t) = T(x,0) f(t)$
    into the diffusion equation and finding $f_n(t)$ for each
    integer $n$. Do long wavelengths or short wavelengths
    decay more quickly?

\subprob \label{P:17.1b}

\marginfig[-1in]{Figures/f07p1a}{Diffusion of the Gaussian temperature
    distribution given in Problem~\ref{P:17.1b} with $\sigma =
    1$ and $D=1$.}

    Show that an initial Gaussian temperature distribution
    like this
    \begin{equation}
        T(x)=T_0 e^{-(x-L/2)^2/ \sigma^2}
    \end{equation}
    decays according to the formula
    \begin{equation}
        T(x,t)={T_0 \over \sqrt{1 + 4 D t / \sigma^2}}
        e^{-(x-L/2)^2/(\sigma^2 + 4 D t)}
    \label{diffgauss}
    \end{equation}
    by showing that this expression satisfies the diffusion
    equation Eq.~(\ref{diffusioneq}) and the initial
    condition. (It doesn't satisfy finite boundary
    conditions, however; it is zero at $\pm \infty$.) Use
    Mathematica.

%\subprob Show by using dimensional arguments that the
%    approximate time it takes for the distribution in
%    Eq.~\eqref{diffgauss} to increase its width by a distance
%    $a$ must be on the order of $t=a^2/D$. Also argue that if
%    you wait time $t$, then the distance the width should
%    increase by must be about $a=\sqrt{Dt}$. (The two
%    arguments are really identical.)
%
%\subprob Show that Eq.~\eqref{diffgauss} agrees with your
%    dimensional analysis by finding the time it takes for the
%    $t=0$ width of the Gaussian ($\sigma$) to increase to $2
%    \sigma$ (look in the exponential in
%    Eq.~(\ref{diffgauss}).)
\end{enumerate}
\end{enumerate}

\labsection{Numerical approach: a first try}

Now let's try to solve the diffusion equation numerically on a grid
as we did with the wave equation. It is similar to the wave equation
in that it requires boundary conditions at the ends of the computing
interval in $x$. But because its time derivative is only first order
we only need to know the initial distribution of $T$. This means that
the trouble with the initial distribution of $\partial T/\partial t$
that we encountered with the wave equation is avoided. But in spite
of this simplification, the diffusion equation is actually more
difficult to solve numerically than the wave equation.

If we finite difference the diffusion equation using a centered time
derivative and a centered second derivative in $x$ to obtain an
algorithm that is similar to leapfrog then we would have
\begin{eqnarray}
    {T_j^{n+1}-T_j^{n-1} \over 2 \tau} &=&
    {D \over h^2} \left( T^n_{j+1}-2T^n_j+T^n_{j-1} \right)
    \\
    T^{n+1}_j &=& T^{n-1}_j + {2 D \tau \over h^2}
    \left( T^n_{j+1}-2T^n_j+T^n_{j-1} \right)
\end{eqnarray}
There is a problem starting this algorithm because of the need to
have $T$ one time step in the past ($T^{n-1}_j$), but even if we work
around this problem this algorithm turns out to be worthless because
no matter how small a time step $\tau$ we choose, we encounter the
same kind of instability that plagues staggered leapfrog (infinite
zig-zags). Such an algorithm is called {\it unconditionally
unstable}, and is an invitation to keep looking. This must have been
a nasty surprise for the pioneers of numerical analysis who first
encountered it. It seems almost intuitively obvious that making an
algorithm more accurate is better, but in this case the increased
accuracy achieved by using a centered time derivative leads to
numerical instability.

For now, let's sacrifice second-order accuracy to obtain a stable
algorithm.  If we don't center the time derivative, but use instead a
forward difference we find
\begin{eqnarray}
    {T_j^{n+1}-T_j^n \over  \tau} &=& {D \over h^2}
    \left( T^n_{j+1}-2T^n_j+T^n_{j-1} \right)
    \label{eq:DiffuseInaccurate}
    \\
    T^{n+1}_j &=& T^n_j + { D \tau \over h^2}
    \left( T^n_{j+1}-2T^n_j+T^n_{j-1} \right)
\end{eqnarray}
You might expect this algorithm to have problems since the left side
of Eq.~(\ref{eq:DiffuseInaccurate}) is centered at time
$t_{n+\frac{1}{2}}$, but the right side is centered at time $t_n$.
This problem makes the algorithm inaccurate, but it turns out that it
is stable if $\tau$ is small enough. In the next lab we'll learn how
to get a stable algorithm with both sides of the equation centered on
the same time, but for now let's use this inaccurate (but stable)
method.\footnote{N.\ Asmar, {\it Partial Differential Equations and
Boundary Value Problems} (Prentice Hall, New Jersey, 2000), p.
412-421.}

\begin{enumerate}
\probtwo \label{P:17.2}
\begin{enumerate}

\subprob \label{P:17.2a}
\marginfig{Figures/f07p2a}{Diffusion of the $n=1$ sine
    temperature distribution given in Problem~\ref{P:17.2a}.}

    Modify one of your staggered leapfrog programs that uses
    a cell-center grid to implement this algorithm to solve
    the diffusion equation on the interval $[0,L]$ with
    initial distribution
    \begin{equation}
        T(x,0)=\sin{(\pi x/L)}
    \end{equation}
    and boundary conditions $T(0)=T(L)=0$. Use $D=2$, $L=3$,
    and $N=20$. (You don't need to make a space-time surface
    plot like Fig.~\ref{Figures/f07p2a}.  Just make a line plot
    that updates each time step as we've done previously.)
    \index{Diffusion equation!CFL condition} \index{CFL
    condition!for diffusion equation} This algorithm has a
    CFL condition on the time step $\tau$ of the form
    \begin{equation}
        \tau \le C {h^2 \over D}
    \end{equation}
    Determine the value of $C$ by numerical experimentation.

    Test the accuracy of your numerical solution by
    overlaying a graph of the exact solution found in
    \ref{P:17.1a}. Plot the numerical solution as points and
    the exact solution as a line so you can tell the
    difference.  Show that your grid solution matches the
    exact solution with increasing accuracy as the number of
    grid points $N$ is increased from 20 to 40 and then to
    80.  You can calculate the error using something like
\begin{Verbatim}
error = mean( abs( y - exact ) )
\end{Verbatim}

\subprob Get a feel for what the diffusion coefficient does
    by trying several different values for $D$ in your code.
    

\subprob Verify your answer to the question in
    problem~\ref{P:17.1a} about the decay rate of long versus
    short wavelengths by trying initial distributions of
    $T(x,0) = \sin(2\pi x/L)$, $T(x,0) = \sin(3\pi x/L)$,
    $T(x,0) = \sin(4\pi x/L)$, etc. and comparing decay
    rates.

\marginfig[-2.5in]{Figures/f07p2bi}{Diffusion of the Gaussian
    temperature distribution given in Problem~\ref{P:17.3a}
    with fixed $T$ boundary conditions.}
\marginfig[-.25in]{Figures/f07p2bii}{Diffusion of the Gaussian
temperature distribution
    given in Problem~\ref{P:17.3a} with insulating boundary conditions.}

\end{enumerate}
\end{enumerate}
\ifsolutions
\textit{Solution:}\\
\begin{codeexample}
\begin{VerbatimOut}{\listingFile}
class heat():


    def __init__(self,a,b,N,D,tau,stabilityCheck = True):
        from numpy import arange
        self.dx = (b - a)/N
        self.N = N
        self.L  = b
        self.tau = tau
        self.D = D
        print(self.dx**2/self.D)        
        self.x = arange(a - self.dx/2,b + self.dx,self.dx)
        if stabilityCheck and self.tau > self.dx**2/self.D/2:
            import sys
            print('Watch out: Numerical instability headed your way.')
            sys.exit()

    def initialWaveForm(self,ftype,freq = 2):
        from numpy import pi, sin,exp
        if ftype == 'oscillatory':
            self.T = sin(2 * pi * self.x/self.L) + sin(3 * pi * self.x/self.L) + sin(4 * pi * self.x/self.L)
        else:
            self.T = exp(-40*(self.x/self.L - 0.5)**2)
    def animate(self,tMax,bc='fixed'):
        from numpy import zeros_like,copy
        from matplotlib import pyplot
        constant = self.D * self.tau/self.dx**2
        
        TNew = zeros_like(self.T)
        counter = 0
        t = 0
        while t < tMax:
            TNew[1:self.N + 1 ] = self.T[1:self.N + 1 ] + constant * (self.T[2:self.N + 2 ] - 2 * self.T[1:self.N + 1 ] + self.T[0:self.N ])
            if bc == 'fixed':
                TNew[0] = - TNew[1] # Boundary conditions for ghost points
                TNew[-1] = - TNew[-2]

            elif bc == 'insulating':
                TNew[0] =  TNew[1] # Boundary conditions for ghost points
                TNew[-1] =  TNew[-2]
            self.T = copy(TNew)

            if counter % 50 == 0:
                pyplot.plot(self.x,self.T,'r.-')
                pyplot.ylim(-1,1)
                pyplot.draw()
                pyplot.pause(.00001)
                pyplot.clf()
            counter += 1
            t += self.tau
            #    def exact(self,x,t):
        


a = 0 
b = 3
N = 80
D = 2
tau = 0.00005


myHeat = heat(a,b,N,D,tau)
myHeat.initialWaveForm(ftype = 'exp')
myHeat.animate(10,bc = 'fixed')
\end{VerbatimOut}
\end{codeexample}
\else
\noindent\rule{5 in}{0.01 in}
\fi

Even though this technique can give us OK results, the time
step constraint for this method are pretty extreme.  The
constrain is of the form $\tau < B h^2$, where $B$ is a
constant, and this limitation scales horribly with $h$.
Suppose, for instance, that to resolve some spatial feature you
need to decrease $h$ by a factor of 5; then you will have to
decrease $\tau$ by a factor of 25. This will make your script
take forever to run, which is usually intolerable.  In the next
lab we'll learn a better way.


\labsection{Homework}
\begin{enumerate}
\prob  \label{P:17.3} Let's continue experimenting with the problem we started in
\ref{P:17.2}
\begin{enumerate}
\subprob \label{P:17.3a} Use as an initial condition a
    Gaussian distribution centered at $x=L/2$:
    \begin{equation}
        T(x,0) = e^{-40(x/L-1/2)^2}
    \end{equation}
    Use two different kinds of boundary conditions:
    \begin{enumerate}
    \item[(i)] $T=0$ at both ends and
    \item[(ii)] $\partial T /
    \partial x=0$ at both ends.
    \end{enumerate}

    Explain what these boundary conditions mean by thinking
    about a watermelon that is warmer in the middle than at
    the edge. Tell physically how you would impose both of
    these boundary conditions on the watermelon and explain
    what the temperature history of the watermelon has to do
    with your plots of $T(x)$ vs. time.

    (Notice how in case (i) the distribution that started out
    Gaussian quickly becomes very much like the $n=1$ sine
    wave. This is because all of the higher harmonics die out
    rapidly.)


\subprob \label{P:17.3b} \marginfig{Figures/f07p2c}{Diffusion of an
    initial Gaussian with $D$ as in
    Problem~\ref{P:17.3b}.} Modify your program to
    handle a diffusion coefficient which varies spatially
    like this:
    \begin{equation}
        D(x) = D_0 {x^2+L/5 \over (L/5)}
    \end{equation}
    with $D_0=2$. Note that in this case the diffusion equation is
    \begin{equation}
        {\partial T \over \partial t} = {\partial  \over \partial x}
        \left( D(x) {\partial T \over \partial x} \right)
    \end{equation}
    Use the two different boundary conditions of part (a) and
    discuss why $T(x,t)$ behaves as it does in this case.

\end{enumerate}
\end{enumerate}
\ifsolutions
\textit{Solution:}\\
\begin{codeexample}
\begin{VerbatimOut}{\listingFile}



class heat():


    def __init__(self,a,b,N,D,tau,stabilityCheck = True):
        from numpy import arange,any
        self.dx = (b - a)/N
        self.N = N
        self.L  = b
        self.tau = tau
        self.D0 = D
        self.x = arange(a - self.dx/2,b + self.dx,self.dx)
        self.D = self.diffusionConstant(self.x + self.dx/2)  # Need to evaluate D in between grid points because
                                                            # that's what comes out of the Crank-Nicholson algorithm
        if stabilityCheck and any(self.tau > self.dx**2/self.D/2):
            import sys
            print('Watch out: Numerical instability headed your way.')
            sys.exit()

    def initialWaveForm(self,ftype,freq = 2):
        from numpy import pi, sin,exp
        if ftype == 'oscillatory':
            self.T = sin(2 * pi * self.x/self.L) + sin(3 * pi * self.x/self.L) + sin(4 * pi * self.x/self.L)
        else:
            self.T = exp(-40*(self.x/self.L - 0.5)**2)

    def diffusionConstant(self,x):
        import numpy
        if type(x) is numpy.ndarray:
            from numpy import array
            return array([self.diffusionConstant(var) for var in x])
        else:
            return 5 * self.D0/self.L * (x**2 + self.L/5)

    def animate(self,tMax,bc='fixed'):
        from numpy import zeros_like,copy
        from matplotlib import pyplot
        constant = 5 * self.D0 * self.tau/(self.L * self.dx)
        constTwo = self.tau /self.dx**2
        TNew = zeros_like(self.T)
        counter = 0
        t = 0
        while t < tMax:
            TNew[1:self.N + 1 ] = self.T[1:self.N + 1 ] + constant * self.x[1:self.N + 1] * (self.T[2:self.N + 2] - self.T[0:self.N])  + constant/self.dx * (self.x[1:self.N + 1]**2 + self.L/5) *  (self.T[2:self.N + 2 ] - 2 * self.T[1:self.N + 1 ] + self.T[0:self.N ])
            #The equation below is more general and emerges when deriving the Crank-Nicholson algorithm.  The equation above arises by putting the specific D(x) into the diffusion equation and discretizing.  They should both work
            #            TNew[1:self.N + 1] = self.T[1:self.N + 1 ] + constTwo *(self.D[1:self.N + 1] *(self.T[2:self.N + 2] - self.T[1: self.N + 1]) - self.D[0:self.N] * (self.T[1:self.N + 1] - self.T[0:self.N]))
            if bc == 'fixed':
                TNew[0] = - TNew[1] # Boundary conditions for ghost points
                TNew[-1] = - TNew[-2]

            elif bc == 'insulating':
                TNew[0] =  TNew[1] # Boundary conditions for ghost points
                TNew[-1] =  TNew[-2]
            self.T = copy(TNew)

            if counter % 1000 == 0:
                pyplot.plot(self.x,self.T,'r.-')
                pyplot.ylim(-1,1)
                pyplot.draw()
                pyplot.pause(.00001)
                pyplot.clf()
            counter += 1
            t += self.tau
            #    def exact(self,x,t):
        


a = 0 
b = 3
N = 80
D = 2
tau = 1e-7


myHeat = heat(a,b,N,D,tau)
myHeat.initialWaveForm(ftype = 'exp')
myHeat.animate(10,bc = 'insulating')
\end{VerbatimOut}
\end{codeexample}
\else
\noindent\rule{5 in}{0.01 in}
\fi
