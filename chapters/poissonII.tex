\chapter{Poisson's Equation II}
\label{Lab:22}

In three dimensions, Poisson's equation is given by
\begin{equation}
    \frac{\partial^2 V}{\partial x^2} +
    \frac{\partial^2 V}{\partial y^2} +
    \frac{\partial^2 V}{\partial z^2} = -
    \frac{\rho}{\epsilon_0}
\end{equation}
You can solve this equation using the SOR technique, but we won't
make you do that here.  Instead, we'll look at several geometries
that are infinitely long in the $z$-dimension with a constant
cross-section in the $x$-$y$ plane. In these cases the $z$
derivative goes to zero, and we can use the 2-D SOR code that we
developed in the last lab to solve for the potential of a
cross-sectional area in the $x$ and $y$ dimensions.

In this lab, we'll also learn how to use information about the
potential ($V$) to calculate the electric field.  Recall that     
\begin{equation}\label{eq:EField}
        {\bf E} = - \nabla V = - \left( {\partial V\over \partial
            x}\hat{i} + {\partial V\over \partial y}\hat{j} + {\partial V\over \partial z}\hat{k}\right)
      \end{equation}

We can use finite, discrete derivatives to calculate the components of
the electric field over the entire domain.  \texttt{Numpy} also has a
\texttt{gradient} function that will perform the derivatives for you.

\marginfig{Figures/f11p1a}{The potential $V(x,y)$ from Problem~\ref{P:11.1a}.}%
%\marginfig{Figures/f11p1a2}{The surface charge density from Problem~\ref{P:11.1a}.}%


\begin{enumerate}
\probtwo \label{P:20.1} Modify your code from last time to model a
rectangular pipe with $V(-2,y) = -2$, $V(2,y) = 2$ and the $y = 0$ and
$y=L$ edges held at $V = 0$.
\begin{enumerate}
  \subprob Make a plot of $V(x,y)$.  \subprob Use
  \texttt{numpy.gradient} to calculate the electric field.  Then make
    a plot of the electric field vectors $\vec{E}(x,y)$.  See section
    9.4 in the python book for help making a vector field plot.
    
\subprob \label{P:20.1b} Now modify your boundary conditions so
    that the boundary condition at the $x=-L_x/2$ edge of the
    computation region is $\partial V/\partial x=0$ and the boundary
    condition on the $y=L_y$ edge is $\partial V/\partial y=0$. You
    can do this problem either by changing your grid and using ghost
    points or by using a quadratic extrapolation technique (see
    Eq.~\eqref{eq:QuadBoundary}). Both methods work fine, but note
    that you will need to enforce boundary conditions inside the main
    SOR loop now instead of just setting the values at the edges and
    then leaving them alone.


    You may discover that the script runs slower on this
    problem. See if you can make it run a little faster by
    experimenting with the value of $\omega$ that you use.
    Again, changing the boundary conditions can change the
    eigenvalues of the operator. (Remember that
    Eq.~\eqref{optimum} only works for cell-edge grids with
    fixed-value boundary conditions, so it only gives a
    ballpark suggestion for this problem.)

\end{enumerate}
\ifsolutions
\textit{Solution:}\\
\begin{codeexample}
\begin{VerbatimOut}{\listingFile}

class Poisson:

    def __init__(self,xDomain,yDomain,Nx,Ny):
        from numpy import linspace,meshgrid,zeros,arange

 
        self.x,self.dx = linspace(xDomain[0],xDomain[1],Nx,retstep = True)
        self.y,self.dy = linspace(yDomain[0],yDomain[1],Ny,retstep = True)
        self.X, self.Y = meshgrid(self.x,self.y)
        self.V = zeros([Nx,Ny])
        self.Nx = Nx
        self.Ny = Ny

    def setOmega(self,omega):
        self.omega = omega

    def setOptimalOmega(self):
        from numpy import cos, pi, sqrt
        R = (self.dy**2 * cos( pi/self.Nx) + self.dx**2 * cos( pi/self.Ny))/(self.dx**2 + self.dy**2)
        self.omega = 2/(1 + sqrt(1 - R**2) )
        print("Using optimal omega = {:8.4f}".format(self.omega))
    def setBoundaryConditions(self):
        from numpy import pi, sin
        # self.V[:,0] = -2
        self.V[:,-1] = 1

    def relax(self,movie = True):
        import time
        from matplotlib import pyplot,cm
        from mpl_toolkits.mplot3d import Axes3D
        constant = (2/self.dx**2 + 2/self.dy**2)


        if movie:
            fig = pyplot.figure()

        shouldContinue = True
        counter = 0
        startTime = time.time()
        while shouldContinue:
            shouldContinue = False
            # Changed bounds on loop because now I need to loop over entire domain.
            for i in range(1,self.Ny - 1):
                for j in range(1,self.Nx - 1):
                    rhs = 1/constant *  ( (self.V[i+1,j] + self.V[i-1,j])/self.dy**2 + (self.V[i,j+1] + self.V[i,j-1])/self.dx**2 )

                    error = abs(self.V[i,j] - rhs)
                    self.V[i,j] = self.omega * rhs + (1 - self.omega) * self.V[i,j]

                    if error > 1e-4:
                        shouldContinue = True
            for i in range(0,self.Nx):
                self.V[i,0] = 4/3 * self.V[i,1] - 1/3 * self.V[i,2]
            for j in range(0,self.Ny):
                self.V[self.Nx - 1,j] = 4/3 * self.V[self.Nx - 2,j] - 1/3 * self.V[self.Nx - 3,j]
            if counter % 10 ==0 and movie:
                pyplot.clf()
                ax = fig.gca(projection = '3d')
                ax.plot_surface(self.X,self.Y,self.V, cmap= cm.coolwarm)
                ax.set_zlim(0,1)
                ax.view_init(elev = 25.5,azim = 220)
                pyplot.draw()
                pyplot.pause(1e-4)
            counter += 1
        endTime = time.time()
        print('done')
        self.elapsedTime = endTime - startTime
        print(self.elapsedTime)
        pyplot.show()

    def plotEField(self):
        from numpy import gradient,transpose
        from matplotlib import pyplot,cm
        from mpl_toolkits.mplot3d import Axes3D

        pyplot.figure(1)
        Ex,Ey = gradient(transpose(self.V),self.dx,self.dy)
        pyplot.quiver(self.X,self.Y,Ex*50,Ey*50)
        pyplot.axis('equal')
        fig = pyplot.figure(2)
        ax = fig.gca(projection = '3d')
        ax.plot_surface(self.X,self.Y,self.V, cmap= cm.coolwarm)
        ax.set_zlim(0,1)
        ax.view_init(elev = 25.5,azim = 30)
        pyplot.show()

xDomain = [-2,2]
yDomain = [0,2]
Nx = 50
Ny = 50
myPoiss = Poisson(xDomain,yDomain,Nx,Ny)
#myPoiss.setOmega(1.2)
myPoiss.setOptimalOmega()
myPoiss.setBoundaryConditions()
myPoiss.relax(movie = True)
myPoiss.plotEField()

\end{VerbatimOut}
\end{codeexample}
\else
\noindent\rule{5 in}{0.01 in}
\fi


\probtwo \label{P:20.2}
\begin{enumerate}

\subprob \label{P:20.2a}
Modify your code to solve the problem of an infinitely long
hollow rectangular pipe of $x$-width 0.2~m and $y$-height
0.4~m with an infinitely long thin diagonal plate from the
lower left corner to the upper right corner. The edges of the
pipe and the diagonal plate are all grounded. There is
uniform charge density $\rho = 10^{-10}~{\rm C/m}^3$
throughout the lower triangular region and no charge density
in the upper region (see Fig.~\ref{Figures/f11p2a}). Find
$V(x,y)$ in both triangular regions. You will probably want
to have a special relation between $N_x$ and $N_y$ and use a
cell-edge grid in order to apply the diagonal boundary
condition in a simple way.

\marginfig[-2in]{Figures/f11p1b}{The potential $V(x,y)$ with
    zero-derivative boundary conditions on two sides
    (Problem~\ref{P:20.1b}.)}
\marginfig[0.25in]{Figures/f11p2a}{The potential $V(x,y)$ with constant charge density
    on a triangular region grounded at its edges
    (Problem~\ref{P:20.2a}.)}
\marginfig[1.in]{Figures/f11p2b}{The electric field from
    Problem~\ref{P:20.2b}.)}

\subprob \label{P:20.2b} Make a {\tt quiver} plot of the
    electric field at the interior points of the grid.  


\end{enumerate}
\ifsolutions
\textit{Solution:}\\
\begin{codeexample}
\begin{VerbatimOut}{\listingFile}

class Poisson:

    def __init__(self,xDomain,yDomain,Nx,Ny):
        from numpy import linspace,meshgrid,zeros,arange

        # For part (c)
        #        self.dx = (xDomain[1] - xDomain[0])/Nx
        # self.dy = (yDomain[1] - yDomain[0])/Ny
        #self.x = arange(xDomain[0] - self.dx/2, xDomain[1] + self.dx,self.dx)
        #self.y = arange(yDomain[0] - self.dy/2, yDomain[1] + self.dy,self.dy)

        self.x,self.dx = linspace(xDomain[0],xDomain[1],Nx,retstep = True)
        self.y,self.dy = linspace(yDomain[0],yDomain[1],Ny,retstep = True)
        self.X, self.Y = meshgrid(self.x,self.y)
        self.V = zeros([Ny,Nx])
        self.chargeDensity = self.rho(self.x,self.y)
        self.Nx = Nx
        self.Ny = Ny
        print(len(self.chargeDensity))
        print(len(self.chargeDensity[0]))
        print(len(self.V))
        print(len(self.V[0]))
        print(self.chargeDensity)
        print(self.X)
        print(self.X[0:self.Nx -1:2,0:self.Nx -1:2])
        print(self.Y)
        print(self.Y[0:self.Ny -1:2,0:self.Ny -1:2])

        import sys
        #        sys.exit()
        self.eps = 8.854e-12
        self.Nx = Nx
        self.Ny = Ny

    def setOmega(self,omega):
        self.omega = omega

    def setOptimalOmega(self):
        from numpy import cos, pi, sqrt
        R = (self.dy**2 * cos( pi/self.Nx) + self.dx**2 * cos( pi/self.Ny))/(self.dx**2 + self.dy**2)
        self.omega = 2/(1 + sqrt(1 - R**2) )
        print("Using optimal omega = {:8.4f}".format(self.omega))
    def setBoundaryConditions(self):
        from numpy import pi, sin
        # self.V[:,0] = -2
        # self.V[:,-1] = 1

    def rho(self,x,y):
        import numpy
        from numpy import array
        if type(x) == numpy.ndarray:
            return array([[self.rho(n,p) for n in x] for p in y])
        if y <= 2 * x:
            return 1e-10
        else:
            return 0

    def relax(self,movie = True):
        import time
        from matplotlib import pyplot,cm
        from mpl_toolkits.mplot3d import Axes3D
        constant = (2/self.dx**2 + 2/self.dy**2)


        if movie:
            fig = pyplot.figure()

        shouldContinue = True
        counter = 0
        startTime = time.time()
        while shouldContinue:
            shouldContinue = False
            # Changed bounds on loop because now I need to loop over entire domain.
            print(self.Nx, 'check here')
            for i in range(1,self.Ny - 1):
                for j in range(1,self.Nx - 1):
                    if j == 2 * i:
                        continue
                    rhs = 1/constant *  ( (self.V[i+1,j] + self.V[i-1,j])/self.dy**2 + (self.V[i,j+1] + self.V[i,j-1] )/self.dx**2 + self.chargeDensity[i,j]/self.eps)

                    error = abs(self.V[i,j] - rhs)
                    self.V[i,j] = self.omega * rhs + (1 - self.omega) * self.V[i,j]

                    if error > 1e-4:
                        shouldContinue = True
            if counter % 10 ==0 and movie:
                pyplot.clf()
                ax = fig.gca(projection = '3d')
                ax.plot_surface(self.X,self.Y,self.V, cmap= cm.coolwarm)
                #ax.pbaspect = [1.0, 4.0, 1.0]
                #pyplot.axes().set_aspect('equal')
                #ax.set_zlim(0,1)
                ax.view_init(elev = 25.5,azim = 220)
                pyplot.draw()
                pyplot.pause(1e-4)
            counter += 1
        endTime = time.time()
        print('done')
        self.elapsedTime = endTime - startTime
        print(self.elapsedTime)
        pyplot.show()

    def plotEField(self,vS = [1,1]):
        from numpy import gradient,transpose,zeros
        from matplotlib import pyplot,cm
        from mpl_toolkits.mplot3d import Axes3D

        # Do the gradient manually
        Ex = zeros((self.Ny,self.Nx))
        Ey = zeros((self.Ny,self.Nx))
        for i in range(1,self.Ny-1):
            for j in range(1,self.Nx - 1):
                Ex[i,j] = -(self.V[i,j+1] - self.V[i,j-1])/( 2.0 * self.dx)
                Ey[i,j] = -(self.V[i+1,j] - self.V[i-1,j])/( 2.0 * self.dy)

        pyplot.figure(1)
        # Use numpy.gradient to do it.
        Ey,Ex = gradient(self.V,self.y,self.x,edge_order = 2)
        pyplot.quiver(self.X[0:len(self.X):vS[0],0:len(self.X[0]):vS[1]],self.Y[0:len(self.X):vS[0],0:len(self.X[0]):vS[1]],-Ex[0:len(self.X):vS[0],0:len(self.X[0]):vS[1]],-Ey[0:len(self.X):vS[0],0:len(self.X[0]):vS[1]])
        pyplot.axis('equal')
        fig = pyplot.figure(2)
        ax = fig.gca(projection = '3d')
        ax.plot_surface(self.X,self.Y,self.V, cmap= cm.coolwarm)
        #        ax.set_zlim(0,1)
        ax.view_init(elev = 25.5,azim = 220)
        pyplot.show()

xDomain = [0,0.2]
yDomain = [0,0.4]
Nx = 100
Ny = 50
vectorSkip = [4,5]
myPoiss = Poisson(xDomain,yDomain,Nx,Ny)
#myPoiss.setOmega(1.2)

myPoiss.setOptimalOmega()
myPoiss.setBoundaryConditions()
myPoiss.relax(movie = True)
myPoiss.plotEField(vS = vectorSkip)
\end{VerbatimOut}
\end{codeexample}
\else
\noindent\rule{5 in}{0.01 in}
\fi



\labsection{Homework}
\begin{enumerate}
\prob \label{P:20.3}
\index{Electrostatic shielding} \index{Shielding, electrostatic}

Study electrostatic shielding by going back to the boundary
conditions of Problem~\ref{P:20.1} with $L_x = 0.2$ and
$L_y=0.4$, while grounding some points in the interior of
the full computation region to build an approximation to a
grounded cage. Allow some holes in your cage so you can see
how fields leak in.

You will need to be creative about how you build your cage
and about how you make SOR leave your cage points grounded
as it iterates. One thing that won't work is to let SOR
change all the potentials, then set the cage points to
$V=0$ before doing the next iteration. It is much better to
set them to zero and force SOR to never change them. An
easy way to do this is to use a cell-edge grid with a
\emph{mask}.  A mask is an array that you build that is the
same size as {\tt V}, initially defined to be full of ones
like this
\begin{Verbatim}
from numpy import ones
mask=ones([Nx,Ny]);
\end{Verbatim}
Then you go through and set the elements of {\tt mask} that you don't
want SOR to change to have a value of zero.  (We'll let you figure out
the logic to do this for the cage.)  The mask can then be used when
you update $V(i,j)$ to ensure that the value of the potential on the
cage is never modified from it's original value. There are a few
different ways to do this.  Feel free to do what makes sense to you.


The use of the mask is powerful because you
can calculate the potential for quite complicated shapes just by
changing the mask array.  Just build the mask once, before the main
loop, and the SOR algorithm can handle the rest.

\inlinefig[3in]{Figures/f11p3}{The potential $V(x,y)$ for an
electrostatic ``cage'' formed by grounding some interior
points. (Problem~\ref{P:20.3}.)}
\end{enumerate}
\end{enumerate}
\ifsolutions
\textit{Solution:}\\
\begin{codeexample}
\begin{VerbatimOut}{\listingFile}

class Poisson:

    def __init__(self,xDomain,yDomain,Nx,Ny):
        from numpy import linspace,meshgrid,zeros,arange

        
        self.x,self.dx = linspace(xDomain[0],xDomain[1],Nx,retstep = True)
        self.y,self.dy = linspace(yDomain[0],yDomain[1],Ny,retstep = True)
        self.X, self.Y = meshgrid(self.x,self.y)
        self.V = zeros([Ny,Nx])
        print(self.X)
        self.Nx = Nx
        self.Ny = Ny

    def setOmega(self,omega):
        self.omega = omega

    def setMask(self):
        from numpy import ones
        self.mask = ones([self.Ny,self.Nx])

        for i in range(self.Ny//3,2 * self.Ny//3,2):
            self.mask[i,self.Nx//3] = 0
            self.mask[i,2 * self.Nx//3] = 0
        for j in range(self.Nx//3,2 * self.Nx//3):
            self.mask[self.Ny//3,j] = 0
            self.mask[2 * self.Ny//3,j] = 0
      
    def setOptimalOmega(self):
        from numpy import cos, pi, sqrt
        R = (self.dy**2 * cos( pi/self.Nx) + self.dx**2 * cos( pi/self.Ny))/(self.dx**2 + self.dy**2)
        self.omega = 2/(1 + sqrt(1 - R**2) )
        print("Using optimal omega = {:8.4f}".format(self.omega))        
    def setBoundaryConditions(self):
        from numpy import pi, sin
        self.V[:,0] = -1
        self.V[:,-1] = 1

    def relax(self,movie = True):
        import time
        from matplotlib import pyplot,cm
        from mpl_toolkits.mplot3d import Axes3D
        constant = (2/self.dx**2 + 2/self.dy**2)

        
        if movie:
            fig = pyplot.figure()
        
        shouldContinue = True
        counter = 0
        startTime = time.time()
        while shouldContinue:
            shouldContinue = False
            for i in range(1,self.Ny - 1):
                for j in range(1,self.Nx - 1):
                    rhs = 1/constant *  ( (self.V[i+1,j] + self.V[i-1,j])/self.dy**2 + (self.V[i,j+1] + self.V[i,j-1])/self.dx**2 )

                    error = abs(self.V[i,j] - rhs)
                    self.V[i,j] = (self.omega * rhs + (1 - self.omega) * self.V[i,j]) * self.mask[i,j]

                    if error > 1e-4:
                        shouldContinue = True

            if counter % 10 ==0 and movie:
                pyplot.clf()
                ax = fig.gca(projection = '3d')
                ax.plot_surface(self.X,self.Y,self.V, cmap= cm.flag)
                ax.set_zlim(-1,1)
                ax.view_init(elev = 25.5,azim = 220)
                pyplot.draw()
                pyplot.pause(1e-4)
            counter += 1
        endTime = time.time()
        print('done')
        self.elapsedTime = endTime - startTime
        print(self.elapsedTime)
        pyplot.show()

    def plotEField(self):
        from numpy import gradient,transpose
        from matplotlib import pyplot,cm
        from mpl_toolkits.mplot3d import Axes3D

        pyplot.figure(1)
        Ex,Ey = gradient(transpose(self.V),self.dx,self.dy)
        pyplot.quiver(self.X,self.Y,Ex*50,Ey*50)
        pyplot.axis('equal')
        fig = pyplot.figure(2)
        ax = fig.gca(projection = '3d')
        ax.plot_surface(self.X,self.Y,self.V, cmap= cm.coolwarm)
        ax.set_zlim(0,1)
        ax.view_init(elev = 25.5,azim = 30)
        pyplot.show()
                
xDomain = [-0.1,0.1]
yDomain = [0,0.4]
Nx = 50
Ny = 50
myPoiss = Poisson(xDomain,yDomain,Nx,Ny)
myPoiss.setOmega(1.2)
myPoiss.setMask()
#myPoiss.setOptimalOmega()
myPoiss.setBoundaryConditions()
myPoiss.relax(movie = True)
myPoiss.plotEField()

\end{VerbatimOut}
\end{codeexample}
\else
\noindent\rule{5 in}{0.01 in}
\fi
