\chapter[Implicit Methods: the Crank-Nicolson Algorithm]{Implicit Methods: the Crank-Nicolson Algorithm}
\label{ch:cn}

So far we have solved the time-independent Schr\"{o}dinger equation
and found the bound states for several potentials.  Now we will solve
the time-dependent version of his equation.  This will allow us to
play ``movies'' of the wavefunction as it interacts with a potential. 
Consider the one-dimensional, time-dependent Schr\"{o}dinger equation
\begin{align}
-\frac{\hbar^2}{2m} \frac{d^2\psi}{dx^2} + V(x) \psi = i \hbar \frac{d\psi}{dt}\label{eq:timeDep}
\end{align}
\vspace{0.25in}
\begin{flushright}
\begin{enumerate}
\item
\prob Just as we did last time, let's rescale this
equation so it is unitless. Follow the recipe below to see how this is
done(write it out on paper).
\flushleft
\begin{enumerate}
\item Since there are two dependent variables in equation
  \eqref{eq:timeDep}, we will need to make two substitutions.  In
  equation \eqref{eq:timeDep} use the substitution $x = x_c\bar{x}$
  and $t = t_c \bar{t}$ , where $x_c$ has units of length, $t_c$ has
  units of time and $\bar{t}$ and $\bar{x}$ are dimensionless.  You
  should arrive at the following equation
\begin{align}
 \left( -\frac{\hbar^2}{2 m x_c^2} \frac{d^2\psi}{d\bar{x}^2} +
   V(x_c\bar{x}) \psi \right) = i \frac{\hbar}{t_c}\frac{d\psi}{d \bar{t}}
\end{align}
\item Now factor $\frac{\hbar}{t_c}$ (which has units of energy) out
  of each term on both sides and cancel them out.  You should arrive
  at the equation
\begin{align}
 \left( -\frac{\hbar t_c}{2 m x_c^2} \frac{d^2\psi}{d\bar{x}^2} +
   \frac{t_c}{\hbar}V(x_c\bar{x}) \psi \right) = i \frac{d\psi}{d \bar{t}}
\end{align}

You just made the entire equation unitless.  Verify that $\frac{\hbar
  t_c}{2 m x_c^2}$ is a unitless number.

\item Now let's make the left hand side as simple as possible by choosing
\begin{align}
  \frac{t_c}{\hbar} = 1 \quad\mathrm{and} \quad  \frac{\hbar t_c}{2 m x_c^2} = 1
  \label{equ:rescale}
\end{align}
Use these two expressions to find expressions for the characteristic
length ($x_c$) and the characteristic time ($t_c$).  You should find that
\begin{align}
  t_c = \hbar \quad\mathrm{and} \quad x_c = \sqrt{\frac{\hbar^2}{2 m }}
\end{align}

You just rescaled the differential equation and the final form is:

\begin{align}
 \left( - \frac{d^2\psi}{d\bar{x}^2} +
   V(x_c\bar{x}) \psi \right) = i \frac{d\psi}{d \bar{t}}
\end{align}
\end{enumerate}
\end{enumerate}
\end{flushright}
\vspace{0.25in}

\marginfig{graphics/reflection.png}{\label{fig:reflecting}
  Snapshots in time of a wave packet interacting with a step potential
  located at $x=0.55$ m.  The parameters used to generate the initial
  wave packet were: $x_0 = 0.4$, $k_0 = 500$, $\sigma^2 = .001$.  The
  height of the step potential was $V_0 = \num{3.285e6}$}

\vspace{0.25in}
Now let's start discretizing this differential equation in time
\underline{and} space.  Using a centered-difference version of the 2nd
order spatial derivative and a forward-difference version of the 1st
order time derivative, we find

\begin{align}
 \left( - \frac{\psi(\bar{x}+h,\bar{t}) - 2 \psi(\bar{x},\bar{t}) + \psi(\bar{x} - h,\bar{t})}{h^2} +
   V(x_c \bar{x}) \psi(\bar{x},\bar{t}) \right) = i \frac{\psi(\bar{x},\bar{t}+\tau) -
   \psi(\bar{x},\bar{t})}{\tau}\label{eq:finiteDiff}
\end{align}

Here $h$ is the spatial grid size and $\tau$ is the time step.  The
notation used in the equation above can get a little messy.  To
simplify it, we'll use index notation from here on.  Equation \eqref{eq:finiteDiff}
becomes:

\begin{align}
 \left( - \frac{\psi_{j+1}^n - 2 \psi_j^n + \psi_{j-1}^n}{h^2} +
   V_j \psi_j^n \right) = i \frac{\psi_j^{n+1} -
   \psi_j^n}{\tau}\label{eq:finiteDiffIndex}
\end{align}

Here $j$ is the spatial index, and $n$ is the temporal index.

\vspace{0.25in}
\begin{minipage}{0.9\linewidth}
\noindent\textbf{P1.2} Take a minute to convince yourself that
equation \eqref{eq:finiteDiffIndex} is the same thing as equation \eqref{eq:finiteDiff}
\end{minipage}
\vspace{0.25in}
 
There is actually a small problem with equation
\eqref{eq:finiteDiffIndex}.  Notice that the derivative on the right hand
side is centered at time $n + \frac{1}{2}$ but the left hand side is
centered at time $n$.  This introduces errors in the algorithm and it
would be better to have both sides centered at the same moment in
time.  

\vspace{0.25in}
\begin{minipage}{0.9\linewidth}
\noindent\textbf{P1.3} To accomplish this, let's replace every occurence of $\psi$ on
the left hand side of equation \eqref{eq:finiteDiffIndex} with the average

\begin{align}
\psi^{n + \frac{1}{2}} = \frac{\psi^{n + 1}+ \psi^n}{2}
\end{align} 

Write this substitution down and see what emerges.  You should find
that: (write it out)

\begin{align}
 \left( - \frac{(\psi_{j+1}^{n + 1}+ \psi_{j+1}^n) - 2 (\psi_j^{n + 1}+ \psi_j^n) + (\psi_{j-1}^{n + 1}+ \psi_{j-1}^n)}{2h^2} +
   V_j (\frac{\psi_j^{n+1} + \psi_j^n}{2}) \right) = i \frac{\psi_j^{n+1} -
   \psi_j^n}{\tau}
\end{align}

Look carefully at this equation.  Notice that $\psi^{n+1}$ are all
over the place.  How are we suppose to find an expression for it in
terms of the current wavefunction ($\psi_j^n$) so that we can
propagate it forward in time.  In hopes of something useful emerging,
let's accumulate everything with $n+1$ superscript on one side and
everything with an $n$ superscript on the other.  Do this by hand and
write it out on your paper.  You should find that: (write it out)


\begin{align}
\psi_{j+1}^{n+1} + \left(2i \lambda - h^2 V_j - 2\right)
\psi_j^{n+1} + \psi_{j-1}^{n+1} = - \psi_{j+1}^n + \left( 2 i \lambda
  + h^2V_j + 2\right) \psi_j^n - \psi_{j-1}^n\label{eq:CN}
\end{align}

where $\lambda = \frac{h^2}{\tau}$

\end{minipage}
\vspace{0.25in}


This equation may look bad, but it's actually not so bad.  To solve
for $\psi^{n+1}$ (a future wavefunction) all we have to do is solve a
linear system of equations.  When a system of equations must be solved
to find a future value of the function, we call it an implicit method.
This is in contrast to a method where an expression for the future
value in terms of the past value can easily be determine.  This
particular implicit method is called the Crank-Nicolson algorithm

\marginfig{graphics/transmission.png}{\label{fig:transmission}
  Snapshots in time of a wave packet interacting with a cliff
  potential located at $x=0.55$ m.  The parameters used to generate
  the initial wave packet were: $x_0 = 0.4$, $k_0 = 500$, $\sigma^2 =
  .001$.  The height of the step potential was $V_0 =
  \num{-3.285e6}$. Notice that the particle in the well is moving
  faster (less potential energy hence more kinetic) than outside.}

To see more clearly how this is a linear algebra problem, let's define
a matrix $\boldsymbol{A}$ for the left hand side of equation
\eqref{eq:CN} and a matrix $\boldsymbol{B}$ for the right hand side of
equation \eqref{eq:CN}.  Then the linear algebra problem becomes:

\begin{align}
\boldsymbol{A} \psi^{n+1} = \boldsymbol{B} \psi^n \label{eq:matrices}
\end{align} 

which can be written out explicitly as:

\[
\begin{bmatrix}
    1       & 0 & 0 & \dots & 0 & 0 \\
    1       & 2 i \lambda - h^2V_2 - 2 & 1 & \dots & 0 & 0 \\
    0 & 1       & 2 i \lambda - h^2V_3 - 2 & \dots & 0 & 0\\
    . & . & . & \dots & . & .\\
    . & . & . & \dots & . & .\\
    . & . & . & \dots & . & .\\
    0 & 0 & 0& \dots& 2 i \lambda - h^2V_{N-1} - 2 & 1 \\
    0       & 0 & 0 & \dots & 0 & 1 \\
\end{bmatrix}
\begin{bmatrix}
    \psi_1^{n+1} \\ 
    \psi_2^{n+1} \\ 
    \psi_3^{n+1} \\ 
      .\\
      .\\
      .\\
      \psi_{N-1}^{n+1}\\
      \psi_{N}^{n+1}\\
\end{bmatrix}\]

\[=
\begin{bmatrix}
    0       & 0 & 0 & \dots & 0 & 0 \\
    -1       & 2 i \lambda + h^2V_2 + 2 & -1 & \dots & 0 & 0 \\
    0       & -1 & 2 i \lambda + h^2V_3 + 2 & \dots & 0 & 0 \\
    .       & . & . & \dots & . & . \\
    .       & . & . & \dots & . & . \\
    .       & . & . & \dots & . & . \\
    0       & 0 & 0 & \dots & 2 i \lambda + h^2V_{N-1} + 2 & -1 \\
    0       & 0 & 0 & \dots & 0 & 0 \\
\end{bmatrix}
\begin{bmatrix}
    \psi_1^n \\ 
    \psi_2^n \\ 
    \psi_3^n \\ 
      .\\
      .\\
      .\\
      \psi_{N-1}^n\\
      \psi_{N}^n\\
\end{bmatrix}
\]

This is not an eigenvalue problem like we have done previously, but
rather a matrix inversion problem. In other words, $\psi^{n+1}$ can be
solved for by

\begin{align}
\psi^{n+1} = \boldsymbol{A}^{-1} \boldsymbol{B} \psi^n
\end{align}

Python (well scipy actually) has routines that can solve this problem
and we will use them today to evolve the wave function in time. Just
as in a previous assignment, the entries in the first and last rows of
matrices $\boldsymbol{A}$ and $\boldsymbol{B}$ have been chosen to
enforce a boundary condition: $\psi=0$ at the boundaries.


\marginpar{
\tt \small
cos(x) \par sin(x) \par tan(x)

acos(x) \par asin(x) \par atan(x) \par atan2(y,x)

exp(x) \par log(x) \par log10(x) \par log2(x) \par sqrt(x)

cosh(x) \par sinh(x) \par tanh(x)
}

Now we are ready to code.  I'll ask you to write some of the code
yourself and I'll give you other parts of the code. Please read
carefully:


\begin{minipage}{0.9\linewidth}
\noindent \textbf{P1.4} (i) Begin by initializing a few needed variables
\begin{enumerate}
\item Define the left boundary of the problem to be $0$.
\item Define the right boundary to be 1.
\item Define h (spatial step size) to be $.0005$
\item Define $\tau$ (temporal step size) to be $\num{5e-7}$
\item \underline{Calculate} the number of spatial grid points you'll
  have.
\item Calculate your spatial grid (linspace is a good choice). After
  you do it, print off the domain and h(the spatial step size defined
  above) and compare to ensure they agree.  Modify as needed.
\item Initialize $\lambda$: $\lambda = \frac{h^2}{\tau}$
\end{enumerate}
\vspace{0.1in}
(ii) Next you'll need to initialize a wave packet.  The initialization
is critical to giving your wave packet an appropriate initial
velocity.  One way to give your packet a positive initial velocity is
using the following function:

\begin{align}
\psi(x,t=0) = e^{-(x-x_0)^2/\sigma^2} \times e^{i k_0 x}
\end{align}

The imaginary \verb!i! here can be expressed in python as \verb!1.j!
or \verb!complex(0,1)!.  Using this function, initialize your
wavefunction using $k_0 = 500$, $x_0 = 0.4$ and $\sigma^2 = .001$. If
you choose to use the function \verb|zeros| to initialize your
wavefunction, you'll want to make sure that you specify that the type
of data in the array are complex.  Something like this ought to work:
\verb|zeros(N,dtype=complex)|.  Before moving on, plot this function
to ensure that it is what you expect. You should probably plot the
square of the wave function ($\psi^*\psi$) which can be done like this

\begin{verbatim}
plt.scatter(self.domain,self.psi * conjugate(self.psi))
\end{verbatim}            

where \verb!conjugate! comes from \verb!numpy!.

\vspace{0.1in}
(iii)  When you load matrices $\boldsymbol{A}$ and $\boldsymbol{B}$
you will have to evaluate the potential on the spatial grid points.
In preparation for that, define a function for the potential barrier.
Let's use a step potential for now:  


\[ V(x) = \left\{\begin{array}{ll}
      $\num{3.285e6}$ & x\geq $0.55$ \\
      0 & \mathrm{otherwise} 
\end{array}\right. \]

\vspace{0.1in}
(iv) Construct the matrices $\boldsymbol{A}$ and $\boldsymbol{B}$ as
defined in  equation \eqref{eq:matrices}.  This should be very similar
to what you did with bound states last time.  I suggest that you
initialize the arrays to zero and then load them with the appropriate
values with a loop. Print them off to verify that they are correct.

\vspace{0.1in}

(v) The size of these matrices, in general, will be large and solving
them is not an easy task.  Computationalists look for every possible
way to speed up a calculation like this.  In this case, notice that
both $\boldsymbol{A}$ and $\boldsymbol{B}$ are banded tri-diagonal
matrices.  In other words, all elements are zero except for along the
diagonal and one element off of the diagonal.  We can use this to our
advantage.  There are specialized algorithms that have been designed
to handle this exact problem.  In preparation for using these
functions, we have to modify matrix $\boldsymbol{A}$
slightly. Immediately after assembling these matrices, use the
following code to modify it into the needed form:

\begin{verbatim}
ud = insert(diag(A,1), 0, 0) # upper diagonal
d = diag(A) # main diagonal
ld = insert(diag(A,-1), self.N-1, 0) # lower diagonal
self.ab = matrix([ud,d,ld]) # simplified matrix
\end{verbatim}
Note: \verb!insert!, \verb!diag! and \verb!matrix! come from \verb!numpy!.
Notice that matrix ``ab'' has been saved as a member variable. This is
the matrix that I'll need when I begin time-evolving my wavefunction.
\end{minipage}

\begin{minipage}{0.9\linewidth}
\noindent(vi) Now let's perform the time evolution of the wavefunction.  This
is actually quite simple.  Equation \eqref{eq:matrices} is solved
using the function \verb!solve_banded! (comes from scipy.linalg).  This
is how you do that:

\begin{verbatim}
b = dot(self.B,self.psi)
self.psi = solve_banded((1,1),self.ab,b)
\end{verbatim}

Note \verb!dot! comes from \verb!numpy!.
The solution to this equation provides the wavefunction at a future
time. (a very short time into the future) Once we have this
wavefunction, we can simply loop over these two lines for as long as
we want to continue to get the next wave function in time.  You'll
want to enclose these two statements in a loop that runs for as long
as you want.  You may also want to include a plot inside the loop to
generate an animation.  Here's the plot that I used:

\begin{verbatim}
plt.scatter(self.domain,self.psi * conjugate(self.psi))
\end{verbatim}            

\end{minipage}

\vspace{0.25in}

\begin{minipage}{0.9\linewidth}
 \noindent \textbf{P1.5} The height of the barrier was chosen to
minimize tunneling into the barrier.  Decrease the height slowly until
you begin to observe some tunneling.
\end{minipage}

\vspace{0.25in}

\begin{minipage}{0.9\linewidth}
 \noindent \textbf{P1.6} Change the potential from a barrier potential
 to a cliff potential.
\[ V(x) = \left\{\begin{array}{ll}
      $-\num{3.285e6}$ & x\geq $0.55$ \\
      0 & \mathrm{otherwise} 
\end{array}\right. \]
and observe the time evolution
\end{minipage}
\marginfig{graphics/transmissionII.png}{\label{fig:transmission}
  Snapshots in time of a wave packet interacting with a step potential
  located at $x=0.55$ m.  The parameters used to generate the initial
  wave packet were: $x_0 = 0.4$, $k_0 = 500$, $\sigma^2 = .001$.  The
  height of the step potential was $V_0 = \num{2.85e5}$. Notice the
  tunneling of the particle into the barrier.}
      