 \chapter{Schr\"{o}dinger's Equation}
 \label{Lab:9}\index{Schr\"{o}dinger equation!time-dependent}


Here is the time-dependent Schr\"{o}dinger equation which governs the
way a quantum wave function changes with time in a one-dimensional
potential well $V(x)$:\footnote{N.\ Asmar, {\it Partial Differential
Equations and Boundary Value Problems} (Prentice Hall, New Jersey,
2000), p. 470-506.}
\begin{equation}
    i \hbar {\partial \psi \over \partial t} = -{\hbar^2 \over 2 m}
    {\partial^2 \psi \over \partial x^2} + V(x) \psi
\end{equation}
Note that except for the presence of the imaginary unit $i$, this is
very much like the diffusion equation. In fact, a good way to solve
it is with the Crank-Nicolson algorithm. Not only is this algorithm
stable for Schr\"{o}dinger's equation, but it has another important
property: it conserves probability. This is very important. If the
algorithm you use does not have this property, then as $\psi$ for a
single particle is advanced in time you have (after a while) 3/4 of a
particle, then 1/2, etc.

\marginfig[-2in]{Figures/f09p2a}{The probability density $|\psi(x)|^2$ of a
particle in a box that initially moves to the right and then
interferes with itself as it reflects from an infinite potential
(Problem~\ref{P:23.2a}).}

\marginfig{Figures/f09p2c}{The expectation position $\left< x \right>$ for
the particle in Fig.~\ref{f09p2a} as time progresses and the
packet spreads out (Problem~\ref{P:23.2c}).}

\begin{enumerate}
    \probtwo \label{P:23.1} 
\begin{enumerate}
\subprob Derive the Crank-Nicolson algorithm for Schr\"{o}dinger's
  equation.  It will probably be helpful to use the material in
  Lab~\ref{Lab:19} as a guide (beginning with Eq.~\eqref{D(x)}).
  Schr\"{o}dinger's equation is simpler in the sense that you don't
  have to worry about a spatially varying diffusion constant, but make
  sure the $V(x)\psi$ term enters the algorithm correctly. You should
  find the following:

\begin{align}
\psi_j^{n+1} \left ({i \hbar\over \tau} - {\hbar^2\over 2 m dx^2} -
  {V(x_j)\over 2} \right) + \psi_{j+1}^{n+1} {\hbar^2\over 4 m dx^2} +
\psi_{j-1}^{n+1} {\hbar^2\over 4 m dx^2} &\\
= \psi_j^n \left ({i \hbar\over \tau} + {\hbar^2\over 2 m dx^2} +
  {V(x_j)\over 2} \right) - \psi_{j+1}^n {\hbar^2\over 4 m dx^2} -
\psi_{j-1}^n {\hbar^2\over 4 m dx^2}& 
\end{align}

\end{enumerate}
\ifsolutions
\textit{Solution:}\\
\begin{align}
i \hbar{\psi_j^{n+1} - \psi_j^n\over \tau} &= - {\hbar^2\over 2 m}
{\psi^n_{j+1} - 2\psi_j^n + \psi_{j-1}^n\over dx^2} + V(x_j) \psi_j^n\\
i \hbar{\psi_j^{n+1} - \psi_j^n\over \tau} &= - {\hbar^2\over 2 m}
{{\psi_{j+1}^{n+1} + \psi_{j+1}^n\over 2} - 2 {\psi_j^{n+1} +
    \psi_j^n\over 2} + {\psi_{j-1}^{n+1} + \psi_{j-1}^n\over 2}\over
  dx^2} + V(x_j) {\psi_j^{n+1} + \psi_j^n\over 2}\\
i \hbar{\psi_j^{n+1} - \psi_j^n\over \tau} &= - 
{\hbar^2\left(\psi_{j+1}^{n+1} + \psi_{j+1}^n - 2 \psi_j^{n+1} +
    \psi_j^n + \psi_{j-1}^{n+1} + \psi_{j-1}^n\right)\over 4mdx^2} + V(x_j) {\psi_j^{n+1} + \psi_j^n\over 2}\\
\psi_j^{n+1} &\left ({i \hbar\over \tau} - {\hbar^2\over 2 m dx^2} -
  {V(x_j)\over 2} \right) + \psi_{j+1}^{n+1} {\hbar^2\over 4 m dx^2} +
\psi_{j-1}^{n+1} {\hbar^2\over 4 m dx^2} \\
= \psi_j^n &\left({i \hbar\over \tau} + {\hbar^2\over 2 m dx^2} +
  {V(x_j)\over 2} \right) - \psi_{j+1}^n {\hbar^2\over 4 m dx^2} -
\psi_{j-1}^n {\hbar^2\over 4 m dx^2} 
\end{align}

This can be written in matrix form like this:

\begin{equation}
    {\bf A} \psi^{n+1} =
    {\bf B} \psi^n
\end{equation}

The elements of ${\bf A}$ are given by
\begin{displaymath}
    A_{j,k}=0~~~{\rm except~for:~~}
\end{displaymath}
\begin{equation}
    A_{j,j-1}={\hbar^2\over 4 m dx^2} ~~~;~~~
    A_{j,j}=\left ({i \hbar\over \tau} - {\hbar^2\over 2 m dx^2} -
  {V(x_j)\over 2} \right)~~~;~~~
    A_{j,j+1}={\hbar^2\over 4 m dx^2}
    \label{eqA}
\end{equation}
and the elements of ${\bf B}$ are given by
\begin{displaymath}
    B_{j,k}=0~~~{\rm except~for:~~}
\end{displaymath}
\begin{equation}
    B_{j,j-1} = -{\hbar^2\over 4 m dx^2}  ~~~;~~~
    B_{j,j}=\left({i \hbar\over \tau} + {\hbar^2\over 2 m dx^2} +
  {V(x_j)\over 2} \right) ~~~;~~~
    B_{j,j+1}= -{\hbar^2\over 4 m dx^2}
    \label{eqB}
\end{equation}
\else
\noindent\rule{5 in}{0.01 in}
\fi
\labsection{Particle in a box}

Let's use this algorithm for solving Schr\"{o}dinger's equation
to study the evolution of a particle in a box with
\begin{equation}
\begin{array}{c}
 \\
V(x) \\
 \\
\end{array}
=
\left\{
\begin{array}{ll}
0 & {\rm if~~~} -L < x < L \\
 & \\
+\infty & {\rm otherwise} \\
\end{array}
\right.
\end{equation}
The infinite potential at the box edges is imposed by forcing the
wave function to be zero at these points: \index{Particle in a
box}
\begin{equation}
    \psi(-L)=0~~~~~;~~~~~\psi(L)=0
\end{equation}

\begin{enumerate}
\probtwo \label{P:23.2}
\begin{enumerate}
\subprob \label{P:23.2a} 
    Write a script to solve the
    time-dependent Schr\"{o}dinger equation using
    Crank-Nicolson as discussed in Lab~\ref{Lab:8}. Use
    natural units in which $\hbar=m=1$ and $L=10$.  We find
    that a \textbf{cell-edge grid works best}, but you can also do cell-center with ghost
    points if you like. Start with a localized wave packet of
    width $\sigma$ and momentum $p$:
    \begin{equation}
        \psi(x,0) = {1 \over \sqrt{ \sigma \sqrt{\pi}}} e^{i p x / \hbar}
        e^{-x^2/(2 \sigma^2)}
    \end{equation}
    with $p=2 \pi$ and $\sigma=2$. This initial condition
    does not exactly satisfy the boundary conditions, but it
    is very close. 

   Run the script with $N=200$ and watch the particle (wave
    packet) bounce back and forth in the well. Plot an animation of
    $\psi^* \psi$ so that you can visualize the probability
    distribution of the particle.  Try switching the sign of
    $p$ and see what happens.



\marginfig[-3.2in]{Figures/f09p2d}{The probability density
$|\psi(x)|^2$ of a particle that is initially more localized
quickly spreads (Problem~\ref{P:23.2d}).}

\marginfig{Figures/f09p2d2}{The expectation position of the particle
in Fig.~\ref{f09p2d} as time progresses.}

\subprob Verify by doing a numerical integral that
$\psi(x,0)$ in the formula given above is properly
normalized. Then run the script and check that it stays
properly normalized, even though the wave function is
bouncing and spreading within the well. 

\subprob \label{P:23.2c}
Run the script and verify by numerical integration that
the expectation value of the particle position
\begin{equation}
    \langle x\rangle = \int_{-L}^{L}  \psi^*(x,t)~ x~ \psi(x,t) dx
\end{equation}
is correct for a bouncing particle. Plot $\langle x \rangle(t)$ to
see the bouncing behavior. Run long enough that the wave packet
spreading modifies the bouncing to something more like a harmonic
oscillator. (Note: you will only see bouncing-particle behavior
until the wave packet spreads enough to start filling the entire
well.)

\subprob \label{P:23.2d}
You may be annoyed that the particle spreads out so much in time.
Try to fix this problem by narrowing the wave packet (decrease the
value of $\sigma$) so the particle is more localized. Run the script
and explain what you see in terms of quantum mechanics.


%  (e)
%  You probably noticed that the expectation value of $x$
%  damps away to zero, suggesting that the particle
%  is sitting still and
%  no longer has any energy.
%  This is false.
%  Calculate the expectation value of the energy\begin{equation}
%  \langle E\rangle = \int_{-L}^{L}  \psi^*(x,t)~
%  [-{\hbar^2 \over 2m} {\partial^2 \psi \over \partial x^2} + V(x) \psi(x,t)] dx
%  \end{equation}
%  and plot it vs. time to verify that energy is conserved.
%  This calculation will work better if you do an integration by parts
%  to remove the second derivative in the integral.


\end{enumerate}
\end{enumerate}

\ifsolutions
\textit{Solution:}\\
\begin{codeexample}
\begin{VerbatimOut}{\listingFile}



class CN():

    def __init__(self,b,N,tau):
        from numpy import linspace
        self.L = b
        self.N = N
        self.tau = tau

        self.dx = 2 * b/N
        # cell-edge grid.
        self.x,self.dx = linspace(-self.L, self.L, N,retstep = True)
        print(len(self.x))

    def V(self,x):

        if -self.L <= x <= self.L:
            return 0
        else:
            return 10000
        
    def initializeWaveFunction(self,sigma):
        from numpy import sin, pi,sqrt,pi,exp,conjugate,real
        from matplotlib import pyplot
        self.psi = 1/sqrt(sigma * sqrt(pi)) * exp(2j * pi * self.x) * exp(-self.x**2/(2 * sigma**2) )
        print(sum(real(self.psi * conjugate(self.psi))) * self.dx )
        
    def loadMatrices(self):
        from numpy import zeros, insert,diag,matrix

        A = zeros([self.N , self.N ],dtype = complex)
        self.B = zeros([self.N, self.N ],dtype = complex)

        for i in range(1,self.N - 1):
            A[i,i-1] = 1/(4 * self.dx**2)
            A[i,i] = 1j/self.tau - 1/(4 * self.dx**2) - self.V(self.x[i])/2
            A[i,i+1] = 1/(4 * self.dx**2)

            self.B[i,i-1] = -1/(4 * self.dx**2)
            self.B[i,i] = 1j/self.tau + 1/(4 * self.dx**2) + self.V(self.x[i])/2
            self.B[i,i+1] = -1/(4 * self.dx**2)

        A[0,0] = 1.
        A[-1,-1] = 1.


        ud = insert(diag(A,1), 0, 0) # upper diagonal
        d = diag(A) # main diagonal
        ld = insert(diag(A,-1), self.N - 1, 0) # lower diagonal
        # simplified matrix
        self.ab = matrix([ud,d,ld])
    def animate(self,tMax):
        from numpy import dot,linspace,conjugate,real
        from scipy.linalg import solve_banded
        from matplotlib import pyplot
        counter = 0
        t = 0
        while t < tMax:
            if counter %5 == 0:
                normalize = sum(real(self.psi * conjugate(self.psi))) * self.dx
                
                pyplot.plot(self.x,1/normalize * conjugate(self.psi) * self.psi,'r.-')
                pyplot.ylim(-0.1,1)
                pyplot.draw()
                pyplot.pause(1e-1)
                pyplot.clf()
            b = dot(self.B,self.psi)
            self.psi = solve_banded((1,1),self.ab,b)

            counter += 1
            t += self.tau
        pyplot.figure(2)

b = 10
N = 200
tau = 5e-3
sigma = 2
tMax = 10
mySchrodinger = CN(b,N,tau)
mySchrodinger.initializeWaveFunction(sigma)
mySchrodinger.loadMatrices()
mySchrodinger.animate(tMax)


\end{VerbatimOut}
\end{codeexample}
\else
\noindent\rule{5 in}{0.01 in}
\fi


\labsection{Tunneling}

Now we will allow the pulse to collide with a non-infinite barrier and
study what happens. Classically, the answer is simple: if the particle
has a kinetic energy less than $V_0$ it will be unable to get over the
barrier, but if its kinetic energy is greater than $V_0$ it will slow
down as it passes over the barrier, then resume its normal speed in
the region beyond the barrier. (Think of a ball rolling over a bump on
a frictionless track.) Can the particle get past a barrier that is
higher than its kinetic energy in quantum mechanics? The answer is
yes, and this effect is called tunneling.

\marginfig{Figures/f09p3b}{The probability distribution $|\psi(x)|^2$
for a particle incident on a narrow potential barrier
located just before $x=10$ with $V_0 > \left< E \right>$. Part
of the wave tunnels through the barrier and part interferes
with itself as it is reflected.}

To see how the classical picture is modified in quantum
mechanics we must first compute the energy of our pulse so we
can compare it to the height of the barrier. The quantum
mechanical formula for the expectation value of the energy is
\begin{equation}
    \langle E \rangle = \int_{-\infty}^{\infty} \psi^* H \psi dx
\end{equation}
where $\psi^*$ is the complex conjugate of $\psi$ and where
\begin{equation}
    H \psi = -{\hbar^2 \over 2m} {\partial^2 \psi \over \partial x^2} + V(x) \psi(x)
\end{equation}
In our case $\psi(x)$ is essentially zero at the location of the
potential barrier, so we may take $V(x)=0$ in the integral.

\begin{enumerate}
\probtwo \label{P:23.3}
    Use Mathematica to compute $\langle E \rangle$ for your
    wave packet. You should find that
    \begin{equation}
        \langle E \rangle = {p^2 \over 2 m} + {\hbar^2 \over 4 m \sigma^2} \approx 20
    \end{equation}
    Since this is a conservative system, the energy remains
    constant and you can just use the initial wave function
    in the integral.

    HINT: You will need to use the {\tt Assuming} command to
    specify that certain quantities are real and/or positive
    to get Mathematica to do the integral.

\end{enumerate}
\ifsolutions
\textit{Solution:}\\
\begin{codeexample}
\begin{VerbatimOut}{\listingFile}
# Mathematica Code!

Psi[x_] := 1/Sqrt[sigma * Sqrt[Pi]]  * Exp[I p * x/hbar]*  Exp[-x^2/(2. * sigma^2)]
Integrate[-Conjugate[Psi[x]] * hbar^2/(2 m) *   Psi''[x], {x,
  -\[Infinity], \[Infinity]}, 
Assumptions -> {hbar \[Element] Reals, p \[Element] Reals,     
m \[Element] Reals, m > 0, hbar > 0, sigma \[Element] Reals, sigma >
0}] 
/. {hbar -> 1, m -> 1, p -> 2 Pi, sigma -> 2}
\end{VerbatimOut}
\end{codeexample}
\else
\noindent\rule{5 in}{0.01 in}
\fi

\begin{enumerate}
\prob \label{H:23.4}
\begin{enumerate}

\subprob    Modify your script from Problem~\ref{P:23.2} so that
    it uses a computing region that goes from $-2L$ to
    $2L$ and a potential
    \begin{equation}
    \begin{array}{c}
     \\
    V(x) \\
     \\
    \end{array}
    =
    \left\{
    \begin{array}{ll}
    0 & {\rm if~~~} -2L < x < 0.98L \\
    V_0 & {\rm if~~~} 0.98L \le x \le L \\
    0 & {\rm if~~~} L < x < 2L \\
     & \\
    +\infty & {\rm otherwise} \\
    \end{array}
    \right.
    \end{equation}
    so that we have a square potential hill $V(x)=V_0$
    between $x= 0.98L$ and $x=L$ and $V=0$
    everywhere else in the well.
    \index{Schr\"{o}dinger equation!potential barrier}
    \index{Potential barrier!Schr\"{o}dinger equation}

    Note: Since $V(x)$ was just zero in the last problem, this is the
    first time to check if your $V(x)$ terms in Crank-Nicolson are
    right.  If you see strange behavior, you might want to look at
    these terms in your code.

\subprob    Run your script several times, varying the height
    $V_0$ from less than your pulse energy to more than
    your pulse energy. Overlay a plot of $V(x)/V_0$ on
    your plot of $|\psi|^2$ and look at the behavior of
    $\psi$ in the barrier region.  

    \subprob Vary the width of the hill with $V_0$ bigger than your
    pulse energy to see how the barrier width affects tunneling.
\end{enumerate}
\end{enumerate}
\end{enumerate}
\ifsolutions
\textit{Solution:}\\
\begin{codeexample}
\begin{VerbatimOut}{\listingFile}


class CN():

    def __init__(self,b,N,tau,V0):
        from numpy import linspace
        self.L = b
        self.N = N
        self.tau = tau
        self.V0 = V0
        self.dx = 4 * b/N
        # cell-edge grid.
        self.x,self.dx = linspace(-2 * self.L, 2* self.L, N,retstep = True)
        print(len(self.x))

    def V(self,x):
        from numpy import ndarray,array
        if isinstance(x,ndarray):
            return array([self.V(i) for i in x])
        if -2 * self.L < x <= 0.98 * self.L:
            return 0
        elif 0.98 * self.L <= x <= self.L:
            return self.V0
        elif self.L < x < 2 * self.L:
            return 0
        else:
            return 10000
        
    def initializeWaveFunction(self,sigma):
        from numpy import sin, pi,sqrt,pi,exp,conjugate,real
        from matplotlib import pyplot
        self.psi = 1/sqrt(sigma * sqrt(pi)) * exp(2j * pi * self.x) * exp(-self.x**2/(2 * sigma**2) )
        print(sum(real(self.psi * conjugate(self.psi))) * self.dx )
        
    def loadMatrices(self):
        from numpy import zeros, insert,diag,matrix

        A = zeros([self.N , self.N ],dtype = complex)
        self.B = zeros([self.N, self.N ],dtype = complex)

        for i in range(1,self.N - 1):
            A[i,i-1] = 1/(4 * self.dx**2)
            A[i,i] = 1j/self.tau - 1/(4 * self.dx**2) - self.V(self.x[i])/2
            A[i,i+1] = 1/(4 * self.dx**2)

            self.B[i,i-1] = -1/(4 * self.dx**2)
            self.B[i,i] = 1j/self.tau + 1/(4 * self.dx**2) + self.V(self.x[i])/2
            self.B[i,i+1] = -1/(4 * self.dx**2)

        A[0,0] = 1.
        A[-1,-1] = 1.


        ud = insert(diag(A,1), 0, 0) # upper diagonal
        d = diag(A) # main diagonal
        ld = insert(diag(A,-1), self.N - 1, 0) # lower diagonal
        # simplified matrix
        self.ab = matrix([ud,d,ld])
    def animate(self,tMax):
        from numpy import dot,linspace,conjugate,real
        from scipy.linalg import solve_banded
        from matplotlib import pyplot
        counter = 0
        t = 0
        while t < tMax:
            if counter %5 == 0:
                normalize = sum(real(self.psi * conjugate(self.psi))) * self.dx
                
                pyplot.plot(self.x,1/normalize * conjugate(self.psi) * self.psi,'r.-')
                pyplot.plot(self.x,self.V(self.x)/self.V0,'r-')
                pyplot.ylim(-0.1,1.1)
                pyplot.xlim(-2 * self.L, 2 * self.L)
                pyplot.draw()
                pyplot.pause(1e-1)
                pyplot.clf()
            b = dot(self.B,self.psi)
            self.psi = solve_banded((1,1),self.ab,b)

            counter += 1
            t += self.tau
        pyplot.figure(2)

b = 10
N = 200
tau = 5e-3
sigma = 2
tMax = 10
barrierHeight = 10
mySchrodinger = CN(b,N,tau,barrierHeight)
mySchrodinger.initializeWaveFunction(sigma)
mySchrodinger.loadMatrices()
mySchrodinger.animate(tMax)



\end{VerbatimOut}
\end{codeexample}
\else
\noindent\rule{5 in}{0.01 in}
\fi
