\chapter[Overview]{Overview}
\label{ch:overview}
%\addcontentsline{toc}{chapter}{Grids and Derivatives}

Most of your experience with math and physics to this point in your
education has involved the use of analytical techniques.  In other
words, you use your pencil and paper (and occasionally Mathematica) to
perform algebra and calculus until you arrive at the solution, a
solution that you can write down on paper.  It turns out that there
are a whole bunch of real-world problems that have no analytical
solution: you can't write down the solution on a piece of paper.  In
these cases, we must use a numerical approach.  What that means
exactly is probably not very clear at this point, but it will soon.

More specifically, this class will focus on numerical solutions to
differential equations.  It seems prudent to begin our study with a
broad overview of differential equations.  What are they?  What are
some common ones that arise in physics?  Can we categorize them into
groups?  

\labsection{What is a differential equation?}
So what is a differential equation? When we say the word
equation, we probably think of something like this
\begin{equation}\label{eq:alg}
5 x^2 + 3x - 2 = 0.
\end{equation}
``Solving'' this equation entails performing some algebra until $x$ is
isolated  and the solution is then found on the other side of the
equals sign.  In this case, the algebra would lead to the following solution
\begin{equation}
x = -1 \mathrm{~~ or~~} x = {2 \over 5}
\end{equation}
Notice that the solution to this equation was a \textit{number}.  Now
let's look at a differential equation:
\begin{equation}\label{eq:diff}
m\frac{d^2y(t)}{dt^2} + c \frac{dy(t)}{dt} + k y(t) = 0  
\end{equation}
This equation emerges when you write down Newton's second law for a
damped oscillator. Unlike the solution to equation \eqref{eq:alg}, 
the solution to this equation is not a number, it's a function.  It
turns out that one solution\sidenote{This is the solution to the
  underdamped case where ${c\over 2 m} \ll \sqrt{k \over m}$} to this differential equation is
\begin{equation}
y(t) = A e^{-\frac{c}{2m}t} \sin(\omega t + \phi) 
\end{equation}
where
\begin{equation}
\omega = \sqrt{ {k\over m} - {c^2\over 4 m^2} }
\end{equation}
\begin{enumerate}
\probtwo Verify, using Mathematica (or by hand), that this equation really does
satisify the differential equation in eq \eqref{eq:diff}.
\end{enumerate}

\labsection{How do we categorize differential equations?}
Differential equations are categorized based on the highest derivative
found in the equation and on the number of independent variable that
the solution function has.  For example, notice that the solution to equation
\eqref{eq:diff} was a function of {\it one } variable: $t$.
Differential equations whose solutions are functions of one and only
one variable are called \textbf{ordinary differential equations}.
They are ordinary because the differential equation only has ordinary
derivatives and no partial derivatives.

If the solution function has multiple dependent variables, then the
differential equation could involved derivatives with respect to each
of them.  We call these partial derivatives and therefore the
differential equation is called a partial differential equation
because.  Here is an example of a partial differential equation
\begin{equation}\label{eq:partial}
\frac{\partial^2u}{\partial t^2} - c^2 \frac{\partial^2 u}{\partial x^2} = 0  
\end{equation}
Notice the ordinary derivative ($\frac{d}{dt}$) in equation
\eqref{eq:diff} replaced with the partial derivative symbol
($\frac{\partial^2}{\partial t^2}$) in equation \eqref{eq:partial}.
The solution to this equation is a function of two variables:
$u(x,t)$.

\labsection{Boundary and Initial Conditions}
It turns out that the differential equation alone is not sufficient to fully determine
the solution. To see why this is so, let's take a very simple differential equation:
\begin{equation}
\frac{dy}{dt} = -\frac{k}{y}
\end{equation}
This equation can be solved by rearranging so that all $y$s and $dy$s
are on one side and all $t$s and $dt$s are on the other:
\begin{equation}
y dy = - c dt
\end{equation}
Now integrating both sides:
\begin{align}
\int y dy &= - \int k dt\\
\frac{y^2}{2} +C_1 &= - k t + C_2\\
\frac{y^2}{2} &= - k t + C_3\\
y^2 &= - 2 k t + 2 C_3\\
y &= \sqrt{- 2 k t + C_4}\\
\end{align}
The integrating constant $C_4$ must be determined to finish the
solution.  This constant is the initial value of the function, or the
value of the function at the initial time ($t=0$s ).  Without this
information, the solution is not complete.  In general, the number of
extra pieces of information needed to solve a differential equation is
equal to the order of the differential equation.  For example, to
solve the following second-order ordinary differential equation
\begin{equation}\label{eq:projectile}
\frac{d^2 y}{dt^2} = - g
\end{equation}
would require two extra pieces of information, usually the initial
position and initial velocity (derivative of position), would be
needed to solve the equation.

There are two types on ``extra'' information that are typically used:
initial conditions and boundary conditions.  

\labsection{A second-order differential equation as two first-orders}
Let's consider a familiar differential equation
corresponding to a one-dimensional projectile experiencing a constant acceleration:

\begin{equation}\label{eq:projectileDEQ}
\frac{d^2y}{dt^2} = -g
\end{equation}
 We call it a second order equation because the
highest derivative found is of second order.  A second order
differential equation can always be written as two first order
equations by using an intermediate variable.  For example by defining $v$:

\begin{equation}
v = \frac{dy}{dt}
\end{equation}
we can now write equation
\eqref{eq:projectileDEQ} as:

\begin{equation}
v = \frac{dy}{dt}\qquad \textrm{and} \qquad -g = \frac{dv}{dt}
\end{equation}
which is a set of \textit{coupled, first-order} differential equations.