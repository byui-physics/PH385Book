\chapter{Lattice Vibrations }
\label{Lab:13} \index{Hanging chain}

Hopefully you're beginning to see that normal modes of vibration show
up everywhere, and it's worth your time to understand the numerical
techniques we've studied to solve these kinds of problems. As you
might already know, the atoms that make up a solid vibrate about their
equilibrium locations.  The collective motion of all atomic vibrations
creates waves, sometimes traveling waves and sometimes standing
(stationary) waves.  These waves determine many of the critical
properties of a material, including: i)heat capaticy, ii) thermal
conductivity, iii) electrical conductivity, iv) speed of sound and
more.  In this lab we will study a simplified version of real lattice
vibrations: a one-dimensional chain of particles connect via ideal
springs.



\labsection{Coupled Oscillators}

\marginfig{Figures/massesSprings}{\label{fig:masses}A collection of masses connected by springs.}

Consider the situation depicted in figure \ref{fig:masses}. All of
the masses are equal and all of the springs that connect the masses are
identical. (We'll change that soon!)  Each mass experiences two
forces: one from the spring on the right and one from the spring on
the left.  We'll assume that both spring forces are given by Hooke's
law:

\begin{equation}
F = - k u
\end{equation}
where $u$ is the distance away from equilibrium.
\begin{enumerate}
\probtwo \label{P:13.1}
\begin{enumerate}
\subprob
 Write down Newton's second law for all of the masses in the figure.
Once you are done, compare to the equations below and fix any
differences between what you wrote and the correct solution. (Pay
close attention to all of the signs in the equations)
\begin{align}
    -k u_1 + k (u_2 - u_1) &= m {d^2u_1\over dt^2}\label{P13.1astart}\\
    -k (u_2 - u_1) + k (u_3 - u_2) &= m {d^2u_2\over dt^2}\\
    -k (u_3 - u_2) + k (u_4 - u_3) &= m {d^2u_3\over dt^2}\\
    -k u_4 + k (u_3 - u_4) &= m {d^2u_4\over dt^2}\label{P13.1aend}
\end{align}
\subprob Now let's assume that the motion of each mass is harmonic.
In other words, that it's displacement is given by:

\begin{equation}
u_j = x_j e^{-i \omega t}
\end{equation}
where $u_j$ is the displacment of particle $j$ and $x_j$ is the
amplitude of the motion. These are standing wave solutions, not
traveling wave. Plug this expression in for all of the $u$'s
in equation \ref{P13.1astart} - \ref{P13.1aend} and reduce so that all constants appear on
the right hand sides.  You should arrive at the following set of
equations:
\begin{align}\label{P13.1b}
     x_1 -  (x_2 - x_1) &= {m\omega^2\over k} x_1\\
     (x_2 - x_1) -  (x_3 - x_2) &= {m\omega^2\over k} x_2\\
     (x_3 - x_2) -  (x_4 - x_3) &= {m\omega^2\over k} x_3\\
     x_4 -  (x_3 - x_4) &= {m\omega^2\over k} x_4\\
\end{align}

\subprob  Hopefully you are starting to see an
eigenvalue problem emerge from this math.  In other words, can you see
that the set of equations above can be represented as a linear algebra
problem of this form:

\begin{equation}
\mathbf{A} \vec{x} = \lambda \vec{x}
\end{equation}

where the vector of amplitudes $\vec{x}$ are the eigenvectors and ${m
  \omega^2\over k}$ are the eigenvalues.    Look at the matrix below
and see if you can determine what the entries with question marks
should be?

\begin{equation}
\left[
\begin{array}{cccccccc}
? & ? & 0 & 0\\
? & ?  & ? & 0 \\
0 & ? & ?  & ?\\
0 & 0 & ? & ?\\
\end{array}
\right]
\left[
\begin{array}{r}
x_1 \\
x_2 \\
x_3  \\
x_4   \\
\end{array}
\right]
= \lambda
\left[
\begin{array}{r}
x_1   \\
x_2 \\
x_3 \\
x_4  \\
\end{array}
\right]
\end{equation}


You should find that the entries in the matrix look like this:

\begin{equation}
\left[
\begin{array}{cccccccc}
2 & -1 & 0 & 0\\
-1 & 2  & -1 & 0 \\
0 & -1 & 2  & -1\\
0 & 0 & -1 & 2\\
\end{array}
\right]
\left[
\begin{array}{r}
x_1 \\
x_2 \\
x_3  \\
x_4   \\
\end{array}
\right]
= \lambda
\left[
\begin{array}{r}
x_1   \\
x_2 \\
x_3 \\
x_4  \\
\end{array}
\right]
\end{equation}

Notice that since we don't have boundary conditions to consider, we
don't need to write this equation as a generalized eigenvalue problem,
just a normal eigenvalue problem.

\subprob  By noticing the pattern in the middle rows of matrix
$\textbf{A}$, build a code to solve \textbf{the more general problem} involving any
number of equal masses (not restricted to 4) connected by identical springs.  

\subprob Instead of plotting the eigenvectors like we've done in the
past, make a movie of the masses as they vibrate.  Remember, making a
movie in python is little more than repeatedly plotting on the same
canvas and then deleting the previous plot before you proceed to the
next one.

\subprob Is the wavelength of the standing wave different for the
different modes or does it stay the same for all?
\end{enumerate}
\end{enumerate}

\ifsolutions
\textit{Solution:}\\
\begin{codeexample}
\begin{VerbatimOut}{\listingFile}


class Phonon():

    def __init__(self,m,k,N):
        from numpy import arange,linspace
        self.m = m
        self.k = k
        self.N = N
        #        self.dx = (b - a)/N
        #self.x = linspace(a - self.dx/2., b + self.dx/2., N + 2)
        #        self.x = arange(a - self.dx/2.,b + self.dx,self.dx)
        # self.g = 9.8

    def loadMatrices(self):
        from numpy import zeros,eye
        
        self.A = zeros([self.N,self.N])
        # Fill in top and bottom rows of A later
        self.A[-1,-1] = 2.
        self.A[-1,-2] = -1.
        self.A[0,0] = 2
        self.A[0,1] = -1.

        #        self.A[0,0] = -1./self.dx
        #self.A[0,1] = 1./self.dx
        
        
        for i in range(1,self.N - 1):
            self.A[i,i] = 2.
            self.A[i,i-1] = -1.
            self.A[i,i+1] =  -1.

        

    def solveProblem(self):
        from scipy.linalg import eig
        from numpy import pi, sqrt
        
        self.lam,self.u = eig(self.A)
        self.omega = sqrt(self.lam * self.k/self.m) /(2 * pi)
        self.key = sorted(range(self.N), key=lambda k: self.omega[k])

    def animate(self,mode):
        from matplotlib import pyplot
        from numpy import real,arange,cos,pi,linspace
        spacing = 2.5
        equilibriumPos = linspace(spacing,(self.N + 1) * spacing,self.N)
        print(equilibriumPos, 'here')
        #  import sys
        #sys.exit()
        counter = 0
        for t in arange(0,(2 * pi)/ real(self.omega[self.key[mode]]) * 10,.5):
            if counter %10 == 0:
                pyplot.plot(equilibriumPos + self.u[:,self.key[mode]] * cos(self.omega[self.key[mode]] * t),[0.5 for t in equilibriumPos],'r.',markersize = 5)
                pyplot.xlim(-1,(self.N + 2) * spacing)
                pyplot.draw()
                pyplot.pause(.0001)
                pyplot.clf()
            counter += 1

m = 1
k = 1
N = 10
myPh = Phonon(m,k,N)
myPh.loadMatrices()
myPh.solveProblem()
print(myPh.omega)
myPh.animate(3)




\end{VerbatimOut}
\end{codeexample}
\else
\noindent\rule{5 in}{0.01 in}
\fi

Clearly what we have studied today is a simplification of a real
solid.  In real solids:
\begin{enumerate}
\item the atoms vibrate in all three dimensions.
\item the forces between neighboring atoms may not be
  well-approximated by Hooke's Law.
\item Non-negligible forces from atoms located further away than
  nearest neighbor may exist making the forcing function more complicated.
\end{enumerate}

Despite the simplifications that we have made here, hopefully you are
beginning to understand the basics of lattice vibrations.
\labsection{Homework}

\begin{enumerate}
\prob  (\textbf{\LaTeX~ problem}) Now let's consider a system of masses where every other mass is
different ($m_1$ and $m_2$) and every other spring is different  ($k_1$
and $k_2$).  You choose the values
of the masses and the spring constants, and then convince yourself
that your results make sense.  Make an animation of the 3rd
and 5th normal mode and verify that it looks correct. (Note: You'll need to
rework the problem from the beginning to see how things change when
the springs and masses are different.  Don't try to modify your code
without working out the details on paper (or whiteboard) first.)
\end{enumerate}


\ifsolutions
\textit{Solution:}\\
\begin{codeexample}
\begin{VerbatimOut}{\listingFile}


class Phonon():

    def __init__(self,m1,m2,k1,k2,N):
        from numpy import arange,linspace
        self.m = [m1,m2]
        self.k = [k1,k2]
        self.N = N
        #        self.dx = (b - a)/N
        #self.x = linspace(a - self.dx/2., b + self.dx/2., N + 2)
        #        self.x = arange(a - self.dx/2.,b + self.dx,self.dx)
        # self.g = 9.8

    def loadMatrices(self):
        from numpy import zeros,eye

        self.A = zeros([self.N,self.N])
        # Fill in top and bottom rows of A later
        self.A[-1,-1] = (self.k[0] + self.k[1])/self.m[(self.N-1) % 2]
        self.A[-1,-2] = -self.k[(self.N - 1) % 2]/self.m[(self.N - 1) % 2]
            
        self.A[0,0] = (self.k[0] + self.k[1])/self.m[0]
        self.A[0,1] = -self.k[1]/self.m[0]

        #        self.A[0,0] = -1./self.dx
        #self.A[0,1] = 1./self.dx


        for i in range(1,self.N - 1):
            self.A[i,i] = (self.k[0] + self.k[1])/self.m[ i%2 ]
            self.A[i,i-1] = -self.k[i%2]/self.m[i%2]
            self.A[i,i+1] =  -self.k[(i-1)%2 ]/self.m[i%2]
 


    def solveProblem(self):
        from scipy.linalg import eig
        from numpy import pi, sqrt

        self.lam,self.u = eig(self.A)
        self.omega = sqrt(self.lam) 
        self.key = sorted(range(self.N), key=lambda k: self.omega[k])
        self.omega = sorted(self.omega)
        
    def animate(self,mode):
        from matplotlib import pyplot
        from numpy import real,arange,cos,pi,linspace
        spacing = 2.5
        equilibriumPos = linspace(spacing,(self.N + 1) * spacing,self.N)
        print(equilibriumPos, 'here')
        #  import sys
        #sys.exit()
        counter = 0
        period = (2 * pi)/ real(self.omega[self.key[mode-1]])
        for t in arange(0,period * 1000,period/10):
            if counter %10 == 0:
                pyplot.plot(equilibriumPos + self.u[:,self.key[mode-1]] * cos(self.omega[mode-1] * t),[0.5 for t in equilibriumPos],'r.',markersize = 25)
                pyplot.xlim(-1,(self.N + 2) * spacing)
                pyplot.draw()
                pyplot.pause(.001)
                pyplot.clf()
            counter += 1 

m1 = 1
m2 = 10
k1 = 1
k2 = 10
N = 10
myPh = Phonon(m1,m2,k1,k2,N)
myPh.loadMatrices()
myPh.solveProblem()
print(myPh.omega)
myPh.animate(3)


\end{VerbatimOut}
\end{codeexample}
\else
\noindent\rule{5 in}{0.01 in}
\fi


