\chapter[Discrete Grids]{Discrete Grids}
\label{ch:grids}
%\addcontentsline{toc}{chapter}{Grids and Derivatives}
\begin{center}
\textbf{Python skills that you will need for today:\\
    Plotting (Chapter 7), numpy array construction (section 5.3)}
\end{center}

Analytical solutions to differential equations yield analytic
functions; functions that can be evaluated at arbitrary
points. Transitioning from analytical solutions to differential
equations to numerical solutions requires a shift in mentality.  When
you solve a differential equation numerically, the solution function
is represented as an array of function values on a discrete grid.  It
is important now to develop the intuition and skills for representing
a function as a discrete set of values on a grid.  Before we proceed to
solving differential equations, let's spend some time getting
comfortable working with spatial grids.

\marginfig[-1in]{Figures/f01Grids}{\label{f01Grids}Three common spatial grids}

\labsection{Spatial grids}

Figure~\ref{f01Grids} shows a graphical representation of three
types of spatial grids for the region $0 \le x \le L$.  We divide
this region into spatial \emph{cells} (the spaces between vertical
lines) and functions are evaluated at $N$ discrete \emph{grid
points} (the dots). In a \emph{cell-edge} grid,\index{Cell-edge
grid}\index{Grids!cell-edge} the grid points are located at the
edge of the cell.  In a \emph{cell-center} grid,\index{Cell-center
grid}\index{Grids!cell-center} the points are located in the middle
of the cell.  Another useful grid is a cell-center grid with {\it
ghost points}.\index{Ghost points} The ghost points (unfilled dots)
are extra grid points on either side of the interval of interest
and are useful when we need to consider the derivatives at the edge
of a grid.


%\begin{example}
%
% Make a Python script that creates a
%  cell-edge spatial grid in the variable {\tt x} over the interval $0 \le x \le \pi$ with $N=500$.
%  \marginfig{Figures/f01p1a}{Plot from \ref{P:1.1a}} 
%    Plot the function $y(x) = \sin(x) \sinh(x)$ on this grid.
%    Explain the relationship between the number of cells and
%    the number of grid points in a cell-edge grid. Then
%    verify that the number of points in this $x$-grid is $N$
%    (using Python's {\tt len} command).
%
%\textit{Solution:}\\
%\begin{codeexample}
%\begin{VerbatimOut}{\listingFile}
%from numpy import linspace,pi,sin,sinh  # Import the needed function
%from matplotlib import pyplot
%N=500             # the number of grid points
%a=0
%b=pi          # the left and right bounds
%x=linspace(a,b,N)           # build the grid
%print(x)
%y = sin(x) * sinh(x)
%print(len(x))
%pyplot.plot(x,y)
%pyplot.show()
%\end{VerbatimOut}
%\end{codeexample}
%\end{example}
\begin{enumerate}
\probtwo \label{P:1.1}
\begin{enumerate}
  \subprob \label{P:1.1a} Make a Python script that creates a
  cell-edge spatial grid in the variable {\tt x} over the interval $0 \le x \le \pi$ with $N=500$.
  \marginfig{Figures/f01p1a}{Plot from \ref{P:1.1a}} 
    Plot the function $y(x) = \sin(x) \sinh(x)$ on this grid.
    Explain the relationship between the number of cells and
    the number of grid points in a cell-edge grid. Then
    verify that the number of points in this $x$-grid is $N$
    (using Python's {\tt len} command).\\
\ifsolutions
\textit{Solution:}\\
\begin{codeexample}
\begin{VerbatimOut}{\listingFile}
from numpy import linspace,pi,sin,sinh  # Import the needed function
from matplotlib import pyplot
N=500             # the number of grid points
a=0
b=pi          # the left and right bounds
x=linspace(a,b,N)           # build the grid
print x
y = sin(x) * sinh(x)
print len(x)
pyplot.plot(x,y)
pyplot.show()
\end{VerbatimOut}
\end{codeexample}
\else
\noindent\rule{4 in}{0.01 in}
\fi

    \subprob \label{P:1.1b} Now let's do a cell-centered grid.  Notice
    that your grid doesn't start at \texttt{a} or end at \texttt{b}.
    Rather, your grid starts at $a + \frac{dx}{2}$ and ends at $b -
    \frac{dx}{2}$, where $dx$ is the grid spacing.  This means that
    you will need to add a line that calculates the grid spacing.
    Think about this and add the needed line to calculate it.  Verify
    that your grid points are correct and that there are still $500$ of them.\\
\ifsolutions
\textit{Solution:}\\
\begin{codeexample}
\begin{VerbatimOut}{\listingFile}
from numpy import linspace, arange,pi,sin, sinh

from matplotlib import pyplot

a = 0
b = pi
N = 10.
dx = (b  - a)/ N  # grid seperation
print dx

x = arange(a+dx/2,b+dx/2,dx)
print x
y = sin(x) * sinh(x)
print y
print len(x)
pyplot.plot(x,y)
pyplot.show()
\end{VerbatimOut}
\end{codeexample}
\else
\noindent\rule{4 in}{0.01 in}
\fi
    \subprob Now write a script like the one in part (b) to build a cell-center
    grid over the interval $0 \le x \le 2$ with $N=5000$. Evaluate the
    function $f(x)=\cos{x}$ on this grid and plot this function. Then
    estimate the area under the curve by summing the products of the
    centered function values $f_j$ with the widths of the cells $h$
    like this (midpoint integration rule):
\begin{Verbatim}
sum(f)*h;
\end{Verbatim}

    Verify that this result is quite close to the exact
    answer obtained by integration:
    \[
        A=\int_0^2 \cos{x} ~dx.
    \]\\
\ifsolutions
\textit{Solution:}\\
\begin{codeexample}
\begin{VerbatimOut}{\listingFile}
from numpy import arange,cos
from matplotlib import pyplot

a = 0
b = 2
N = 5000.
dx = (b  - a)/ N  # grid seperation
print dx

x = arange(a+dx/2,b+dx/2,dx)
print x
y = cos(x)
print y
print len(x)
area = sum(y) * dx
print area
pyplot.plot(x,y)
pyplot.show()
\end{VerbatimOut}
\end{codeexample}
\else
\noindent\rule{4 in}{0.01 in}
\fi
\marginfig{Figures/f01p1b}{Plot from \ref{P:1.1b}}

\end{enumerate}
\end{enumerate}

%Python has a convenient command {\tt linspace} for building one
%dimensional grids.  The syntax for building a grid is
%\begin{Verbatim}
%x = linspace(a,b,N);
%\end{Verbatim}
%where {\tt a} is the $x$ position of the first point in the grid,
%{\tt b} is the $x$-position of the last point in the grid, and {\tt
%N} is the number of grid points.  This method doesn't give you the
%grid spacing back, but you can always calculate it by subtraction:
%\begin{Verbatim}
%h = x(2) - x(1);
%\end{Verbatim}
%Depending on what you choose for {\tt a} and {\tt b}, {\tt linspace}
%can give you either cell-edge or cell-center grids.

\labsection{Interpolation and extrapolation}
\index{Interpolation} \index{Extrapolation}

Grids only represent functions at discrete points, and there will be
times when we want to find good values of a function {\it between}
grid points (interpolation) or \emph{beyond} the last grid point
(extrapolation). We will use interpolation and extrapolation
techniques fairly often during this course, so let's review these
ideas.

\marginfig{Figures/f01Linear}{The line defined by two points can be used to
interpolate between the points and extrapolate beyond the points.}

The simplest way to estimate these values is to use the fact that two
points define a straight line. For example, suppose that we have
function values $(x_1,y_1)$ and $(x_2,y_2)$. The formula for a
straight line that passes through these two points is
\begin{equation} \label{eq:linear}
    y-y_1 = \frac{ (y_2-y_1) }{ (x_2-x_1) } (x-x_1)
\end{equation}
Once this line has been established it provides an approximation to
the true function $y(x)$ that is pretty good in the neighborhood of
the two data points. To linearly interpolate or extrapolate we simply
evaluate Eq.~(\ref{eq:linear}) at $x$ values between or beyond $x_1$
and $x_2$.

\begin{enumerate}
\probtwo \label{P:1.3} Use Eq.~(\ref{eq:linear}) to do the following
    special cases:

\begin{enumerate}
\subprob Find an approximate value for $y(x)$ halfway between
    the two points $x_1$ and $x_2$. Does your answer make
    sense?\\
\ifsolutions
\textit{Solution:}
\begin{equation}
x = x_1 + {1 \over 2} (x_2 - x_1)
\end{equation}
\begin{align}
    y-y_1 &= \frac{ (y_2-y_1) }{ (x_2-x_1) } (x_1 + {1 \over 2} (x_2 -
    x_1)-x_1)\\
y &= \frac{ (y_2-y_1) }{ (x_2-x_1) } ({1 \over 2} (x_2 -
x_1)) + y_1\\
y &= y_1 + {1 \over 2} (y_2-y_1) \\
\end{align}

\fi
\subprob Find an approximate value for $y(x)$ 3/4 of the way
    from $x_1$ to $x_2$. Do you see a pattern?\\
\ifsolutions
\textit{Solution:}
\begin{align}
x &= x_1 + {3\over 4}(x_2 - x_1)\\
\end{align}
\begin{align}
    y-y_1 &= \frac{ (y_2-y_1) }{ (x_2-x_1) } (x_1 + {3 \over 4} (x_2 -
    x_1)-x_1)\\
y &= \frac{ (y_2-y_1) }{ (x_2-x_1) } ({3 \over 4} (x_2 -
x_1)) + y_1\\
y &= y_1 + {3 \over 4} (y_2-y_1) \\
\end{align}

\fi

Note: You should find that 
\begin{equation}
y(x_1 + p h) = y_1 + p (y_2 - y_1)
\end{equation}
, where $p$ is a fraction of the grid spacing. (${1\over 2}$ for part
a and ${3\over 4}$ for part b.)
\subprob If the spacing between grid points is $h$ (i.e.
    $x_2-x_1=h$), show that the linear extrapolation formula
    for $y(x_2+h)$ is
    \begin{equation}\label{eq:linExtrap}
        y(x_2+h) = 2 y_2 - y_1
    \end{equation}
    This provides a convenient way to estimate the function
    value one grid step beyond the last grid point. \\ 
\ifsolutions
\textit{Solution:}
Start with equation \eqref{eq:linear}
\begin{align} 
    y-y_1 &= \frac{ (y_2-y_1) }{ (x_2-x_1) } (x-x_1)\\
y &= \frac{ (y_2-y_1) }{ h } (x_2 + h -x_1) + y_1\\
&= \frac{ (y_2-y_1) }{ h } (h  + h) + y_1\\
&= \frac{ (y_2-y_1) }{ h } 2h + y_1\\
&= 2(y_2-y_1) + y_1\\
&= 2y_2-2y_1 + y_1\\
&= 2y_2-y_1\\
\end{align}
\fi

\subprob Also show that
    \begin{equation}\label{eq:linExtraphalf}
        y(x_2+{ h\over 2}) = {3\over 2} y_2 - {1\over 2}y_1 .
    \end{equation}
    We will use both of these formulas during the course.\\
\ifsolutions
Start with equation \eqref{eq:linear}
\begin{align} 
    y-y_1 &= \frac{ (y_2-y_1) }{ (x_2-x_1) } (x-x_1)\\
y &= \frac{ (y_2-y_1) }{ h } (x_2 + {h \over 2} -x_1) + y_1\\
&= \frac{ (y_2-y_1) }{ h } (h  + {h\over 2}) + y_1\\
&= {3\over 2}(y_2-y_1) + y_1\\
&= {3\over 2} y_2- {3\over 2} y_1 + y_1\\
&= {3\over 2} y_2-{1\over 2} y_1\\
\end{align}

\fi
\noindent\rule{4 in}{0.01 in}

\end{enumerate}
\end{enumerate}

\marginfig{Figures/f01Quadratic}{Three points define a parabola that can be
used to interpolate between the points and extrapolate beyond the
points.}

A fancier technique for finding values between and beyond grid points
is to use a parabola instead of a line. It takes three data points to
define a parabola, so we need to start with the function values
$(x_1,y_1)$, $(x_2,y_2)$, and $(x_3,y_3)$. The general formula for a
parabola is
\begin{equation}\label{eq:Parabola}
    y=a + bx + cx^2
\end{equation}
where the coefficients $a$, $b,$ and $c$ need to be chosen so that
the parabola passes through our three data points. To determine these
constants, you set up three equations that force the parabola to
match the data points, like this:
\begin{equation}\label{eq:ParabolaSet}
    y_j = a + bx_j + cx_j^2
\end{equation}
with $j=1,2,3$, and then solve for $a$, $b$, and $c$.

\begin{enumerate}
\probtwo \label{P:1.4} Use Eq.~(\ref{eq:ParabolaSet}) to create a
    set of three equations in Mathematica. For simplicity, assume
    that the points are on an evenly-spaced grid and set
    $x_2=x_1+h$ and $x_3=x_1+2h$.  Solve this set of equations to
    obtain some messy formulas for $a$, $b$, and $c$ that involve
    $x_1$, $y_1$, $y_2$, $y_3$, and $h$. Then use these formulas to solve the following
    problems:

\begin{enumerate}

\subprob Estimate $y(x)$ half way between $x_1$ and $x_2$,
    and then again halfway between $x_2$ and $x_3$. Do you
    see a pattern? (You will need to simplify the answer that
    Mathematica spits out to see the pattern.)

\subprob Show that the quadratic extrapolation formula for
    $y(x_3+h)$ (i.e. the value one grid point beyond $x_3$)
    is
    \begin{equation}\label{eq:quadExtrap}
        y(x_3+h) = y_1 - 3 y_2 + 3 y_3
    \end{equation}
    Also find the formula for $y(x_3+h/2)$.
\ifsolutions
Answer:  You should find that the interpolated value between $x_1$ and
$x_2$ is
\begin{equation}
{1\over 8} \left( 3 y_1 + 6 y_2 - y_3\right)
\end{equation}
and the value in between $x_2$ and $x_3$ is
\begin{equation}
{1\over 8} \left( - y_1 + 6 y_2 + 3 y_3\right)
\end{equation}

\fi
\end{enumerate}
\end{enumerate}

\labsection{Homework}
\begin{enumerate}
\prob 
\begin{enumerate}
\item Build a cell-centered grid in the variable {\tt x} over the
interval $0 \le x \le \pi$ with $N=500$.
\item Evaluate the function $\sinh(x)$ on the grid.
\item Use linear extrapolation to determine the value of the function
  one grid point beyond the last one.  Compare to the true value of the
  function at that point.
\item Use quadratic extrapolation to determine the value of the function
  one grid point beyond the last one.  Compare to the true value of
  the function at that point.
\item Use linear interpolation to determine the value of the function
  between the last and second-to-last grid point.  Compare to the true
  value of the function.
\item Use quadratic interpolation to determine the value of the
  function between the last and second-to-last grid point.  Compare to
  the true value of the function.
\ifsolutions
\textit{Solution:}\\
\begin{codeexample}
\begin{VerbatimOut}{\listingFile}
from numpy import sinh,arange
from matplotlib import pyplot

a = 0
b = 2
N = 100.
dx = (b  - a)/ N  # grid seperation

x = arange(a+dx/2,b+dx/2,dx)
y = sinh(x)

yExtrapLinear = 2 * y[-1] - y[-2]
yExtrapQuad = y[-3] - 3 * y[-2] + 3 * y[-1]
yExtrapExact = sinh(x[-1] + dx)
print yExtrapLinear, yExtrapQuad, yExtrapExact
yInterpLinear =  y[-2] + 1/2. * (y[-1] - y[-2])
yInterpQuad = 1/8. * (-y[-3] + 6 * y[-2] + 3 * y[-1])
yInterpExact = sinh(x[-2] + dx/2.)
print yInterpLinear, yInterpQuad, yInterpExact
\end{VerbatimOut}
\end{codeexample}
\else
\noindent\rule{4 in}{0.01 in}
\fi
\end{enumerate}
\end{enumerate}
